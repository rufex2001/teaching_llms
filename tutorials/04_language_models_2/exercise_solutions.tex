\documentclass[11pt,a4paper]{article}

% packages
\usepackage[utf8]{inputenc}
\usepackage{amsmath}
\usepackage[T1]{fontenc}
\usepackage{setspace}
\usepackage{enumitem}
\usepackage{amsmath}
\usepackage{booktabs}
\usepackage{fullpage} 
\usepackage{tabularx}
\usepackage{amssymb, amstext, amsmath}
\usepackage{fancyhdr}
\usepackage{graphicx}
\usepackage{algorithmic}
\usepackage[ruled,vlined]{algorithm2e}
\usepackage{url}
\usepackage[bookmarks,unicode=true,pdftex,a4paper]{hyperref}
\usepackage[round]{natbib}
\usepackage[usenames,dvipsnames]{color, xcolor}
\headsep1cm

% macros
% misc
\newcommand\todo[1]{\textcolor{red}{TODO: #1}}
\newcommand\hide[1]{\textcolor{white}{#1}}

% formatting
\newcommand\bld[1]{\textbf{#1}}
\newcommand\ul[1]{\underline{#1}}
\newcommand\n[1]{\numprint{#1}}
\newcommand{\ts}{\textsuperscript}
\newcommand\red[1]{\textcolor{red}{#1}}
\newcommand\blue[1]{\textcolor{blue}{#1}}
\newcommand\link[2]{\href{#1}{\textcolor{blue}{\underline{#2}}}}

% sets
\newcommand\set[1]{\mathcal{#1}}
\newcommand\bb[1]{\mathbb{#1}}
\renewcommand\:{\colon} % for use with \sset, etc.
\newcommand{\sset}[1]{\left\{\,#1\,\right\}} % { ? }, automatic brackets
\newcommand{\ssets}[1]{\left\{#1\right\}} % {?}, automatic brackets
\newcommand{\ssetn}[1]{\{\,#1\,\}} % { ? }, normal brackets

% table formatting
% To better align bold entries in S columns (still broken)
% \usepackage{siunitx}
% \robustify\bfseries
% \newrobustcmd{\bfcell}{\bfseries}

% vector variables (taken from macros by Rainer Gemulla)
\newcommand\vect[1]{{\boldsymbol{#1}}}
\newcommand\va{\vect{a}}
\newcommand\vb{\vect{b}}
\newcommand\vc{\vect{c}}
\newcommand\vd{\vect{d}}
\newcommand\ve{\vect{e}}
\newcommand\vf{\vect{f}}
\newcommand\vg{\vect{g}}
\newcommand\vh{\vect{h}}
\newcommand\vi{\vect{i}}
\newcommand\vj{\vect{j}}
\newcommand\vk{\vect{k}}
\newcommand\vl{\vect{l}}
\newcommand\vm{\vect{m}}
\newcommand\vn{\vect{n}}
\newcommand\vo{\vect{o}}
\newcommand\vp{\vect{p}}
\newcommand\vq{\vect{q}}
\newcommand\vr{\vect{r}}
\newcommand\vs{\vect{s}}
\newcommand\vt{\vect{t}}
\newcommand\vu{\vect{u}}
\newcommand\vv{\vect{v}}
\newcommand\vw{\vect{w}}
\newcommand\vx{\vect{x}}
\newcommand\vy{\vect{y}}
\newcommand\vz{\vect{z}}
\newcommand\vzero{\vect{0}}
\newcommand\vone{\vect{1}}

\newcommand\valpha{\vect{\alpha}}
\newcommand\vbeta{\vect{\beta}}
\newcommand\veps{\vect{\epsilon}}
\newcommand\vdelta{\vect{\delta}}
\newcommand\vtheta{\vect{\theta}}
\newcommand\vsigma{\vect{\sigma}}
\newcommand\vpi{\vect{\pi}}
\newcommand\vlambda{\vect{\lambda}}

% matrix variables (taken from macros by Rainer Gemulla)
\newcommand\mA{\vect{A}}
\newcommand\mB{\vect{B}}
\newcommand\mC{\vect{C}}
\newcommand\mD{\vect{D}}
\newcommand\mE{\vect{E}}
\newcommand\mF{\vect{F}}
\newcommand\mG{\vect{G}}
\newcommand\mH{\vect{H}}
\newcommand\mI{\vect{I}}
\newcommand\mJ{\vect{J}}
\newcommand\mK{\vect{K}}
\newcommand\mL{\vect{L}}
\newcommand\mM{\vect{M}}
\newcommand\mN{\vect{N}}
\newcommand\mO{\vect{O}}
\newcommand\mP{\vect{P}}
\newcommand\mQ{\vect{Q}}
\newcommand\mR{\vect{R}}
\newcommand\mS{\vect{S}}
\newcommand\mT{\vect{T}}
\newcommand\mU{\vect{U}}
\newcommand\mV{\vect{V}}
\newcommand\mW{\vect{W}}
\newcommand\mX{\vect{X}}
\newcommand\mY{\vect{Y}}
\newcommand\mZ{\vect{Z}}
\newcommand\mzero{\vect{0}}

\newcommand{\mPi}{{\ensuremath{\vect{\Pi}}}}
\newcommand{\mSigma}{{\ensuremath{\vect{\Sigma}}}}
\newcommand{\mLambda}{{\ensuremath{\vect{\Lambda}}}}

% argmin, argmax
\DeclareMathOperator*{\argmin}{argmin} % amsmath package required
\DeclareMathOperator*{\argmax}{argmax} % amsmath package required

% matrix operations
\newcommand\xdiag{\operatorname{diag}}      
\newcommand\diag[1]{\xdiag\left(#1\right)}    % diagonal matrix


% header and footer
\lhead{Advanced Methods in Text Analytics, FSS2025}
\chead{}
\rhead{\thepage\ }
\cfoot{}
\pagestyle{fancy}

\title{Advanced Methods in Text Analytics \\ 
Exercise 4: Language Models - Part 2 \\
\textbf{Solutions}}
\author{Daniel Ruffinelli}
\date{FSS 2025}

\begin{document}
\maketitle

\section{N-Gram Language Models}

\begin{enumerate}[label=(\alph*)]
    \item See Jupyter notebook.
    \item See Jupyter notebook.
    \item See Jupyter notebook.
    \item Most popular n-grams are those that have padding on the left.
          This is because of the way we construct examples.
          More meaningful n-grams are found lower in the ranking of most common
          n-grams.
    \item The output looks very much like Shakespeare's work, but it is not
          actually meaningful most of the time.
          This has to do with our model's inability to capture patterns
          between more distant words, with our sampling approach, or both.
          Consequently, short chunks of text make sense, e.g.\ a subsequence of
          roughly the size of $n$.
    \item The output is still as before, with most generated sequences seeming
          like Shakespeare's work, but not being quite meaningful.
          The output does change considerably with different runs, mostly due to
          the way we sample.
          Different prompts also have a considerable impact in how meaningful
          the generated text appears.
    \item Recall that perplexity can be interpreted as proportional to the size
          of our vocabulary.
          In that sense, the reported perplexity is quite high, as it's below
          but close to 50\% of the vocabulary size.
          However, without a proper baseline, it's always difficult (or
          impossible) to determine how well a model performs.
\end{enumerate}

\section{Language Models with Fully-Connected Neural Networks}

\begin{enumerate}[label=(\alph*)]
    \item See Jupyter notebook.
    \item See Jupyter notebook.
    \item See Jupyter notebook.
    \item Perplexity is now much lower than when using an n-gram LM before.
          However, it is not exactly fair to make this comparison, as the set of
          examples used to compute the likelihood of the data are not the same.
          With the n-gram LM, we predicted every word in the validation dataset
          using the last two words only, which is what that model can do.
          Here, however, we split the validation set into sequences of a fixed
          length and then just computed the likelihood of the last words in each
          sequence given all previous ones in the sequence.
          This is something that the n-grams model cannot do, but we should be
          able to compute likelihood as done with the n-gram model using this
          neural LM.
          Instead, we'll compare it with an RNN-based LM soon.
\end{enumerate}

\section{Language Models with RNNs}

\begin{enumerate}[label=(\alph*)]
    \item It is more expensive to train due to the computation done by the
          network, which is considerably more involved than the simple forward
          pass of the neural LM from before.
          In addition, for a given input sequence, we are constructing many
          more training examples, which we average over.
    \item Perplexity is much lower compared to the neural LM using
          fully-connected neural networks.
          The RNN likely takes advantage of its architecture, designed to keep
          track of patterns across input sequences, but as mentioned before, it
          does use more training examples compared to the neural LM from before,
          which may account for some of this improvement.
          The neural LM can in principle be trained in this way as well, all we
          need for that is construct the training examples accordingly.
    \item The text generated with our n-gram models from Task 1 actually seems
          of higher quality, despite the RNN showing lower perplexity on
          validation data, and an inductive bias designed for processing
          sequences of tokens.
          This is likely because of the difference in sampling approaches.
          Here, we are using random sampling, i.e.\ sampling from the entire
          distribution over our vocabulary.
          This explains why we do sample tokens that aren't necessarily
          frequent, e.g.\ certain punctuation marks.

          In Task 1, however, we sample from a small set of words in the
          vocabulary, not from the entire vocabulary.
          This small set of words is given by the words the n-gram model has
          seen in the same context during training, and even with add-1
          smoothing, the vast majority of the probability mass is in the words
          seen next to each other during training.
          This is quite a strong bias and likely not good for generalization,
          but it does produce better quality text in this case compared to
          using random sampling.

          In addition, the quality of the generated text is highly dependent on
          the temperature used with the softmax function.
          Higher temperatures mean that the softmax distribution becomes more
          uniform, whereas lower temperatures mean that the mass in the softmax
          distribution focuses more on elements which already have more mass.
          In this case, with a temperature value of 1.5, we can generate text
          that seems of much higher quality compared to using a temperature of
          10 or 0.1.
          However, even in the best cases, the results are still sentences that
          seem \emph{shakespearean}, but aren't quite meaningful.
    \item The choice of sampling method has a clear impact on the generated
          text.
          With some combinations of temperature and $k$, e.g.\ with temperature
          set to 2.2 and $k$ to 100, the output can sometimes be similar to what
          the n-grams produced, but this is often not the case, especially with
          other values for temperature and $k$.
          The model often outputs names of characters or other words outside of
          dialogue, e.g. `SONNETS' and similar words in all-caps, as well as
          punctuation marks in unusual places.
          It is unclear why this is the case, but it may be that the model is
          already overfitting to the corpus, which would explain why it
          produces words that pertain to the format rather than actual dialogue.

          Such issues could be addressed by better pre-processing the data
          before training (e.g. remove character names, unless we want a model
          to generate scripts), or by applying different forms of
          regularization.
          These decisions are task-dependent.
          More generally, language models often generate text in undesired ways,
          which is why fine-tuning them on a more specific task is often done
          after pre-traning.

          This issue also nicely illustrates the difference between intrinsic
          evaluation (in this case, perplexity) and extrinsic evaluation (in
          this case, text generation).
          While we are holistically and subjectively evaluating the quality of
          the generated text, it is quite clear that the generated text from the
          RNN model is of inferior quality, despite its much lower perplexity.
          In short, it can happen that improvements in an intristic evaluation
          metric do not translate to better performance in an extrinsic
          evaluation task.
    \item When training for 20 epochs as done with the FNN, the RNN achieves a
          PPL of about 350.
          We can see from PPL on training data that the model is indeed
          converging, as the PPL decreased slower in the last epochs.
          The model is not fully converged yet, so there is potential room for
          improvement, but the remaining improvement is likely not significant.
\end{enumerate}

\end{document}
