\documentclass[11pt,a4paper]{article}

% packages
\usepackage[utf8]{inputenc}
\usepackage{amsmath}
\usepackage[T1]{fontenc}
\usepackage{setspace}
\usepackage{enumitem}
\usepackage{booktabs}
\usepackage{fullpage} 
\usepackage{tabularx}
\usepackage{amssymb,amstext,amsmath}
\usepackage{fancyhdr}
\usepackage{graphicx}
\usepackage{algorithmic}
\usepackage[ruled,vlined]{algorithm2e}
\usepackage{url}
\usepackage[colorlinks=true,bookmarks,unicode=true,pdftex,a4paper]{hyperref}
\usepackage[round]{natbib}
\usepackage[usenames,dvipsnames]{color, xcolor}
\headsep1cm

% macros
% misc
\newcommand\todo[1]{\textcolor{red}{TODO: #1}}
\newcommand\hide[1]{\textcolor{white}{#1}}

% formatting
\newcommand\bld[1]{\textbf{#1}}
\newcommand\ul[1]{\underline{#1}}
\newcommand\n[1]{\numprint{#1}}
\newcommand{\ts}{\textsuperscript}
\newcommand\red[1]{\textcolor{red}{#1}}
\newcommand\blue[1]{\textcolor{blue}{#1}}
\newcommand\link[2]{\href{#1}{\textcolor{blue}{\underline{#2}}}}

% sets
\newcommand\set[1]{\mathcal{#1}}
\newcommand\bb[1]{\mathbb{#1}}
\renewcommand\:{\colon} % for use with \sset, etc.
\newcommand{\sset}[1]{\left\{\,#1\,\right\}} % { ? }, automatic brackets
\newcommand{\ssets}[1]{\left\{#1\right\}} % {?}, automatic brackets
\newcommand{\ssetn}[1]{\{\,#1\,\}} % { ? }, normal brackets

% table formatting
% To better align bold entries in S columns (still broken)
% \usepackage{siunitx}
% \robustify\bfseries
% \newrobustcmd{\bfcell}{\bfseries}

% vector variables (taken from macros by Rainer Gemulla)
\newcommand\vect[1]{{\boldsymbol{#1}}}
\newcommand\va{\vect{a}}
\newcommand\vb{\vect{b}}
\newcommand\vc{\vect{c}}
\newcommand\vd{\vect{d}}
\newcommand\ve{\vect{e}}
\newcommand\vf{\vect{f}}
\newcommand\vg{\vect{g}}
\newcommand\vh{\vect{h}}
\newcommand\vi{\vect{i}}
\newcommand\vj{\vect{j}}
\newcommand\vk{\vect{k}}
\newcommand\vl{\vect{l}}
\newcommand\vm{\vect{m}}
\newcommand\vn{\vect{n}}
\newcommand\vo{\vect{o}}
\newcommand\vp{\vect{p}}
\newcommand\vq{\vect{q}}
\newcommand\vr{\vect{r}}
\newcommand\vs{\vect{s}}
\newcommand\vt{\vect{t}}
\newcommand\vu{\vect{u}}
\newcommand\vv{\vect{v}}
\newcommand\vw{\vect{w}}
\newcommand\vx{\vect{x}}
\newcommand\vy{\vect{y}}
\newcommand\vz{\vect{z}}
\newcommand\vzero{\vect{0}}
\newcommand\vone{\vect{1}}

\newcommand\valpha{\vect{\alpha}}
\newcommand\vbeta{\vect{\beta}}
\newcommand\veps{\vect{\epsilon}}
\newcommand\vdelta{\vect{\delta}}
\newcommand\vtheta{\vect{\theta}}
\newcommand\vsigma{\vect{\sigma}}
\newcommand\vpi{\vect{\pi}}
\newcommand\vlambda{\vect{\lambda}}

% matrix variables (taken from macros by Rainer Gemulla)
\newcommand\mA{\vect{A}}
\newcommand\mB{\vect{B}}
\newcommand\mC{\vect{C}}
\newcommand\mD{\vect{D}}
\newcommand\mE{\vect{E}}
\newcommand\mF{\vect{F}}
\newcommand\mG{\vect{G}}
\newcommand\mH{\vect{H}}
\newcommand\mI{\vect{I}}
\newcommand\mJ{\vect{J}}
\newcommand\mK{\vect{K}}
\newcommand\mL{\vect{L}}
\newcommand\mM{\vect{M}}
\newcommand\mN{\vect{N}}
\newcommand\mO{\vect{O}}
\newcommand\mP{\vect{P}}
\newcommand\mQ{\vect{Q}}
\newcommand\mR{\vect{R}}
\newcommand\mS{\vect{S}}
\newcommand\mT{\vect{T}}
\newcommand\mU{\vect{U}}
\newcommand\mV{\vect{V}}
\newcommand\mW{\vect{W}}
\newcommand\mX{\vect{X}}
\newcommand\mY{\vect{Y}}
\newcommand\mZ{\vect{Z}}
\newcommand\mzero{\vect{0}}

\newcommand{\mPi}{{\ensuremath{\vect{\Pi}}}}
\newcommand{\mSigma}{{\ensuremath{\vect{\Sigma}}}}
\newcommand{\mLambda}{{\ensuremath{\vect{\Lambda}}}}

% argmin, argmax
\DeclareMathOperator*{\argmin}{argmin} % amsmath package required
\DeclareMathOperator*{\argmax}{argmax} % amsmath package required

% matrix operations
\newcommand\xdiag{\operatorname{diag}}      
\newcommand\diag[1]{\xdiag\left(#1\right)}    % diagonal matrix


% header and footer
\lhead{Advanced Methods in Text Analytics, FSS2025}
\chead{}
\rhead{\thepage\ }
\cfoot{}
\pagestyle{fancy}

\title{Advanced Methods in Text Analytics \\ 
Exercise 4: Language Models - Part 2}
\author{Daniel Ruffinelli}
\date{FSS 2025}

\begin{document}
\maketitle

The main goal of this exercise is to implement the fundamental components of
language models using \href{https://pytorch.org/}{\underline{PyTorch}}, so that
when you use higher-level libraries, you have an intuition of what is happening
underneath.
To this end, we provide a Jupyter notebook with incomplete implementations of
three types of language models, from n-grams to basic deep learning models.
You will be asked to complete the missing code throughout the tasks, which were
designed to make you think about how different components interact with each
other, how the data is shaped/shifted/transformed throughout the pipeline, and
how models are trained from data to make useful predictions and be used in
applications.
So, follow the instructions in each question plus the documentation in the code
when implementing tasks in this exercise.

\section{N-Gram Language Models}

To implement this model, we make use of a simple tokenizer defined in the
function \emph{tokenize}.
It first separates punctuation marks by whitespace, and then creates tokens by
splitting the string with whitespaces.

\begin{enumerate}[label=(\alph*)]
    \item Complete the function \emph{compute\_ngrams}, which we will use to
          count n-grams.
          To make predictions with n-gram models, we compute the probability of
          a sequence of n words and compare it against the probability of the
          same sequence minus the last word, i.e. the $n-1$ words sequence.
          Thus, instead of representing n-grams as $n$-size tuples, we will
          represent them as tuples of the form \emph{(context, next\_word)},
          where \emph{context} is the first $ n-1$ tokens in a given sequence,
          and \emph{next\_word} is the last token.
    \item We implement n-gram models with the \emph{NGramModel} class.
          Following the format of n-grams defined above, this model keeps track
          of \emph{(context, next\_word)} pairs and an n-gram counter.
          Complete the functions \emph{update} and \emph{next\_word\_prob} by
          following the documentation in the code.
          In this class, we make use of
          \href{https://docs.python.org/3/library/collections.html#collections.defaultdict}{\underline{defaultdicts}}
          in Python, but feel free to replace them with standard dictionaries if
          you prefer.
    \item Complete the function \emph{predict\_next\_word} used to predict the
          next word given a prompt and an n-gram model implemented with the
          \emph{NGramModel} class above.
          For sampling the next word, use the probabilities of each possible
          next word given by the model.
          Use the model's ``context to next word'' dictionary and its
          \emph{next\_word\_prob} function to construct such a distribution.
    \item Let's test the code we worked with in the questions above.
          First, we load our Shakespeare dataset, which is nothing more than
          all of Shakespeare's work, obtained from
          \href{https://www.gutenberg.org/}{\underline{Project Gutenberg}}.
          Then, we create a trigram model, count the n-grams in the tokenized
          corpus, and then check the most common ngrams.
          Do the most common n-grams make sense? Why do they look like that?
    \item Let's generate text with our model.
          Complete the function \emph{generate\_text} and then test it with some
          prompts.
          Try different ones based on the n-grams counted in the corpus as
          prompts.
          Does the output make sense? Does it change a lot given the same input?
    \item Let's use stronger models now.
          Create a fourgram and fivegram model and, as before, count n-grams and
          then check for the most common ones in the corpus.
          Then, use some of those n-grams as prompts to generate text with
          these models.
          Are they better than the trigram model?
    \item Complete the function \emph{compute\_likelihood}, which we use
          to compute the likelihood and perplexity of the validation data.
          Note that we can compute the perplexity directly from the likelihood.
          For details, see
          \href{https://en.wikipedia.org/wiki/Perplexity}{\underline{here}}.
          Is the likelihood low? Is the perplexity low? How can you tell?
\end{enumerate}

\section{Language Models with Fully-Connected Neural Networks}

In this task, we use PyTorch (official tutorials
\href{https://pytorch.org/tutorials/}{here}) to implement neural LMs.
Specifically, we implement a language model similar to the one proposed by
\href{https://www.jmlr.org/papers/volume3/bengio03a/bengio03a.pdf}{\underline{Bengio et al.}}
For simplicity, we will not implement the optional skip connection between input
and output, but we will implement a more flexible model that can have multiple
hidden layers of different sizes.
This should allow us to test different architectures.
But first, we need to load up PyTorch and preprocess our dataset.
These models are different from ngram models, so we prepare our data for this,
including the creation of our vocabulary.

\begin{enumerate}[label=(\alph*)]
    \item We implement our LMs with fully-connected neural networks using the
          \emph{NeuralLM} class.
          Read the code and complete the \emph{forward} function, which computes
          the forward pass of the network.
          In this function, do not use softmax! We will do that during training.
          \textbf{Hint:} use
          \href{https://pytorch.org/docs/stable/generated/torch.Tensor.view.html}{\underline{view}}
          to get the necessary embeddings.
    \item A \href{https://pytorch.org/tutorials/beginner/basics/data_tutorial.html}{\underline{dataloader}}
          is an important component of using PyTorch, as in deep learning,
          loading data can be expensive.
          The \emph{dataloader} is in charge of that, but it needs two things: 
          (i) a dataset, and (ii) a \emph{collate} function, in charge of 
          turning raw data into training examples, i.e.\ an (input, target) 
          pair. 
          Complete the class \emph{SelfSupervisedTextDataset}, which is a custom
          PyTorch
          \href{https://pytorch.org/tutorials/beginner/data_loading_tutorial.html}{\underline{dataset}},
          following the instructions in the code documentation.
          Then, complete the \emph{collate\_fn}, which takes a batch as input
          during training, and processes it to create the training examples and
          corresponding targets. This is where self-supervision takes place!
    \item We now need to write a training loop, a fundamental part of
          implementing deep learning models.
          This is the loop that takes place when you call functions such as
          \emph{model.fit()} in higher-level libraries like \emph{scikit-learn}.
          Complete the function \emph{train} (ignore the \emph{rnn} flag for
          now).
          To implement the training loop, you need to iterate over the epochs,
          and over batches given by the dataloader.
          For each batch, you need to compute the forward pass, the backward
          pass, and update the parameters of your model using the optimizer.
          We use the
          \href{https://pytorch.org/docs/stable/generated/torch.nn.CrossEntropyLoss.html}{\underline{cross entropy}}
          loss, which expects logits (i.e.\ it handles the softmax computation),
          and the Adagrad optimizer.
          Also, and perhaps more specific to PyTorch, gradients are accumulated
          over different batches by default, so you need to zero them after each
          batch, using the \emph{optimizer.zero\_grad()} function.
    \item To compute perplexity on the validation data, we also need a loop for
          evaluation.
          Complete the function \emph{evaluate} (ignore the \emph{rnn} flag for
          now).
          Use the same likelihood as computed during training to then obtain the
          perplexity.
          The function should print the perplexity of the data at the end.
          As with the training loop, you need to iterate over the batches in the
          dataloader, but there are no epochs this time.
          In addition, in order to compute the perplexity over the entire data,
          we do not reduce the loss per examples but sum it.
          Thus, you also need to count the number of examples so we can then
          compute average likelihood and with it, perplexity.
          (Note that you should also do this during training if you want to see
          perplexity over the entire training set.)
          How is the perplexity on validation data? Can you compare it to the
          n-gram model from before?
    \item Try different architectures to see their impact on validation
          perplexity. You can change the number and size of hidden layers, but
          be careful with overfitting.
          Consider including different forms of regularization, such as
          \href{https://pytorch.org/docs/stable/generated/torch.nn.Dropout.html}{\underline{dropout}}.
\end{enumerate}

\section{Language Models with RNNs}

In this task, we focus on language models based on RNNs.

\begin{enumerate}[label=(\alph*)]
    \item We use the class \emph{RNNLM} to implement an RNN-based language
          model.
          Read the code and understand what the function
          \emph{init\_hidden} does. We will use it later for training.
          Then, complete the \emph{forward} function.
          Make sure you read PyTorch
          \href{https://pytorch.org/docs/stable/generated/torch.nn.RNN.html}{\underline{documentation on RNNs}}
          to understand the size of the input that the \emph{rnn} object
          expects.
          \emph{(input, target)} pairs are constructed differently from a given
          input sequence.
          To this end, complete the function \emph{rnn\_collate\_fn}
          accordingly.
          Second, we need to modify the training loop from the neural LM used
          before so it can \emph{also} train an RNN.
          You may use the \emph{rnn} flag for this, but if you prefer, create an
          entirely new training loop for this RNN in a new cell.
          Again, read the RNN documentation to understand both the expected
          input to the RNN and the output it produces.
          Is it more expensive to train than the neural model from before? Why?
    \item As before, we need to modify our evaluation loop so it supports RNNs.
          Use the \emph{rnn} flag and modify the \emph{evaluation} function in
          the neural LM from before to allow evaluation of RNN-based LMs.
          How does the perplexity compare to the models we tested before?
    \item Let's now generate text with our RNN.
          For that, we first create a reverse dictionary that maps token IDs to
          the actual tokens.
          Complete the function \emph{generate\_text\_with\_rnn} by sampling the
          next word using the distribution over the vocabulary provided by the
          model.
          Then use this function to generate new text using any prompt you want,
          so long as it uses the vocabulary our model understands.
          How does the generated text compare to the output of the n-gram-based
          model from Task 1? Can you think of a reason that would explain the
          differences you observe in quality?
          Test the generation of text using different temperatures.
          Does it get better or worse? When? Why do you think that is?
    \item To see the impact of different sampling methods on the generated text,
          we generalize the function \emph{generate\_text\_with\_rnn} so it 
          takes a \emph{sampling\_fn} parameter. Complete the function 
          \emph{topk\_sampling} so it returns a single sample from a given set
          of logits and a value of $k$. Note that the new version of 
          \emph{generate\_text\_with\_rnn} no longer computes softmax, because 
          each \emph{sampling\_fn} is now responsible for that. For that reason,
          the function \emph{topk\_sampling} now takes the temperature 
          parameter.
          Test the impact of different values of $k$ on the generated text. Make 
          sure to also try different values of $k$ with different temperatures.
          Did you find a combination of $k$ and temperature that generates text 
          similar to or better than the $n$-gram model from Task 1?
    \item Test the impact of using different architectures on validation
          perplexity and text generation. You may try different sizes of hidden
          layers, or even stack them. You may also implement an RNN with
          \href{https://pytorch.org/docs/stable/generated/torch.nn.LSTM.html}{\underline{LSTMs}},
          which would require minimal changes to your training and evaluation 
          code (you need to manage the cell state in addition to the hidden 
          state).
\end{enumerate}

\end{document}
