\documentclass[11pt,a4paper]{article}

% packages
\usepackage[utf8]{inputenc}
\usepackage{amsmath}
\usepackage[T1]{fontenc}
\usepackage{setspace}
\usepackage{enumitem}
\usepackage{amsmath}
\usepackage{booktabs}
\usepackage{fullpage} 
\usepackage{tabularx}
\usepackage{amssymb, amstext, amsmath}
\usepackage{fancyhdr}
\usepackage{graphicx}
\usepackage{algorithmic}
\usepackage[ruled,vlined]{algorithm2e}
\usepackage{url}
\usepackage[bookmarks,unicode=true,pdftex,a4paper]{hyperref}
\usepackage[round]{natbib}
\usepackage[usenames,dvipsnames]{color, xcolor}
\headsep1cm

% macros
% misc
\newcommand\todo[1]{\textcolor{red}{TODO: #1}}
\newcommand\hide[1]{\textcolor{white}{#1}}

% formatting
\newcommand\bld[1]{\textbf{#1}}
\newcommand\ul[1]{\underline{#1}}
\newcommand\n[1]{\numprint{#1}}
\newcommand{\ts}{\textsuperscript}
\newcommand\red[1]{\textcolor{red}{#1}}
\newcommand\blue[1]{\textcolor{blue}{#1}}
\newcommand\link[2]{\href{#1}{\textcolor{blue}{\underline{#2}}}}

% sets
\newcommand\set[1]{\mathcal{#1}}
\newcommand\bb[1]{\mathbb{#1}}
\renewcommand\:{\colon} % for use with \sset, etc.
\newcommand{\sset}[1]{\left\{\,#1\,\right\}} % { ? }, automatic brackets
\newcommand{\ssets}[1]{\left\{#1\right\}} % {?}, automatic brackets
\newcommand{\ssetn}[1]{\{\,#1\,\}} % { ? }, normal brackets

% table formatting
% To better align bold entries in S columns (still broken)
% \usepackage{siunitx}
% \robustify\bfseries
% \newrobustcmd{\bfcell}{\bfseries}

% vector variables (taken from macros by Rainer Gemulla)
\newcommand\vect[1]{{\boldsymbol{#1}}}
\newcommand\va{\vect{a}}
\newcommand\vb{\vect{b}}
\newcommand\vc{\vect{c}}
\newcommand\vd{\vect{d}}
\newcommand\ve{\vect{e}}
\newcommand\vf{\vect{f}}
\newcommand\vg{\vect{g}}
\newcommand\vh{\vect{h}}
\newcommand\vi{\vect{i}}
\newcommand\vj{\vect{j}}
\newcommand\vk{\vect{k}}
\newcommand\vl{\vect{l}}
\newcommand\vm{\vect{m}}
\newcommand\vn{\vect{n}}
\newcommand\vo{\vect{o}}
\newcommand\vp{\vect{p}}
\newcommand\vq{\vect{q}}
\newcommand\vr{\vect{r}}
\newcommand\vs{\vect{s}}
\newcommand\vt{\vect{t}}
\newcommand\vu{\vect{u}}
\newcommand\vv{\vect{v}}
\newcommand\vw{\vect{w}}
\newcommand\vx{\vect{x}}
\newcommand\vy{\vect{y}}
\newcommand\vz{\vect{z}}
\newcommand\vzero{\vect{0}}
\newcommand\vone{\vect{1}}

\newcommand\valpha{\vect{\alpha}}
\newcommand\vbeta{\vect{\beta}}
\newcommand\veps{\vect{\epsilon}}
\newcommand\vdelta{\vect{\delta}}
\newcommand\vtheta{\vect{\theta}}
\newcommand\vsigma{\vect{\sigma}}
\newcommand\vpi{\vect{\pi}}
\newcommand\vlambda{\vect{\lambda}}

% matrix variables (taken from macros by Rainer Gemulla)
\newcommand\mA{\vect{A}}
\newcommand\mB{\vect{B}}
\newcommand\mC{\vect{C}}
\newcommand\mD{\vect{D}}
\newcommand\mE{\vect{E}}
\newcommand\mF{\vect{F}}
\newcommand\mG{\vect{G}}
\newcommand\mH{\vect{H}}
\newcommand\mI{\vect{I}}
\newcommand\mJ{\vect{J}}
\newcommand\mK{\vect{K}}
\newcommand\mL{\vect{L}}
\newcommand\mM{\vect{M}}
\newcommand\mN{\vect{N}}
\newcommand\mO{\vect{O}}
\newcommand\mP{\vect{P}}
\newcommand\mQ{\vect{Q}}
\newcommand\mR{\vect{R}}
\newcommand\mS{\vect{S}}
\newcommand\mT{\vect{T}}
\newcommand\mU{\vect{U}}
\newcommand\mV{\vect{V}}
\newcommand\mW{\vect{W}}
\newcommand\mX{\vect{X}}
\newcommand\mY{\vect{Y}}
\newcommand\mZ{\vect{Z}}
\newcommand\mzero{\vect{0}}

\newcommand{\mPi}{{\ensuremath{\vect{\Pi}}}}
\newcommand{\mSigma}{{\ensuremath{\vect{\Sigma}}}}
\newcommand{\mLambda}{{\ensuremath{\vect{\Lambda}}}}

% argmin, argmax
\DeclareMathOperator*{\argmin}{argmin} % amsmath package required
\DeclareMathOperator*{\argmax}{argmax} % amsmath package required

% matrix operations
\newcommand\xdiag{\operatorname{diag}}      
\newcommand\diag[1]{\xdiag\left(#1\right)}    % diagonal matrix


% new commands
\newcommand\op[1]{\operatorname{#1}}

% header and footer
\lhead{Advanced Methods in Text Analytics, FSS 2025}
\chead{}
\rhead{\thepage\ }
\cfoot{}
\pagestyle{fancy}

\title{Advanced Methods in Text Analytics \\ 
Exercise 8: Large Language Models - Part 2\\
\textbf{Solutions}}
\author{Daniel Ruffinelli}
\date{FSS 2025}

\begin{document}
\maketitle

\section{Making Predictions with LLMs}

\begin{enumerate}[label=(\alph*)]
    \item
          \begin{itemize}
              \item GPT3-Neo 1.3B: vocab. size 50257, 24 layers with hidden size
                    2048, 24 layers, MLP up projection to 8192 and
                    \href{https://docs.pytorch.org/docs/stable/generated/torch.nn.GELU.html#torch.nn.GELU}{GELU activation}.
              \item Llama3.2-1B: vocab size 128256, 16 layers with hidden size
                    2048, MLP up projection to 8192 and
                    \href{https://docs.pytorch.org/docs/stable/generated/torch.nn.SiLU.html}{Swish activation}.
          \end{itemize}
    \item The output we get from the tokenizer (\texttt{\_\_call\_\_}) is a
          \texttt{BatchEncoding} object (read details
          \href{https://huggingface.co/docs/transformers/v4.51.3/en/internal/tokenization_utils#transformers.PreTrainedTokenizerBase.__call__}{here}).
          This is a dictionary with keys \texttt{input\_ids} and
          \texttt{attention\_mask}.
          The former contains the list of token ids used to encode the prompt,
          given as a tensor of size batch size $\times$ sequence length, where
          batch size is typically because we feed models multiple input
          sequences at once during training (i.e.\ in this case batch size = 1).
          So, computing the length of the last dimension gives us the number of
          tokens used to encode the prompt.

          The \texttt{attention\_mask} component is a tensor of the same shape,
          where each element is 1 if the corresponding token should attended to
          by the model, and 0 otherwise.
          This is commonly used for things like padding tokens, which we don't
          want the model to attend to, e.g. when computing training loss.
    \item The output we get from the model is a
          \texttt{CausalLMOutputWithPast} object (read details
          \href{https://huggingface.co/docs/transformers/en/main_classes/output#transformers.modeling_outputs.CausalLMOutputWithPast}{here}).
          This is a dictionary with keys \texttt{logits} and
          \texttt{past\_key\_values}.
          The former contains the logits of the model in a tensor of size
          batch size $\times$ input sequence length $\times$ vocab size.
          So, as expected from an autoregressive language model, we have a
          distribution over the vocabulary for each token in the input sequence,
          and we need the distribution corresponding to the last token to make
          our next-token prediction.

          The \texttt{past\_key\_values} components contains pre-computed keys
          and values in self-attention, useful to more efficiently decode tokens
          at each step (decoding step $n + 1$ uses many of the same keys and
          values used by the model in decoding step $n$). The object contains
          this information for each of the layers in the model.
    \item See code for implementation. We used sorting to keep the solution more
          general, but if the top $k$ elements is all you need, then using
          \href{https://docs.pytorch.org/docs/stable/generated/torch.topk.html}{torch.topk}
          is a more efficient solution.
          The top $k$ tokens make sense given the prompt.
          E.g. we can see that the top token is a new line token, possibly
          indicating the model wants to start a new sentence in a new line.
          Given the prompt \emph{Hello?}, we also see words like \emph{Hello},
          \emph{you}, \emph{are}, \emph{my}, etc.
          The same holds for both models.
          This function is useful for sampling, e.g.\ greedy, top $k$.
\end{enumerate}

\section{Prompting}

\begin{enumerate}[label=(\alph*)]
    \item All prompts we test in the code here are natural language questions,
          e.g.\ \emph{What is the capital of France?}.
          If we look at the top $k$ tokens, the answer is never the top token,
          nor is it in the top $k$ tokens (exception: GPT3 does have
          \emph{Paris} when asked the question above).
          As before, the top token in GPT3 is always a new line, and for both
          models, the top tokens seems similar to one another, e.g.\ words like
          \emph{what} or \emph{how}, but not necessarily similar to the expected
          answer.
          But even if the answer were in the top $k$ tokens, how could we use it
          to consistently make a good prediction?
          In factual questions, we want a deterministic answer.
          This is where prompting comes in.

          Autoregressive models are designed to generate sequences, so unless
          specifically asked for, it's not fair to ask models to give a single
          expected answer in the first token they produce.
          How do we specifically ask for that?
          One simple way that works since the days of GPT3 is an ICL-like
          prompt.
          Then the top token is the correct answer.
          Note that this too is brittle because of tokenization issues, e.g.
          Rome is often tokenized with a space, so we need to add the space to
          the prompt to see the correct answer, and this is not consistent
          across models or examples.
          Try to out!
    \item ICL prompts typically have three components: instructions of the task,
          set of demonstrations of the task, and the question we ask the model.
          The code produces prompts that clearly show these components.
          Note that in the code we have two different \emph{templates} to
          construct ICL prompts, and that using instructions is optional, as is
          typically the case in ICL settings.
          Finally, we are randomizing the set and order of demonstrations, so
          that running the code multiple times produces different prompts.
    \item Most tasks are not clear unless we provide demonstrations. This is
          true even if we asked humans to solve the tasks.
          When adding demonstrations, we see the models being able to predict
          the expected answer most of the time, but this is still (as always)
          brittle to the chosen prompt.
          In particular, we can see here that the choice of template makes a
          difference in the top token predicted, since one of the templates
          requires a leading whitespace, which can result in models producing
          a single token with the correct answer (recall that many words end
          up tokenized with a leadins whitespace).
          Additionally, the translating task is interesting because while
          Llama3.2 does support French officially, GPT-3 does not.
    \item Using natural language instructions does help in a zero-shot setting.
          For example, adding the question \emph{What is the capital of the
              following country?} results in both GPT3 and Llama returning the
          correct answer in the top token even without using demonstrations.
          However, the choice of prompt is important.
          For example, asking Llama to simply \emph{Translate to French} does
          not work, while asking it to
          \emph{Translate the following word to French} does.
    \item Given that we were already able to construct prompts that result in
          models understanding the task, either via demonstrations or natural
          language instructions, the main difference we see with an instruction
          tuned model is that it can better handle natural language questions.
          E.g. now when asked \emph{What is the capital of France?}, the top
          token is indeed the correct answer, which we did not get with previous
          models unless we used clever ICL prompts.
          Do you find any other differences? Try all of the ICL tasks provided
          in the code.
          More generally, we are still looking at top $k$ tokens and trying to
          use greedy decoding, but autoregressive mdoels are designed to
          generate sequences longer than a single token, and instruction tuned
          models (often referred to as \emph{chat} models) are designed to
          generate longer sequences in a more conversational way.
          So, we explore decoding longer sequences in the next task.
\end{enumerate}

\section{Generating Longer Responses}

\begin{enumerate}[label=(\alph*)]
    \item See code.
    \item If we go back to our natural language questions, e.g.\
          \emph{What is the capital of France?} and decode a sequence of several
          tokens, we now see the models were indeed sometimes trying to generate
          coherent sentences that we could not see by simply inspecting the top
          tokens produced in the first decoding step.
          However, we also get to see interesting things about these models in
          this case.
          For example, we see that if we keep decoding tokens, the models just
          ``keep talking'', often revealing the type of data used to train them
          (they may reveal a multiple choice type of setting).
          Also, Llama3.2-1B is able to answer the question about the capital of
          France, but it may reproduce the question first, which is an
          unintuitive things to do (though this does not really happen with
          the instruction-tuned model).
          GPT-3 is very sensitive to whether we include a whitespace at the end
          of the prompt or not.
          If not, GPT-3 will often produce an endless sequence of new lines, but
          if we include the whitespace, it sometimes produces more coherent
          sequences. Try that with the simple prompts \emph{``Hello?''} and
          \emph{``Hello? ''}.
    \item Indeed different sampling methods make a difference, as expected.
          Try it for both prompts that expect a factual answer as well as
          prompts that have more open-ended tasks.
          In particular, it is interesting to note that it's difficult for the
          instruct variant of Llama3.2 to produce an incoherent sentence, but
          when increasing the temperature in random sampling, we do get the
          model to rant about relevant things instead of answerinng the
          factual question we ask, e.g. \emph{What is the capital of France?}.
    \item Here we can get an idea of how to use pre-trained LLMs to construct
          chatbots. We set up a system prompt, i.e.\ instructions given to the 
          model that the user does not see, and we keep feeding the model a 
          new prompt that contains the system prompt, the chat history and the 
          new dialogue entry from the user. 
          Recall that these models do not possess long-term memory, which is why
          there has been significant effort in recent years to increase the size
          of the context window in LLMs.
          
          Try different system prompts. Do all models follow the instructions?
          How about differences with instruction-tuned models?
          For example, instruction-tuned models seem to be capable of knowing 
          when to stop generating the sequence (perhaps by producing the 
          end-of-sequence EOS token), but if asked the non instructon-tuned 
          Llama to generate many tokens in this setting, it will just complete 
          a dialogue between the user and the assistant.
\end{enumerate}

\end{document}
