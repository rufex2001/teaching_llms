\documentclass[11pt,a4paper]{article}

% packages
\usepackage[utf8]{inputenc}
\usepackage{amsmath}
\usepackage[T1]{fontenc}
\usepackage{setspace}
\usepackage{enumitem}
\usepackage{amsmath}
\usepackage{booktabs}
\usepackage{fullpage} 
\usepackage{tabularx}
\usepackage{amssymb, amstext, amsmath}
\usepackage{fancyhdr}
\usepackage{graphicx}
\usepackage{algorithmic}
\usepackage[ruled,vlined]{algorithm2e}
\usepackage{url}
\usepackage[bookmarks,unicode=true,pdftex,a4paper]{hyperref}
\usepackage[round]{natbib}
\usepackage[usenames,dvipsnames]{color, xcolor}
\headsep1cm

% macros
% misc
\newcommand\todo[1]{\textcolor{red}{TODO: #1}}
\newcommand\hide[1]{\textcolor{white}{#1}}

% formatting
\newcommand\bld[1]{\textbf{#1}}
\newcommand\ul[1]{\underline{#1}}
\newcommand\n[1]{\numprint{#1}}
\newcommand{\ts}{\textsuperscript}
\newcommand\red[1]{\textcolor{red}{#1}}
\newcommand\blue[1]{\textcolor{blue}{#1}}
\newcommand\link[2]{\href{#1}{\textcolor{blue}{\underline{#2}}}}

% sets
\newcommand\set[1]{\mathcal{#1}}
\newcommand\bb[1]{\mathbb{#1}}
\renewcommand\:{\colon} % for use with \sset, etc.
\newcommand{\sset}[1]{\left\{\,#1\,\right\}} % { ? }, automatic brackets
\newcommand{\ssets}[1]{\left\{#1\right\}} % {?}, automatic brackets
\newcommand{\ssetn}[1]{\{\,#1\,\}} % { ? }, normal brackets

% table formatting
% To better align bold entries in S columns (still broken)
% \usepackage{siunitx}
% \robustify\bfseries
% \newrobustcmd{\bfcell}{\bfseries}

% vector variables (taken from macros by Rainer Gemulla)
\newcommand\vect[1]{{\boldsymbol{#1}}}
\newcommand\va{\vect{a}}
\newcommand\vb{\vect{b}}
\newcommand\vc{\vect{c}}
\newcommand\vd{\vect{d}}
\newcommand\ve{\vect{e}}
\newcommand\vf{\vect{f}}
\newcommand\vg{\vect{g}}
\newcommand\vh{\vect{h}}
\newcommand\vi{\vect{i}}
\newcommand\vj{\vect{j}}
\newcommand\vk{\vect{k}}
\newcommand\vl{\vect{l}}
\newcommand\vm{\vect{m}}
\newcommand\vn{\vect{n}}
\newcommand\vo{\vect{o}}
\newcommand\vp{\vect{p}}
\newcommand\vq{\vect{q}}
\newcommand\vr{\vect{r}}
\newcommand\vs{\vect{s}}
\newcommand\vt{\vect{t}}
\newcommand\vu{\vect{u}}
\newcommand\vv{\vect{v}}
\newcommand\vw{\vect{w}}
\newcommand\vx{\vect{x}}
\newcommand\vy{\vect{y}}
\newcommand\vz{\vect{z}}
\newcommand\vzero{\vect{0}}
\newcommand\vone{\vect{1}}

\newcommand\valpha{\vect{\alpha}}
\newcommand\vbeta{\vect{\beta}}
\newcommand\veps{\vect{\epsilon}}
\newcommand\vdelta{\vect{\delta}}
\newcommand\vtheta{\vect{\theta}}
\newcommand\vsigma{\vect{\sigma}}
\newcommand\vpi{\vect{\pi}}
\newcommand\vlambda{\vect{\lambda}}

% matrix variables (taken from macros by Rainer Gemulla)
\newcommand\mA{\vect{A}}
\newcommand\mB{\vect{B}}
\newcommand\mC{\vect{C}}
\newcommand\mD{\vect{D}}
\newcommand\mE{\vect{E}}
\newcommand\mF{\vect{F}}
\newcommand\mG{\vect{G}}
\newcommand\mH{\vect{H}}
\newcommand\mI{\vect{I}}
\newcommand\mJ{\vect{J}}
\newcommand\mK{\vect{K}}
\newcommand\mL{\vect{L}}
\newcommand\mM{\vect{M}}
\newcommand\mN{\vect{N}}
\newcommand\mO{\vect{O}}
\newcommand\mP{\vect{P}}
\newcommand\mQ{\vect{Q}}
\newcommand\mR{\vect{R}}
\newcommand\mS{\vect{S}}
\newcommand\mT{\vect{T}}
\newcommand\mU{\vect{U}}
\newcommand\mV{\vect{V}}
\newcommand\mW{\vect{W}}
\newcommand\mX{\vect{X}}
\newcommand\mY{\vect{Y}}
\newcommand\mZ{\vect{Z}}
\newcommand\mzero{\vect{0}}

\newcommand{\mPi}{{\ensuremath{\vect{\Pi}}}}
\newcommand{\mSigma}{{\ensuremath{\vect{\Sigma}}}}
\newcommand{\mLambda}{{\ensuremath{\vect{\Lambda}}}}

% argmin, argmax
\DeclareMathOperator*{\argmin}{argmin} % amsmath package required
\DeclareMathOperator*{\argmax}{argmax} % amsmath package required

% matrix operations
\newcommand\xdiag{\operatorname{diag}}      
\newcommand\diag[1]{\xdiag\left(#1\right)}    % diagonal matrix


% new commands
\newcommand\op[1]{\operatorname{#1}}

% header and footer
\lhead{Advanced Methods in Text Analytics, FSS 2025}
\chead{}
\rhead{\thepage\ }
\cfoot{}
\pagestyle{fancy}

\title{Advanced Methods in Text Analytics \\ 
Exercise 8: Large Language Models - Part 2\\
\textbf{Solutions}}
\author{Daniel Ruffinelli}
\date{FSS 2025}

\begin{document}
\maketitle

\section{Making Predictions with LLMs}

\begin{enumerate}[label=(\alph*)]
    \item TODO: num layers, hidden sizes and vocab sizes of each of the models.
    \item TODO: what is BatchEncoding object, what is input\_ids (token ids),
          what is attention mask, how many tokens are used to encode your
          prompt? len(input\_ids)
    \item TODO: what is CausalLMOutputWithPast, what is logits, what is
          past\_key\_values. Logits are of size
          (batch size, input sequence length, vocab size)
    \item TODO: See code, explain params in batch decode, tokens make sense,
          difference between models?
          This function is useful for sampling, e.g. greedy, top k.
\end{enumerate}

\section{Prompting}

\begin{enumerate}[label=(\alph*)]
    \item TODO: difference gpt vs llama, output of gpt is always new line,
          likely wants to create a sentence, so greedy on first token does not
          work, we would have to decode more next tokens as well, answer Paris
          contained in top 10 tokens for gpt, but how to use it? Top k sampling?
          It's a factual question, we want a deterministic answer.
          If we try a more ICL-like prompt, then the top token is the correct
          answer, but this too is brittle because of tokenization issues, e.g.
          Rome is often tokenized with a space, so we need to add the space to
          the prompt, and this is not consistent across models or examples.
          Try to out!
    \item TODO: ICL prompt has instruction, demonstration and question.
    \item TODO: are all tasks clear with zero shot? Which ones and which ones
          not? Do both models understand the tasks? What tasks become clear with
          demonstrations?
    \item TODO: can any task be solved with an added natural instruction?
    \item TODO: go from asking natural question to ICL in zero shot, n shot and
          then with instructions. I think the difference is: instruct model can
          do natural question (e.g. ask a capital city), and it can also do zero
          shot when a natural language instruction is included (e.g. do
          translate to french). They should be able to also generate
          responses in a given mood, but we do that in the next task.
\end{enumerate}

\section{Generating Longer Responses}

\begin{enumerate}[label=(\alph*)]
    \item See code.
    \item TODO: before we saw greedy not good for llama and gpt sometimes, but
          now with more tokens decoded? Were they trying to talk?
    \item TODO: try different sampling methods, see code in RAW notebook.
    \item TODO: Try different lengths: short means incomplete sentences, long
          means model continues conversation by itself. Also try different 
          system prompts. Do all models follow the instructions?  
\end{enumerate}

\end{document}
