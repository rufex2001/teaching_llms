\documentclass[11pt,a4paper]{article}

% packages
\usepackage[utf8]{inputenc}
\usepackage{amsmath}
\usepackage[T1]{fontenc}
\usepackage{setspace}
\usepackage{enumitem}
\usepackage{booktabs}
\usepackage{fullpage} 
\usepackage{tabularx}
\usepackage{amssymb,amstext,amsmath}
\usepackage{fancyhdr}
\usepackage{graphicx}
\usepackage{algorithmic}
\usepackage[ruled,vlined]{algorithm2e}
\usepackage{url}
\usepackage[colorlinks=true,bookmarks,unicode=true,pdftex,a4paper]{hyperref}
\usepackage[round]{natbib}
\usepackage[usenames,dvipsnames]{color, xcolor}
\headsep1cm

% macros
% misc
\newcommand\todo[1]{\textcolor{red}{TODO: #1}}
\newcommand\hide[1]{\textcolor{white}{#1}}

% formatting
\newcommand\bld[1]{\textbf{#1}}
\newcommand\ul[1]{\underline{#1}}
\newcommand\n[1]{\numprint{#1}}
\newcommand{\ts}{\textsuperscript}
\newcommand\red[1]{\textcolor{red}{#1}}
\newcommand\blue[1]{\textcolor{blue}{#1}}
\newcommand\link[2]{\href{#1}{\textcolor{blue}{\underline{#2}}}}

% sets
\newcommand\set[1]{\mathcal{#1}}
\newcommand\bb[1]{\mathbb{#1}}
\renewcommand\:{\colon} % for use with \sset, etc.
\newcommand{\sset}[1]{\left\{\,#1\,\right\}} % { ? }, automatic brackets
\newcommand{\ssets}[1]{\left\{#1\right\}} % {?}, automatic brackets
\newcommand{\ssetn}[1]{\{\,#1\,\}} % { ? }, normal brackets

% table formatting
% To better align bold entries in S columns (still broken)
% \usepackage{siunitx}
% \robustify\bfseries
% \newrobustcmd{\bfcell}{\bfseries}

% vector variables (taken from macros by Rainer Gemulla)
\newcommand\vect[1]{{\boldsymbol{#1}}}
\newcommand\va{\vect{a}}
\newcommand\vb{\vect{b}}
\newcommand\vc{\vect{c}}
\newcommand\vd{\vect{d}}
\newcommand\ve{\vect{e}}
\newcommand\vf{\vect{f}}
\newcommand\vg{\vect{g}}
\newcommand\vh{\vect{h}}
\newcommand\vi{\vect{i}}
\newcommand\vj{\vect{j}}
\newcommand\vk{\vect{k}}
\newcommand\vl{\vect{l}}
\newcommand\vm{\vect{m}}
\newcommand\vn{\vect{n}}
\newcommand\vo{\vect{o}}
\newcommand\vp{\vect{p}}
\newcommand\vq{\vect{q}}
\newcommand\vr{\vect{r}}
\newcommand\vs{\vect{s}}
\newcommand\vt{\vect{t}}
\newcommand\vu{\vect{u}}
\newcommand\vv{\vect{v}}
\newcommand\vw{\vect{w}}
\newcommand\vx{\vect{x}}
\newcommand\vy{\vect{y}}
\newcommand\vz{\vect{z}}
\newcommand\vzero{\vect{0}}
\newcommand\vone{\vect{1}}

\newcommand\valpha{\vect{\alpha}}
\newcommand\vbeta{\vect{\beta}}
\newcommand\veps{\vect{\epsilon}}
\newcommand\vdelta{\vect{\delta}}
\newcommand\vtheta{\vect{\theta}}
\newcommand\vsigma{\vect{\sigma}}
\newcommand\vpi{\vect{\pi}}
\newcommand\vlambda{\vect{\lambda}}

% matrix variables (taken from macros by Rainer Gemulla)
\newcommand\mA{\vect{A}}
\newcommand\mB{\vect{B}}
\newcommand\mC{\vect{C}}
\newcommand\mD{\vect{D}}
\newcommand\mE{\vect{E}}
\newcommand\mF{\vect{F}}
\newcommand\mG{\vect{G}}
\newcommand\mH{\vect{H}}
\newcommand\mI{\vect{I}}
\newcommand\mJ{\vect{J}}
\newcommand\mK{\vect{K}}
\newcommand\mL{\vect{L}}
\newcommand\mM{\vect{M}}
\newcommand\mN{\vect{N}}
\newcommand\mO{\vect{O}}
\newcommand\mP{\vect{P}}
\newcommand\mQ{\vect{Q}}
\newcommand\mR{\vect{R}}
\newcommand\mS{\vect{S}}
\newcommand\mT{\vect{T}}
\newcommand\mU{\vect{U}}
\newcommand\mV{\vect{V}}
\newcommand\mW{\vect{W}}
\newcommand\mX{\vect{X}}
\newcommand\mY{\vect{Y}}
\newcommand\mZ{\vect{Z}}
\newcommand\mzero{\vect{0}}

\newcommand{\mPi}{{\ensuremath{\vect{\Pi}}}}
\newcommand{\mSigma}{{\ensuremath{\vect{\Sigma}}}}
\newcommand{\mLambda}{{\ensuremath{\vect{\Lambda}}}}

% argmin, argmax
\DeclareMathOperator*{\argmin}{argmin} % amsmath package required
\DeclareMathOperator*{\argmax}{argmax} % amsmath package required

% matrix operations
\newcommand\xdiag{\operatorname{diag}}      
\newcommand\diag[1]{\xdiag\left(#1\right)}    % diagonal matrix


% new commands
\newcommand\op[1]{\operatorname{#1}}

% header and footer
\lhead{Advanced Methods in Text Analytics, FSS 2025}
\chead{}
\rhead{\thepage\ }
\cfoot{}
\pagestyle{fancy}

\title{Advanced Methods in Text Analytics \\ 
Exercise 8: Large Language Models - Part 2}
\author{Daniel Ruffinelli}
\date{FSS 2025}

\begin{document}
\maketitle

In this exercise, we use code to explore some basic uses of pre-trained LLMs.
The code runs well on CPU in a laptop with 16GB of RAM, where the most expensive
tasks take about 10 seconds. Still, feel free to modify the code to better suit
your resources.

% \section{KV-Caching and More Efficient Self-Attention}

% TODO: consider covering KV cache and multiquery attention and grouped query 
% attention.

\section{Making Predictions with LLMs}

In this task, we look at how to make predictions with pre-trained LLMs.

\begin{enumerate}[label=(\alph*)]
    \item We first load two pre-trained LLMs:
          \href{https://github.com/EleutherAI/gpt-neo}{GPT Neo} of 1.3B
          parameters and released in 2021 by EleutherAI, and
          \href{https://huggingface.co/meta-llama/Llama-3.2-1B}{Llama-3.2-1B},
          a model released in late 2024.
          We also load their corresponding tokenizers, so we can encode and
          decode text to interact with the models.
          Run the given code to load the models.
          Then give a basic description of their architecture as described by
          the output in the code, e.g.\ number of transformer layers, size of
          hidden states, size of the vocabulary.
    \item Next, we write a simple prompt and tokenize it with the
          tokenizer of any of the models we loaded. Run the give code and
          inspect the object given as output. What type of object is it? What
          components does it have? What do they represent? Go over some of
          \href{https://huggingface.co/docs/transformers/index}{HuggingFace's documentation}
          and add some code to explore the output in order to find the answer.
          How many tokens are required to encode your prompt?
    \item We now use the model to predict the next token given your prompt
          from the previous subtask. Run the given code and inspect the object
          returned by the model. What type of object is it? What components does
          it have? What do they represent?
          As before, use HuggingFace's documentation and run some code to find
          the answer.
          Where is the distribution over vocabulary that we should use to
          predict the next token?
    \item Finally, we obtain the top $k$ tokens predicted by a model. For that,
          fill in the function \texttt{get\_top\_k\_tokens} in the given code.
          To decode tokens, use the function \texttt{batch\_decode}. You can
          learn about it
          \href{https://huggingface.co/docs/transformers/en/main_classes/tokenizer#transformers.PreTrainedTokenizerFast.batch_decode}{here}.
          Do the top $k$ tokens make sense given the prompt you tested? Are
          there any differences in the output of the two different models?
          What could this function \texttt{get\_top\_k\_tokens} be used for?
\end{enumerate}

\section{Prompting}

In this task, we test the impact that different prompts have on the predictions
made by LLMs.

\begin{enumerate}[label=(\alph*)]
    \item Let's inspect the top $k$ tokens predicted by the models from the
          previous task.
          Run the code and test the output of the model when given the different
          prompts in the code. Try different prompts of your own as well.
          What do the top $k$ tokens look like? Do they contain the answer to
          the questions? Can we use them to make predictions? If so, how? If
          not, can you find a prompt that results in top tokens we can actually
          use for predictions?
    \item Let's now try a more specific type of prompt: in-context learning
          (ICL) prompts. Look at the tasks defined in the code, run it and
          see whether the format of the prompts matches what we know about ICL.
          What are the different components of an ICL prompt?
    \item We now try our ICL prompts with our loaded pre-trained models.
          Let's start with zero-shot prompts and no instruction. Do the models
          understand some tasks? Which ones? Then add a few demonstrations and
          compare with the zero-shot setting. Do some tasks benefit from having
          demonstrations more than others? Do both models perform equally well?
          Does the prompt template make a difference?
    \item Try adding natural language instructions to the task. Does it help?
          If so, on which tasks? And on which models?
    \item We now add a new model, an instruction-tuned variant of the same Llama
          we have been testing. Try all prompts so far starting with the natural
          questions and all the way to the ICL prompts with natural language
          instruction. Do you see any difference compared to the other two
          models?
\end{enumerate}

\section{Generating Longer Responses}

In this task, we explore how to generate longer responses with LLMs and what
these responses look like under different settings.

\begin{enumerate}[label=(\alph*)]
    \item Complete the function \texttt{generate} given in the code using
          everything we've learned so far. Make sure to use greedy sampling.
    \item Try the \texttt{generate} function with different prompts. Do the
          models produce reasonable responses? Is there a difference between
          models? Does the output of some models make more sense now that we are
          decoding more tokens?
    \item HuggingFace models already include a
          \href{https://huggingface.co/docs/transformers/v4.51.3/en/main_classes/text_generation#transformers.GenerationMixin.generate}{\texttt{generate}}
          function similar to ours, which we will use from now for simplicity.
          Read about that function and try using it to generate responses using
          different sampling methods. What differences do you observe between
          them? Note that the output of this functio already includes the given 
          prompt, i.e.\ it isn't generated by the model.
    \item We now contruct a prompt that simulates a dialogue system as a way to
          see if this type of prompts make a difference with model outputs.
          Inspect the function \texttt{construct\_chat\_prompt} and run the code
          to test it. Try different system prompts and generate responses of 
          different lengths. Use HuggingFace's \texttt{generate} function, as it 
          is more efficient. Are the responses as you expected? If not, what 
          issues do you see? And how could we solve them? 
\end{enumerate}

\end{document}
