\documentclass[11pt,a4paper]{article}

% packages
\usepackage[utf8]{inputenc}
\usepackage{amsmath}
\usepackage[T1]{fontenc}
\usepackage{setspace}
\usepackage{enumitem}
\usepackage{booktabs}
\usepackage{fullpage} 
\usepackage{tabularx}
\usepackage{amssymb,amstext,amsmath}
\usepackage{fancyhdr}
\usepackage{graphicx}
\usepackage{algorithmic}
\usepackage[ruled,vlined]{algorithm2e}
\usepackage{url}
\usepackage[colorlinks=true,bookmarks,unicode=true,pdftex,a4paper]{hyperref}
\usepackage[round]{natbib}
\usepackage[usenames,dvipsnames]{color, xcolor}
\headsep1cm

% macros
% misc
\newcommand\todo[1]{\textcolor{red}{TODO: #1}}
\newcommand\hide[1]{\textcolor{white}{#1}}

% formatting
\newcommand\bld[1]{\textbf{#1}}
\newcommand\ul[1]{\underline{#1}}
\newcommand\n[1]{\numprint{#1}}
\newcommand{\ts}{\textsuperscript}
\newcommand\red[1]{\textcolor{red}{#1}}
\newcommand\blue[1]{\textcolor{blue}{#1}}
\newcommand\link[2]{\href{#1}{\textcolor{blue}{\underline{#2}}}}

% sets
\newcommand\set[1]{\mathcal{#1}}
\newcommand\bb[1]{\mathbb{#1}}
\renewcommand\:{\colon} % for use with \sset, etc.
\newcommand{\sset}[1]{\left\{\,#1\,\right\}} % { ? }, automatic brackets
\newcommand{\ssets}[1]{\left\{#1\right\}} % {?}, automatic brackets
\newcommand{\ssetn}[1]{\{\,#1\,\}} % { ? }, normal brackets

% table formatting
% To better align bold entries in S columns (still broken)
% \usepackage{siunitx}
% \robustify\bfseries
% \newrobustcmd{\bfcell}{\bfseries}

% vector variables (taken from macros by Rainer Gemulla)
\newcommand\vect[1]{{\boldsymbol{#1}}}
\newcommand\va{\vect{a}}
\newcommand\vb{\vect{b}}
\newcommand\vc{\vect{c}}
\newcommand\vd{\vect{d}}
\newcommand\ve{\vect{e}}
\newcommand\vf{\vect{f}}
\newcommand\vg{\vect{g}}
\newcommand\vh{\vect{h}}
\newcommand\vi{\vect{i}}
\newcommand\vj{\vect{j}}
\newcommand\vk{\vect{k}}
\newcommand\vl{\vect{l}}
\newcommand\vm{\vect{m}}
\newcommand\vn{\vect{n}}
\newcommand\vo{\vect{o}}
\newcommand\vp{\vect{p}}
\newcommand\vq{\vect{q}}
\newcommand\vr{\vect{r}}
\newcommand\vs{\vect{s}}
\newcommand\vt{\vect{t}}
\newcommand\vu{\vect{u}}
\newcommand\vv{\vect{v}}
\newcommand\vw{\vect{w}}
\newcommand\vx{\vect{x}}
\newcommand\vy{\vect{y}}
\newcommand\vz{\vect{z}}
\newcommand\vzero{\vect{0}}
\newcommand\vone{\vect{1}}

\newcommand\valpha{\vect{\alpha}}
\newcommand\vbeta{\vect{\beta}}
\newcommand\veps{\vect{\epsilon}}
\newcommand\vdelta{\vect{\delta}}
\newcommand\vtheta{\vect{\theta}}
\newcommand\vsigma{\vect{\sigma}}
\newcommand\vpi{\vect{\pi}}
\newcommand\vlambda{\vect{\lambda}}

% matrix variables (taken from macros by Rainer Gemulla)
\newcommand\mA{\vect{A}}
\newcommand\mB{\vect{B}}
\newcommand\mC{\vect{C}}
\newcommand\mD{\vect{D}}
\newcommand\mE{\vect{E}}
\newcommand\mF{\vect{F}}
\newcommand\mG{\vect{G}}
\newcommand\mH{\vect{H}}
\newcommand\mI{\vect{I}}
\newcommand\mJ{\vect{J}}
\newcommand\mK{\vect{K}}
\newcommand\mL{\vect{L}}
\newcommand\mM{\vect{M}}
\newcommand\mN{\vect{N}}
\newcommand\mO{\vect{O}}
\newcommand\mP{\vect{P}}
\newcommand\mQ{\vect{Q}}
\newcommand\mR{\vect{R}}
\newcommand\mS{\vect{S}}
\newcommand\mT{\vect{T}}
\newcommand\mU{\vect{U}}
\newcommand\mV{\vect{V}}
\newcommand\mW{\vect{W}}
\newcommand\mX{\vect{X}}
\newcommand\mY{\vect{Y}}
\newcommand\mZ{\vect{Z}}
\newcommand\mzero{\vect{0}}

\newcommand{\mPi}{{\ensuremath{\vect{\Pi}}}}
\newcommand{\mSigma}{{\ensuremath{\vect{\Sigma}}}}
\newcommand{\mLambda}{{\ensuremath{\vect{\Lambda}}}}

% argmin, argmax
\DeclareMathOperator*{\argmin}{argmin} % amsmath package required
\DeclareMathOperator*{\argmax}{argmax} % amsmath package required

% matrix operations
\newcommand\xdiag{\operatorname{diag}}      
\newcommand\diag[1]{\xdiag\left(#1\right)}    % diagonal matrix


% header and footer
\lhead{Advanced Methods in Text Analytics, FSS2025}
\chead{}
\rhead{\thepage\ }
\cfoot{}
\pagestyle{fancy}

\title{Advanced Methods in Text Analytics \\ 
Exercise 3: Language Models - Part 1}
\author{Daniel Ruffinelli}
\date{FSS 2025}

\begin{document}
\maketitle

\section{Language Modeling}

Language models are able to predict the next word in a given sequence of $n$
words. Similarly, language models can give the probability of a given
sequence of $n$ words.

\begin{enumerate}[label=(\alph*)]
    \item Give a formal expression for the probability of word $n+1$ given the
          previous $n$ words in a sequence.
    \item Give a formal expression for the probability of a sequence of $n$
          words and specify how this probability is computed using the marginal
          probabilities of each word in the sequence.
    \item What is the main assumption that n-gram language models make?
          What is the name of this assumption?
          What is a bigram model? And a trigram model?
          Give the expressions for computing the probability of a sequence of
          words (i.e. your answer to b)) using a bigram model, and a trigram
          model.
    \item How is the probability of predicting the next word given previous
          words \emph{estimated} in n-gram models?
    \item What are some of the problems that n-gram language models have?
    \item Neural language models can address some of the issues with n-gram
          models.
          Describe the neural language model designed by
          \href{https://www.jmlr.org/papers/volume3/bengio03a/bengio03a.pdf}{\underline{Bengio et al.}},
          which was discussed in the lecture.
          What kind of neural network was it? How many hidden layers did it
          have? How did they represent the input sequence? How did it compute
          the relevant probabilities for language modeling?
          How was it parameterized? And how many parameters does it have with a
          vocabulary $V$ and word embedding of size $d$?
          For simplicity, you may ignore the optional linear layer and all
          biases.
    \item What is \emph{self-supervision}? Describe the self-supervised training
          approach used by Bengio et al. to train their neural language model.
    \item What problems in n-gram models were no longer an issue with neural
          language models? And what problems still persisted with such models?
          Explain why the existing limitations are indeed limitations.
\end{enumerate}

\newpage
\section{Language Model Evaluation}

In this task, we discuss how language models (LMs) can be evaluated.

\begin{enumerate}[label=(\alph*)]
    \item LMs are often used as part of various \emph{downstream tasks}.
          Give two examples of downstream tasks that may use LMs in their
          pipeline, and describe how may use LMs.
    \item LMs can be evaluated extrinsically and intrinsically.
          Describe the difference between intrinsic and extrinsic evaluation.
          Discuss some advantages and disadvantages of each.
    \item Say you are tasked with designing the simplest
          intrinsic evaluation for language models using the basic machine
          learning principle of \emph{held-out data} and a given text
          corpus.
          What sort of evaluation would you propose?
    \item The most common form of intrinsic evaluation of language models is
          computing its \emph{perplexity} on held-out data.
          Perplexity is defined as follows:
          \begin{align*}
              ppl(w_1,w_2,\ldots,w_n) = p(w_1,w2,\ldots,w_n)^{-\frac{1}{n}},
          \end{align*}
          where $p$ is the \emph{likelihood} of held-out sequence
          $x_1, x_2, \ldots, x_n$.
          Describe the intuition of perplexity from the perspective of the
          likelihood of the data. How does one change when the other changes?
    \item Assume a uniform unigram model over some vocabulary $V$. What
          probability does such a model assign to each word in the vocabulary?
          Compute the likelihood of this model and use it to compute its
          perplexity over a held-out sequence of $n$ words.
\end{enumerate}

\section{Recurrent Neural Networks}

In this section, we review some basic concepts about RNNs, including how they
can be used to implement language models.

\begin{enumerate}[label=(\alph*)]
    \item Describe the components of an RNN and discuss how it differs from
          fully-connected neural networks. What are the parameters of an RNN?
          Are RNNs deep? In what way(s)?
    \item How can a RNN be used for a sequence classification task? And what
          about for a task where each token in the sequence requires a predicted
          label, e.g.\ part-of-speech (POS) tagging? Describe the input, output
          and general architecture of the model.
    \item How can RNNs be used for language modeling? Describe the model's
          architecture.
          What is the fundamental problem that RNNs solve over fully-connected
          neural networks for language modeling?
    \item What are bidirectional RNNs? What are their advantages and
          disadvantages compared to standard (unidimensional) RNNs?
    \item Describe the main problem with RNNs and how LSTMs were designed to
          address this issue. Give a high-level intuition for each of the gates
          in an LSTM unit.
    \item Describe an encoder-decoder architecture and explain how it can be
          used for machine translation, i.e.\ the task of predicting a sequence
          of words in a target language given a sequence of words in a source
          language.
          When implementing such an architecture with RNNs, what is the input
          and output of the hidden state in the encoder at each time step? And
          in the decoder?
\end{enumerate}

\section{Sampling for Text Generation}

An important aspect of using language models to (autoregressively) generate text
is the step of \emph{sampling} the next word from the distribution over words in
our vocabulary that is produced by a language model.

\begin{enumerate}[label=(\alph*)]
    \item What is autoregressive text generation? What are its advantages
          compared non-autoregressive text generation? And its disadvantages?
    \item In a machine translation setting, let $X = \{x_1, x_2, \ldots, x_n\}$
          be the input sequence in some source language,
          $Y = \{y_1, y_2, \ldots, y_m\}$ the output sequence in some target
          language, and $p(y_t|X,\vtheta)$ the distribution for the $t$-th word
          in a non-autoregressive setting using a model parameterized by
          $\vtheta$.
          Give the corresponding expression for $p$ in an autoregressive model.
    \item Let $\operatorname{sample}(p)$ be a function to sample a word
          from the autoregressive expression for $p$ you provided in the
          previous question, and $<s>, </s>$ be the start-of-sequence and
          end-of-sequence tokens, respectively.
          Write an algorithm in pseudocode to autoregressively generate a
          sequence of words of arbitrary length.
          % TODO this question about the algorithm is not so correct, because
          % the sample function should take a distribution over words as input,
          % not the probability value of some specific word!
          % So sample should take as input the sequence only
    \item The most straightforward way to implement the \emph{sample}
          function used in the previous question is by choosing the word with
          the highest probability. This is known as \emph{greedy} sampling.
          Can you think of advantages and disadvantages to such an approach?
          Use the following two incomplete sentences to reason about this.
          \begin{enumerate}[label=(\roman*)]
              \item The capital of Germany is $\ldots$
              \item The title of my talk is $\ldots$
          \end{enumerate}
    \item Another way to implement the \emph{sample} function is by
          \emph{randomly} sampling from $p$.
          This means we sample a word $w$ with probability $p(w)$, so that words
          with higher probability get sampled more often, while words with lower
          probability get sampled less often.
          Can you think of advantages and disadvantages to such an approach?
          \textbf{Hint:} the size of the vocabulary is often very large, and for
          a given input sequence, the distribution over words is very skewed.
    \item Can you think of an approach that can be used to sample from the
          distribution over words in a more balanced way compared to both
          \emph{greedy} and \emph{random} sampling?
          What are its advantages and disadvantages compared to the approaches
          described in the previous questions?
    \item \emph{Temperature} sampling is another approach, where instead of
          truncating a model distribution, we reshape it.
          It is defined in the context of softmax distributions as follows:
          \begin{align*}
              \vy = \operatorname{softmax}\left(\frac{\vu}{\tau}\right),
          \end{align*}
          where $\vu$ is the unnormalized distribution over words, i.e.\
          a vector of logits, $\tau$ is the temperature parameter, and we do an
          element-wise division with it.
          Describe the impact that different values of $\tau$ have over the
          resulting model distribution that we will then use to sample the next
          word from.
          Specifically, how does the distribution change as $\tau$ is less than
          1 and approaches zero? And as it approaches infinity?
\end{enumerate}

\end{document}
