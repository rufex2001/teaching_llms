\documentclass[11pt,a4paper]{article}

% packages
\usepackage[utf8]{inputenc}
\usepackage{amsmath}
\usepackage[T1]{fontenc}
\usepackage{setspace}
\usepackage{enumitem}
\usepackage{booktabs}
\usepackage{fullpage} 
\usepackage{tabularx}
\usepackage{amssymb,amstext,amsmath}
\usepackage{fancyhdr}
\usepackage{graphicx}
\usepackage{algorithmic}
\usepackage[ruled,vlined]{algorithm2e}
\usepackage{url}
\usepackage[colorlinks=true,bookmarks,unicode=true,pdftex,a4paper]{hyperref}
\usepackage[round]{natbib}
\usepackage[usenames,dvipsnames]{color, xcolor}
\headsep1cm

% macros
% misc
\newcommand\todo[1]{\textcolor{red}{TODO: #1}}
\newcommand\hide[1]{\textcolor{white}{#1}}

% formatting
\newcommand\bld[1]{\textbf{#1}}
\newcommand\ul[1]{\underline{#1}}
\newcommand\n[1]{\numprint{#1}}
\newcommand{\ts}{\textsuperscript}
\newcommand\red[1]{\textcolor{red}{#1}}
\newcommand\blue[1]{\textcolor{blue}{#1}}
\newcommand\link[2]{\href{#1}{\textcolor{blue}{\underline{#2}}}}

% sets
\newcommand\set[1]{\mathcal{#1}}
\newcommand\bb[1]{\mathbb{#1}}
\renewcommand\:{\colon} % for use with \sset, etc.
\newcommand{\sset}[1]{\left\{\,#1\,\right\}} % { ? }, automatic brackets
\newcommand{\ssets}[1]{\left\{#1\right\}} % {?}, automatic brackets
\newcommand{\ssetn}[1]{\{\,#1\,\}} % { ? }, normal brackets

% table formatting
% To better align bold entries in S columns (still broken)
% \usepackage{siunitx}
% \robustify\bfseries
% \newrobustcmd{\bfcell}{\bfseries}

% vector variables (taken from macros by Rainer Gemulla)
\newcommand\vect[1]{{\boldsymbol{#1}}}
\newcommand\va{\vect{a}}
\newcommand\vb{\vect{b}}
\newcommand\vc{\vect{c}}
\newcommand\vd{\vect{d}}
\newcommand\ve{\vect{e}}
\newcommand\vf{\vect{f}}
\newcommand\vg{\vect{g}}
\newcommand\vh{\vect{h}}
\newcommand\vi{\vect{i}}
\newcommand\vj{\vect{j}}
\newcommand\vk{\vect{k}}
\newcommand\vl{\vect{l}}
\newcommand\vm{\vect{m}}
\newcommand\vn{\vect{n}}
\newcommand\vo{\vect{o}}
\newcommand\vp{\vect{p}}
\newcommand\vq{\vect{q}}
\newcommand\vr{\vect{r}}
\newcommand\vs{\vect{s}}
\newcommand\vt{\vect{t}}
\newcommand\vu{\vect{u}}
\newcommand\vv{\vect{v}}
\newcommand\vw{\vect{w}}
\newcommand\vx{\vect{x}}
\newcommand\vy{\vect{y}}
\newcommand\vz{\vect{z}}
\newcommand\vzero{\vect{0}}
\newcommand\vone{\vect{1}}

\newcommand\valpha{\vect{\alpha}}
\newcommand\vbeta{\vect{\beta}}
\newcommand\veps{\vect{\epsilon}}
\newcommand\vdelta{\vect{\delta}}
\newcommand\vtheta{\vect{\theta}}
\newcommand\vsigma{\vect{\sigma}}
\newcommand\vpi{\vect{\pi}}
\newcommand\vlambda{\vect{\lambda}}

% matrix variables (taken from macros by Rainer Gemulla)
\newcommand\mA{\vect{A}}
\newcommand\mB{\vect{B}}
\newcommand\mC{\vect{C}}
\newcommand\mD{\vect{D}}
\newcommand\mE{\vect{E}}
\newcommand\mF{\vect{F}}
\newcommand\mG{\vect{G}}
\newcommand\mH{\vect{H}}
\newcommand\mI{\vect{I}}
\newcommand\mJ{\vect{J}}
\newcommand\mK{\vect{K}}
\newcommand\mL{\vect{L}}
\newcommand\mM{\vect{M}}
\newcommand\mN{\vect{N}}
\newcommand\mO{\vect{O}}
\newcommand\mP{\vect{P}}
\newcommand\mQ{\vect{Q}}
\newcommand\mR{\vect{R}}
\newcommand\mS{\vect{S}}
\newcommand\mT{\vect{T}}
\newcommand\mU{\vect{U}}
\newcommand\mV{\vect{V}}
\newcommand\mW{\vect{W}}
\newcommand\mX{\vect{X}}
\newcommand\mY{\vect{Y}}
\newcommand\mZ{\vect{Z}}
\newcommand\mzero{\vect{0}}

\newcommand{\mPi}{{\ensuremath{\vect{\Pi}}}}
\newcommand{\mSigma}{{\ensuremath{\vect{\Sigma}}}}
\newcommand{\mLambda}{{\ensuremath{\vect{\Lambda}}}}

% argmin, argmax
\DeclareMathOperator*{\argmin}{argmin} % amsmath package required
\DeclareMathOperator*{\argmax}{argmax} % amsmath package required

% matrix operations
\newcommand\xdiag{\operatorname{diag}}      
\newcommand\diag[1]{\xdiag\left(#1\right)}    % diagonal matrix


% header and footer
\lhead{Advanced Methods in Text Analytics, FSS 2025}
\chead{}
\rhead{\thepage\ }
\cfoot{}
\pagestyle{fancy}

\title{Advanced Methods in Text Analytics \\ 
Exercise 5: Transformers - Part 1}
\author{Daniel Ruffinelli}
\date{FSS 2025}

\begin{document}
\maketitle

\section{Transformer Basics}

In this task, we review some basic concepts about the transformer architecture,
and have a closer look at the original approach for positional embeddings.

\begin{enumerate}[label=(\alph*)]
    \item Let $x_1,x_2,\ldots, x_n$ be a sequence of input tokens and
          $\mH = \{\vh_1,\vh_2,\ldots, \vh_n\}$ where $\vh_i\in\bb{R}^D$ the
          corresponding sequence of representations given by some neural
          network's hidden layer (e.g.\ an RNN or transformer encoder).
          Give a formal expression for how to compute context vector $\vc_k$
          using dot-product attention over $\mH$ with given query
          $\vk\in\bb{R}^K$ (e.g.\ the context vector corresponding to input
          token $k$ in self-attention).
          Make sure you also provide a formal definition for every component in
          the expression you provide for $\vc_k$.
          What are the names of these components? What is the size of $\vc_k$?
          And how does $D$ relate to $K$?
    \item What is the cost of a self-attention layer w.r.t. the input size?
          Why? Answer this for both the settings were we attend to all tokens in
          the input sequence and when we attend only to tokens previously seen
          in the input sequence.
          Why are the computations in a self-attention layer parallelizable?
    \item In this question, we aim at developing an intuition for how
          non-learned approaches to encoding positions work, by looking at the
          approach used in the original
          \href{https://arxiv.org/pdf/1706.03762.pdf}{\underline{transformer architecture}}.
          Specifically, the authors used the following encoding to create the
          $d$-dimensional positional embedding (PE) for position $k$:
          \begin{align*}
              \operatorname{PE}_{(k,2i)}   & = \sin\left(\frac{1}{n^{2i/d}}k\right), \\
              \operatorname{PE}_{(k,2i+1)} & = \cos\left(\frac{1}{n^{2i/d}}k\right).
          \end{align*}
          Here, $0\leq i \leq \frac{d}{2}$ and $n$ is a hyperparameter originally
          set to $10000$.
          Note that the first equation is used to assign values to the
          \emph{even} elements of the PE, and the second equation to the
          \emph{uneven} elements.
          In other words, each element in positional embeddings is determined
          by either a sine or cosine function, each with different arguments.

          \begin{enumerate}[label=(\roman*)]
              \item Given a maximum input length $L$, we can use these positional
                    encodings to construct a PE matrix of size $L\times d$.
                    Compute that matrix for the following sequences using $d=4$
                    and $n=1000$:
                    \begin{quote}
                        \emph{My cat is fine} \\
                        \emph{My dog is well} \\
                        \emph{My pets are very well}
                    \end{quote}
                    How do the resulting matrices compare to one another?
                    What about the size of the vectors? Are they large in terms
                    of magnitude? Is that what we want?
                    Discuss this w.r.t.\ the information that we expect that
                    positional encodings provide to our transformer model.
              \item The multiplying factor to $k$ in the argument of the
                    $\sin$/$\cos$ functions above is known as its frequency.
                    Given a fixed frequency, i.e.\ using a fixed value of
                    $d, n, i$, what happens to the value given by the
                    $\sin$/$\cos$ functions as we increase the values of $k$?
                    What will the features of such vectors look like as a
                    result?
              \item The frequency discussed above is defined as the inverse to
                    the period of the function as follows:
                    \begin{align*}
                        period = \frac{2\pi}{freq}
                    \end{align*}
                    The period of a function refers to how long it takes to
                    complete a full cycle (after which they start to repeat
                    themselves).
                    In other words, its horizontal stretch.
                    Using the values for $d$ and $n$ given above, compute the
                    periods for different values of $i$.
                    How do these periods relate to each other?
                    And what happens to the value of a given element of $i$ for
                    different values of $k$?
          \end{enumerate}
\end{enumerate}

\section{Thinking about Perplexity}

In this task, we take a high-level look at the relation between perplexity,
cross-entropy and log loss.

\begin{enumerate}[label=(\alph*)]
    \item For some random variable $X$, entropy $H(X)$ is defined as follows:
          \begin{align}\label{eq:def_entropy}
              H(x) = -\sum_{x\in X} p(x)\log p(x)
          \end{align}
          Say $X$ is the number shown when you toss a fair die.
          What is the entropy of $X$?
          Use $\log$ base 2 throughout.
    \item Following a), perplexity is defined as $2^{H(x)}$. Compute the
          perplexity of the result you obtained in a).
          Can you interpret the result?
          How does this expression relate to how we compute perplexity in code?
    \item In NLP, we are interested in the entropy of sequences.
          Say $X$ is now a random variable over all possible sequences of length
          $n$ over some language $L$.
          Write the expression to compute this entropy.
          How would you compute the average entropy per word in such a sequence?
    \item In the context of language models, $p$ is the distribution over some
          natural language $L$, and we don't have access to this distribution.
          Instead, we learn an estimate $m$ of this distribution from data.
          In such a context, the concept of \emph{cross-entropy} $H(p,m)$ is
          suitable.
          It is defined as the sum of $p(x)$ for all $x\in X$ weighted by their
          corresponding $\log$ probabilities according to $m$.
          Write the expression for cross-entropy \emph{rate} for a sequence of 
          length $n$.
    \item It can be shown that the cross-entropy of a language can be
          approximated by the following expression:
          \begin{align}\label{eq:language_entropy}
              H(W) = -\frac{1}{N} \log p(w_1,w_2,\ldots,w_N)
          \end{align}
          where $N$ is a sufficiently large number and $w_i$ are words in that
          language.
          As seen in question (b), perplexity is formally defined as
          $ppl(W) = 2^{H(W)}$, where $W$ is $w_1, w_2, \ldots, w_N$, and we
          define entropy using log base 2.
          Use this approximation of cross-entropy given above to compute the
          expression for perplexity given in the lecture.
          What does this expression say about how we evaluate a language
          model by computing perplexity on a held-out corpus?
\end{enumerate}

\end{document}
