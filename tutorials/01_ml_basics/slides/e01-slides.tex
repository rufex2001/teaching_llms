\documentclass[t]{beamer}

% packages
\usepackage[english]{babel}
\usepackage[utf8x]{inputenc}
\usepackage{mathtools}
\usepackage{amsfonts}
\usepackage{amsthm}
\usepackage{numprint}
\usepackage{amsxtra}
\usepackage{amsfonts}
\usepackage{graphicx}
\usepackage{color,colortbl}
\usepackage{enumerate}
\usepackage{algpseudocode}
\usepackage{algorithm}

% misc
\newcommand\todo[1]{\textcolor{red}{TODO: #1}}
\newcommand\hide[1]{\textcolor{white}{#1}}

% formatting
\newcommand\bld[1]{\textbf{#1}}
\newcommand\ul[1]{\underline{#1}}
\newcommand\n[1]{\numprint{#1}}
\newcommand{\ts}{\textsuperscript}
\newcommand\red[1]{\textcolor{red}{#1}}
\newcommand\blue[1]{\textcolor{blue}{#1}}
\newcommand\link[2]{\href{#1}{\textcolor{blue}{\underline{#2}}}}

% sets
\newcommand\set[1]{\mathcal{#1}}
\newcommand\bb[1]{\mathbb{#1}}
\renewcommand\:{\colon} % for use with \sset, etc.
\newcommand{\sset}[1]{\left\{\,#1\,\right\}} % { ? }, automatic brackets
\newcommand{\ssets}[1]{\left\{#1\right\}} % {?}, automatic brackets
\newcommand{\ssetn}[1]{\{\,#1\,\}} % { ? }, normal brackets

% table formatting
% To better align bold entries in S columns (still broken)
% \usepackage{siunitx}
% \robustify\bfseries
% \newrobustcmd{\bfcell}{\bfseries}

% vector variables (taken from macros by Rainer Gemulla)
\newcommand\vect[1]{{\boldsymbol{#1}}}
\newcommand\va{\vect{a}}
\newcommand\vb{\vect{b}}
\newcommand\vc{\vect{c}}
\newcommand\vd{\vect{d}}
\newcommand\ve{\vect{e}}
\newcommand\vf{\vect{f}}
\newcommand\vg{\vect{g}}
\newcommand\vh{\vect{h}}
\newcommand\vi{\vect{i}}
\newcommand\vj{\vect{j}}
\newcommand\vk{\vect{k}}
\newcommand\vl{\vect{l}}
\newcommand\vm{\vect{m}}
\newcommand\vn{\vect{n}}
\newcommand\vo{\vect{o}}
\newcommand\vp{\vect{p}}
\newcommand\vq{\vect{q}}
\newcommand\vr{\vect{r}}
\newcommand\vs{\vect{s}}
\newcommand\vt{\vect{t}}
\newcommand\vu{\vect{u}}
\newcommand\vv{\vect{v}}
\newcommand\vw{\vect{w}}
\newcommand\vx{\vect{x}}
\newcommand\vy{\vect{y}}
\newcommand\vz{\vect{z}}
\newcommand\vzero{\vect{0}}
\newcommand\vone{\vect{1}}

\newcommand\valpha{\vect{\alpha}}
\newcommand\vbeta{\vect{\beta}}
\newcommand\veps{\vect{\epsilon}}
\newcommand\vdelta{\vect{\delta}}
\newcommand\vtheta{\vect{\theta}}
\newcommand\vsigma{\vect{\sigma}}
\newcommand\vpi{\vect{\pi}}
\newcommand\vlambda{\vect{\lambda}}

% matrix variables (taken from macros by Rainer Gemulla)
\newcommand\mA{\vect{A}}
\newcommand\mB{\vect{B}}
\newcommand\mC{\vect{C}}
\newcommand\mD{\vect{D}}
\newcommand\mE{\vect{E}}
\newcommand\mF{\vect{F}}
\newcommand\mG{\vect{G}}
\newcommand\mH{\vect{H}}
\newcommand\mI{\vect{I}}
\newcommand\mJ{\vect{J}}
\newcommand\mK{\vect{K}}
\newcommand\mL{\vect{L}}
\newcommand\mM{\vect{M}}
\newcommand\mN{\vect{N}}
\newcommand\mO{\vect{O}}
\newcommand\mP{\vect{P}}
\newcommand\mQ{\vect{Q}}
\newcommand\mR{\vect{R}}
\newcommand\mS{\vect{S}}
\newcommand\mT{\vect{T}}
\newcommand\mU{\vect{U}}
\newcommand\mV{\vect{V}}
\newcommand\mW{\vect{W}}
\newcommand\mX{\vect{X}}
\newcommand\mY{\vect{Y}}
\newcommand\mZ{\vect{Z}}
\newcommand\mzero{\vect{0}}

\newcommand{\mPi}{{\ensuremath{\vect{\Pi}}}}
\newcommand{\mSigma}{{\ensuremath{\vect{\Sigma}}}}
\newcommand{\mLambda}{{\ensuremath{\vect{\Lambda}}}}

% argmin, argmax
\DeclareMathOperator*{\argmin}{argmin} % amsmath package required
\DeclareMathOperator*{\argmax}{argmax} % amsmath package required

% matrix operations
\newcommand\xdiag{\operatorname{diag}}      
\newcommand\diag[1]{\xdiag\left(#1\right)}    % diagonal matrix


% choose how your presentation looks.
% for more themes, color themes and font themes, see:
% http://deic.uab.es/~iblanes/beamer_gallery/index_by_theme.html

\mode<presentation>
{%
	\usetheme{default}      % or try Darmstadt, Madrid, Warsaw, ...
	\usecolortheme{default} % or try albatross, beaver, crane, ...
	\usefonttheme{default}  % or try serif, structurebold, ...
	\setbeamertemplate{navigation symbols}{}
	\setbeamertemplate{caption}[numbered]
	% For a numbered table of contents
	\setbeamertemplate{section in toc}[sections numbered] 
	% For slide numbers
	\addtobeamertemplate{navigation symbols}{}{%
	\usebeamerfont{footline}
	\usebeamercolor[fg]{footline}
	\hspace{1em}
	\insertframenumber%/\inserttotalframenumber
	}
} 

%% so table of content appears before each section, highlighting what's next
%\AtBeginSection[]
%{%
%	\setbeamercolor{section in toc shaded}{fg=structure}
%	\begin{frame}<beamer>
%	  \frametitle{Outline}
%	  \tableofcontents[currentsection]
%	\end{frame}
%}

% adds title slides for each section
\AtBeginSection[]{
  \begin{frame}
  \vfill
  \centering
  \begin{beamercolorbox}[sep=8pt,center,shadow=true,rounded=true]{title}
    \usebeamerfont{title}\Huge\insertsectionhead\par%
  \end{beamercolorbox}
  \vfill
  \end{frame}
}

\title[Write your short title here]{Advanced Methods in Text Analytics}
\subtitle{Exercise 01: ML Basics}
\author{Daniel Ruffinelli}
\institute{University of Mannheim}
\date{FSS 2025}

\begin{document}

% no "Figure X" prefix in image captions when using the figure environment
\setbeamertemplate{caption}{\raggedright\insertcaption\par}

\begin{frame}
    \titlepage{}
\end{frame}

\begin{frame}{About These Tutorials}{}
    \begin{itemize}
        \item \textbf{Goals}
              \begin{itemize}
                  \item \emph{Reinforce/deepen} content from lectures
                  \item \emph{Introduce new concepts} to support lecture content
                  \item Get some hands-on experience with \emph{code}
              \end{itemize}
        \item \textbf{Tutorial slides}
              \begin{itemize}
                  \item These are the type of slides I'll use for tutorials
                  \item They are mostly a copy of exercise sheets in slide
                        format
                  \item I often make notes on them during tutorials
                  \item I can't guarantee notes are clear without being here
                  \item I also want to promote that students come to tutorials
                  \item So, \textbf{I will not make these slides publicly
                            available}
              \end{itemize}
    \end{itemize}
\end{frame}

\begin{frame}{Mathematical Notation}{}
    Notation we will use throughout the semester:
    \begin{itemize}
        \item Scalars: $a,b\in\bb{R}$
        \item Vectors: $\vx,\vy\in\bb{R}^n$ (by default, vectors are all \emph{column}
              vectors, i.e.\ size is $n\times 1$)
        \item Matrices: $\mX,\mY\in\bb{R}^{m\times n}$
        \item Scalar product: $a\vx\in\bb{R}^n$
        \item Dot product: $\vx^\top\vy\in\bb{R}$ (some sources call this a scalar product)
        \item Matrix-vector product: $\mX\vx\in\bb{R}^{m\times 1}$
        \item Matrix-matrix product: $\mX^\top\mY\in\bb{R}^{n\times n}$
    \end{itemize}
\end{frame}

\begin{frame}{Machine Learning Basics}{Context}
    \begin{itemize}
        \item In machine learning, we typically develop models that can be
              trained to perform
              a specific task.
        \item Formally, a model can be represented by a function
              $f_{\vtheta}\colon \bb{R}^n\mapsto \bb{R}^m$, where
              $\vtheta\in\bb{R}^k$ are the parameters of the model.
        \item For a given $\vx\in\bb{R}^n$, we can write this as follows:
              $f(\vx\mid\vtheta) = \vy$.
    \end{itemize}
\end{frame}

\begin{frame}{Machine Learning Basics}{Question a)}
    What part of the basic machine learning pipeline do the symbols in the
    formal description above denote?
    Specifically:
    \begin{enumerate}[(i)]
        \item What does $\vx$ represent? And each of $\vx_i$?
        \item What is $\vy$ and what is it used for?
        \item What is $f(\vx\mid\vtheta)$? Explain this notation.
        \item What is a possible value for the input dimension $n$?
              And for the output dimension $m$? What factors determine
              these values?
    \end{enumerate}
\end{frame}

\begin{frame}{Machine Learning Basics}{Answer a)}
    \begin{enumerate}[(i)]
        \item $\vx$ is input vector, typically represents point in some dataset.
              Input vectors often referred to as \emph{feature vectors}, their
              corresponding vector space $\bb{R}^n$ as a \emph{feature space}.
              These names derived from each $\vx_i$ being referred to
              as a \emph{feature}.
        \item Output of our function, used for inference or to make
              predictions with the model.
        \item This is read as ``function $\vh$ of $\vx$ \emph{given}
              parameters $\vtheta$''.
        \item $n$ can take any value, typically in $\bb{N}$, this is
              typically a design decision, e.g.\ number of features we
              manually crafted, number of PCA components used for
              dimensionality reduction, the size of each hidden layer
              in  a feed-forward neural networm (FNN). \\
              $m$ can also take any value in $\bb{N}$, i.e.\
              $\vy\in\bb{R}^d$, i.e.\ $d$ not necessarily equal to
              $n$, less a design decision, more a property of
              the task we are modeling. E.g.\ for binary classification,
              we typically have $m=1$, for classification
              problem with $k$ classes, $m=k$.
    \end{enumerate}
\end{frame}

\begin{frame}{Machine Learning Basics}{Question b)}
    \begin{itemize}
        \item \textbf{Question:}
              \begin{itemize}
                  \item What are the common types of machine learning tasks in
                        natural language processing based on the output a model
                        produces?
                  \item What does the output look like for each of these task
                        types?
              \end{itemize}
              \pause
        \item \textbf{Answer:}
              \begin{itemize}
                  \item \textbf{Classification:} output is categorical, e.g.\
                        part-of-speech (POS) tagging or next-word prediction.
                  \item \textbf{Regression:} output is a real number, e.g.\ in
                        automated essay scoring.
              \end{itemize}
    \end{itemize}
\end{frame}

\begin{frame}{Machine Learning Basics}{Question c)}
    \begin{itemize}
        \item \textbf{Question:}
              \begin{itemize}
                  \item How does \textbf{``learning''} take place in machine
                        learning?
                  \item What is the \textbf{goal} of this learning process?
              \end{itemize}
              \pause
        \item \textbf{Answer:}
              \begin{itemize}
                  \item \textbf{Learning:} estimating parameters $\vtheta$
                        of function $f$ in order to solve an optimization
                        problem, usually minimizing some training objective
                  \item \textbf{Goal:} not to solve this optimization problem,
                        but to perform well on unseen data, i.e.\ to perform a
                        task well \emph{generally}.
              \end{itemize}
    \end{itemize}
\end{frame}

\begin{frame}{Machine Learning Basics}{Question d)}
    \begin{itemize}
        \item \textbf{Question:}
              \begin{itemize}
                  \item What are the \textit{hyperparameters} of a machine
                        learning model?
                        Provide an example.
                  \item How do we find the optimal values for a model's
                        hyperparameters?
              \end{itemize}
              \pause
        \item \textbf{Answer:}
              \begin{itemize}
                  \item Parameters that we do not learn, but manually set.
                  \item These are often not parameters of the model, but they
                        related to the training setting.
                  \item For example:
                        \begin{itemize}
                            \item Using a bias or not
                            \item The choice of a loss function
                            \item The choice of a model
                        \end{itemize}
                  \item We typically find useful values for hyperparameters by
                        conducting grid search or random search, although more
                        involved methods exist.
              \end{itemize}
    \end{itemize}
\end{frame}

\begin{frame}{Machine Learning Basics}{Question e)}
    \begin{itemize}
        \item Why do machine learning pipelines typically rely on
              datasets split into three parts: \textbf{training},
              \textbf{validation}, and \textbf{test} sets?
        \item What is the \textbf{role} of each of these data splits?
    \end{itemize}
\end{frame}

\begin{frame}{Machine Learning Basics}{Answer e)}
    \begin{itemize}
        \item Related to goal of ML: models should
              generalize well.
        \item Datasets usually split into \textbf{training} and
              \textbf{test} splits, so we may estimate parameters of our
              models on training data, them evaluate our models on
              unseen data using the test set.
        \item But we make decisions before training (hyperparameters)
        \item Generally, we want know how these decisions impact our
              model, but when comparing different hyperparameter
              settings on test data, we would identify which one
              works best on \emph{that} test set, which would make it
              no longer unseen.
        \item \textbf{Validation} split exists to allow for model
              selection without compromising test data, i.e.\ it's
              evaluation data we can use to train various models under
              different settings and choose the settings that give the
              best performace.
        \item To ensure all splits come from same distribution, splits
              constructed by shuffling a dataset. Then for each split, randomly
              sampling examples without replacement.
    \end{itemize}
\end{frame}

\begin{frame}{Machine Learning Basics}{Question f)}
    \begin{itemize}
        \item \textbf{Question:}
              \begin{itemize}
                  \item Briefly explain the concepts of \textbf{underfitting}
                        and \textbf{overfitting}.
                  \item Why do they often occur?
                  \item How may we recognize them in practice?
              \end{itemize}
              \pause
        \item \textbf{Answer:}
              \begin{itemize}
                  \item \textbf{Underfitting}: model is too simple to perform
                        well on a task, even on the data it is trained on.
                  \item \textbf{Overfitting:} model is capable of
                        performing task well to the point it learns
                        all the detail/noise in training data, while still not
                        performing well on unseen data
                  \item We can \emph{measure underfitting} by looking at the
                        model's performance on the training set, e.g.\ compute
                        accuracy on training set in a classification setting
                  \item We can \textbf{measure overfitting} by looking at
                        model performance on unseen data, e.g.\ a validation
                        split, and contrasting it with its performance on
                        training data.
                  \item High performance during training but low performance on
                        unseen data: \emph{clear indication of overfitting},
                        often not so straightforward, this gap is common,
                        baselines required
              \end{itemize}
    \end{itemize}
\end{frame}

\begin{frame}{Log-linear Classifiers}{Context}
    \begin{itemize}
        \item Linear classifiers are those that model the classification task as
              a \emph{linear combination} of input features and model
              parameters, where the model parameters act as the \emph{weights} of
              that linear combination.
        \item In this task, we review some log-linear classifiers commonly used
              in NLP.
    \end{itemize}
\end{frame}

\begin{frame}{Log-linear Classifiers}{Question a)}
    \begin{itemize}
        \item Recall that the logistic regression model is a linear classifier
              designed for binary classification.
        \item We introduced this model in the
              lectures with the following equations:
              \begin{align}
                  p(y=1\mid x) & = \sigma(w x + b)     \\
                  p(y=0\mid x) & = 1 - \sigma(w x + b)
              \end{align}
              where $x,y,w,b\in\bb{R}$, $\vtheta=(w,b)$ are model parameters,
              $x$ is the input and $\sigma$ is the logistic (sigmoid) function
              defined as follows:
              \begin{align}\label{eq:sigmoid}
                  \sigma(z) = \frac{\exp{(z)}}{1 + \exp{(z)}}
              \end{align}
        \item Give a formal expression for a logistic regression model that
              takes as input a vector $\vx\in\bb{R}^n$ and that does not use a
              bias term.
        \item How many parameters does this model have?
    \end{itemize}
\end{frame}

\begin{frame}{Log-linear Classifiers}{Answer a)}
    \begin{itemize}
        \item Given that logistic regression is a linear classifier, a logistic
              regression model that takes an $n$-dimensional vector as input,
              and does not use a bias, can be formally described as follows:
              \begin{align}
                  p(y=1\mid x) & = \sigma\left(\sum_{i=1}^{n}w_i\cdot x_i\right) \\
                  p(y=0\mid x) & = 1-p(y=1),
              \end{align}
              where $w_i\in\bb{R}$ is the parameter (weight) that corresponds to
              input feature $x_i$.
        \item Being a linear combination of the input features, this model has
              as many parameters as there are features in the input vector.
    \end{itemize}
\end{frame}

\begin{frame}{Log-linear Classifiers}{Question b)}
    \begin{itemize}
        \item \textbf{Question:}
              \begin{itemize}
                  \item If not done already, write the same logistic regression
                        model from the previous question but in vectorized form,
                        i.e.\ as a function of its parameter vector $\vw$.
              \end{itemize}
              \pause
        \item \textbf{Answer:}
              \begin{itemize}
                  \item We have:
                        \begin{align}
                            p(y=1\mid x) & = \sigma(\vw^\top\vx).
                        \end{align}
              \end{itemize}
    \end{itemize}
\end{frame}

\begin{frame}{Log-linear Classifiers}{Question c)}
    \begin{itemize}
        \item \textbf{Question:}
              \begin{itemize}
                  \item Now add a bias term to your logistic regression model.
                  \item Can we still write it in vectorized form?
              \end{itemize}
              \pause
        \item \textbf{Answer:}
              \begin{itemize}
                  \item We have:
                        \begin{align}
                            p(y=1\mid x) & = \sigma\left(\sum_{i=1}^{n}(w_i\cdot x_i) + b\right) \\
                            p(y=0\mid x) & = 1-p(y=1),
                        \end{align}
                        where $b\in\bb{R}$ is the bias term.
                  \item We can still write this in vectorized form, but to do
                        this comfortably we make use of what we call a
                        \emph{bias feature} $x_0=1$, which acts as the
                        coefficient for the bias weight in the linear
                        combination. That is:
                        \begin{align}
                            p(y=1\mid x) & = \sigma(\vw^\top\vx).
                        \end{align}
                        where $\vx\in\bb{R}^{n+1}$, i.e.\
                        $\vx=(x_o, x_1, \ldots, x_n)$.
              \end{itemize}
    \end{itemize}
\end{frame}

\begin{frame}{Log-linear Classifiers}{Question d)}
    \begin{itemize}
        \item \textbf{Question:}
              \begin{itemize}
                  \item \emph{Log}-linear models are those that model
                        classification as a linear combination of $\vx$ and
                        $\vw$ with the exponential function applied to it.
                  \item Give a formal expression for a \emph{general} log-linear
                        model $f(y\mid\vx)$ designed for binary classification
                        and parameterized by $\vw$, where $\vx,\vw\in\bb{R}^n$.
              \end{itemize}
              \pause
        \item \textbf{Answer:}
              \begin{itemize}
                  \item We have:
                        \begin{align}
                            f(y=1\mid\vx) & = \exp(\vw^\top\vx) = e^{\vw^\top\vx}, \\
                            f(y=0\mid\vx) & = 1 - f(y=1\mid\vx).
                        \end{align}
              \end{itemize}
    \end{itemize}
\end{frame}

\begin{frame}{Log-linear Classifiers}{Question e)}
    \begin{itemize}
        \item \textbf{Question:}
              \begin{itemize}
                  \item Now generalize your answer from the previous question so
                        the model is designed for \emph{multi-class}
                        classification.
                  \item How many parameters does this model have?
              \end{itemize}
              \pause
        \item \textbf{Answer:}
              \begin{itemize}
                  \item For $c\in C$ classes, we have:
                        \begin{align}\label{eq:log_linear}
                            f(y=c\mid\vx) & = \exp(\vw_c^\top\vx) = e^{\vw_c^\top\vx},
                        \end{align}
                        where $\vw_c$ are the parameters used for making
                        predictions for class $c$.
                        That is, this general log-linear model has as many
                        parameters as there are input features
                        \emph{for each class} $c\in C$.
              \end{itemize}
    \end{itemize}
\end{frame}

\begin{frame}{Log-linear Classifiers}{Question f)}
    \begin{itemize}
        \item If not done already, rewrite your model from the
              previous question so it's a probabilistic model, i.e.\
              for each input vector $\vx$ and class $c$, the models
              provides the probability that the input belongs to that
              given class $c$.
        \item Can you do this in vectorized form? I.e.\ can you define
              a function that returns a distribution over classes for
              a given input vector $\vx$?
    \end{itemize}
\end{frame}

\begin{frame}{Log-linear Classifiers}{Answer f) (1)}
    \begin{itemize}
        \item We can accomplish this by normalizing the output of the model, so
              that it represents a probability distribution over the classes.
        \item Concretely, we have:
              \begin{align}\label{eq:softmax}
                  p(y=c\mid\vx) & = \frac{\exp(\vw_c^\top\vx)}{\sum_{c'\in C}\exp(\vw_{c'}^\top\vx)},
              \end{align}
              where $C$ is the set of classes in our classification task.

        \item Eq.~\ref{eq:softmax} is known as a \emph{softmax function}, which
              gives the name to this multi-class log-linear model: softmax
              regression.
        \item Note that given input vector $\vx$ and class $c$, the function
              returns a single probability value.
        \item That means, for a given input vector
              $\vx$, we can get an entire ``softmax distribution'' over classes
              by computing this function for all classes $c\in C$.
    \end{itemize}
\end{frame}

\begin{frame}{Log-linear Classifiers}{Answer f) (2)}
    \begin{itemize}
        \item Then, by normalizing each entry in that vector as with
              Eq.~\ref{eq:softmax}, we get a probability vector $\vy\in\bb{R}^C$
              that contains the distribution over all classes, i.e.\ all
              of its entries add up to 1.
        \item That is:
              \begin{align}
                  \vy = softmax(\vl),
              \end{align}
              where the $i$-th component of $\vy$ is given by:
              \begin{align}
                  \vy_i = \frac{e^{l_i}}{\sum_{j=1}^{C}e^{l_j}}.
              \end{align}
        \item This is typically the way the softmax function is represented in
              vectorized form, a very interpretable output
    \end{itemize}
\end{frame}

\begin{frame}{Log-linear Classifiers}{Question g)}
    \begin{itemize}
        \item \textbf{Question:}
              \begin{itemize}
                  \item Let's put it all together now. Given
                        $\mX\in\bb{R}^{m\times n}$ that represents $m$ examples
                        of $n$ dimensions, write a formal expression for a
                        general log-linear model designed for multi-class
                        classification over $C$ classes
                        \emph{in vectorized form} (assume no biases).
              \end{itemize}
              \pause
        \item \textbf{Answer:}
              \begin{itemize}
                  \item Let $\mW\in\bb{R}^{n\times C}$ be parameters of our
                        model such that column $\vw_i$ corresponds to class
                        $i\in C$.
                  \item Using our vectorized softmax function from the previous
                        question, we have:
                        \begin{align}
                            \mY = softmax(\mX\mW),
                        \end{align}
                        where we assume our softmax operation is applied
                        row-wise.
              \end{itemize}
    \end{itemize}
\end{frame}

\begin{frame}{Log-linear Classifiers}{Question h)}
    \begin{itemize}
        \item \textbf{Question:}
              \begin{itemize}
                  \item Can you guess why we use \emph{log}-probabilities in
                        log-linear models instead of just probabilities as in
                        linear classifiers?
              \end{itemize}
              \pause
        \item \textbf{Answer:}
              \begin{itemize}
                  \item There are a few reasons for this, but in short, there
                        are some computational reasons, and some statistical
                        reasons.
                  \item The computational reasons are (i) it simplifies the math
                        and therefore computations, e.g.\ when computing
                        likehoods of multiple data points, and (ii) it helps to
                        avoid numerical underflow issues that can occur when
                        multiplying many probabilities together.
                  \item The main statistical reason is that it makes the model
                        more interpretable, as it can be related to concepts
                        like odds and independence.
              \end{itemize}
    \end{itemize}
\end{frame}

\begin{frame}{Training Log-linear Classifiers}{Context}
    \begin{itemize}
        \item In this task, we are interested in training a probabilistic
              log-linear model, like those from the previous task, using maximum
              likelihood estimation (MLE).
    \end{itemize}
\end{frame}

\begin{frame}{Training Log-linear Classifiers}{Question a)}
    \begin{itemize}
        \item \textbf{Question:}
              \begin{itemize}
                  \item In MLE, we aim to maximize the likelihood of a model
                        given some data.
                  \item Let $X=\{x_1, x_2, \ldots, x_n\}$ be a dataset of $n$
                        examples and $p$ a probabilistic log-linear model
                        parameterized by $\vw\in\bb{R}^d$ and designed for
                        binary classification.
                  \item Given the following joint distribution of the data as
                        given by the model:
                        \begin{align}
                            p(X\mid\vw) = p(x_1, x_2, \ldots, x_n),
                        \end{align}
                        rewrite this expression assuming that all examples are
                        independent and identically distributed (i.i.d.).
              \end{itemize}
              \pause
        \item \textbf{Answer:}
              \begin{itemize}
                  \item We have:
                        \begin{align}
                            p(X\mid\vw) = p(x_1\mid\vw)p(x_2\mid\vw)\ldots p(x_n\mid\vw) = \prod_{i=1}^{n}p(x_i\mid\vw).
                        \end{align}
              \end{itemize}
    \end{itemize}
\end{frame}

\begin{frame}{Training Log-linear Classifiers}{Question b)}
    \begin{itemize}
        \item \textbf{Question:}
              \begin{itemize}
                  \item For clarity, let $X_{pos}$ be the set examples from the
                        positive class, and $X_{neg}$ the set of examples from
                        the negative class.
                  \item Rewrite your answer from the previous question as a
                        function of $X_{pos}$ and $X_{neg}$.
              \end{itemize}
              \pause
        \item \textbf{Answer:}
              \begin{itemize}
                  \item We have:
                        \begin{align}
                            p(X\mid\vw) = \prod_{x_p\in X_{pos}}p(x_p\mid\vw)\prod_{x_n\in X_{neg}}\left(1 - p(x_n\mid\vw)\right).
                        \end{align}
              \end{itemize}
    \end{itemize}
\end{frame}

\begin{frame}{Training Log-linear Classifiers}{Question c)}
    \begin{itemize}
        \item \textbf{Question:}
              \begin{itemize}
                  \item Write the corresponding likelihood function $\set{L}$
                        for the model $p$ described in your previous answer.
              \end{itemize}
              \pause
        \item \textbf{Answer:}
              \begin{itemize}
                  \item We have:
                        \begin{align}
                            \set{L}(\vw\mid X) = \prod_{x_p\in X_{pos}}p(x_p\mid\vw)\prod_{x_n\in X_{neg}}\left(1 - p(x_n\mid\vw)\right).
                        \end{align}
              \end{itemize}
    \end{itemize}
\end{frame}

\begin{frame}{Training Log-linear Classifiers}{Question d)}
    \begin{itemize}
        \item For reasons discussed in the previous task, we often
              work with log-probabilities instead of probabilities.
        \item Write the log-likelihood function $\set{LL}$ for the
              model $p$.
        \item Use $\ln$ throughout.
    \end{itemize}
\end{frame}

\begin{frame}{Training Log-linear Classifiers}{Answer d)}
    \begin{itemize}
        \item We have:
              \begin{align}
                  \set{LL}(\vw\mid X) & = \ln p(X\mid\vw)                                                                                                            \\
                                      & = \ln \left(\prod_{i=1}^{n}p(x_i\mid\vw)\right)                                                                              \\
                                      & = \ln \left(\prod_{x_p\in X_{pos}}p(x_p\mid\vw)\prod_{x_n\in X_{neg}}\left(1 - p(x_n\mid\vw)\right)\right)                   \\
                                      & = \sum_{x_p\in X_{pos}}\ln p(x_p\mid\vw) + \sum_{x_n\in X_{neg}}\ln\left(1 - p(x_n\mid\vw)\right). \label{eq:log_linear_mle}
              \end{align}
        \item This is the general expression we maximize when training
              log-linear models with MLE using the i.i.d. assumption.
        \item We want to find value of $\vw$ that maximizes
              Eq.~\ref{eq:log_linear_mle}.
    \end{itemize}
\end{frame}

\begin{frame}{Training Log-linear Classifiers}{Question e)}
    \begin{itemize}
        \item Assume $p$ is a logistic regression model and each data
              point in $X$ is represented by $\vx_i\in\bb{R}^d$.
        \item Rewrite the log-likelihood function you wrote in the
              previous answer by including the logistic regression
              model explicitly, i.e.\ using the sigmoid function in
              Eq.~\ref{eq:sigmoid}.
    \end{itemize}
\end{frame}

\begin{frame}{Training Log-linear Classifiers}{Answer e)}
    \begin{itemize}
        {\small
        \item We have:
              \begin{align}
                  \set{LL}(\vw\mid X) & = \sum_{x_p\in X_{pos}}\ln\sigma(\vw^\top\vx) + \sum_{x_n\in X_{neg}}\ln\left(1 - \sigma(\vw^\top\vx)\right)                                                          \\
                                      & = \sum_{x_p\in X_{pos}}\ln \frac{e^{\vw^\top\vx_p}}{1 + e^{\vw^\top\vx_p}} + \sum_{x_n\in X_{neg}}\ln\left(1 - \frac{e^{\vw^\top\vx_n}}{1 + e^{\vw^\top\vx_n}}\right) \\
                                      & = \sum_{x_p\in X_{pos}}\ln \frac{e^{\vw^\top\vx_p}}{1 + e^{\vw^\top\vx_p}} + \sum_{x_n\in X_{neg}}\ln\left(\frac{1}{1 + e^{\vw^\top\vx_n}}\right)                     \\
                                      & = \sum_{x_p\in X_{pos}} \left(\vw^\top\vx_p - \ln(1 + e^{\vw^\top\vx_p})\right) - \sum_{x_n\in X_{neg}}\ln\left(1 + e^{\vw^\top\vx_n}\right).
              \end{align}
              }
    \end{itemize}
\end{frame}

\begin{frame}{Training Log-linear Classifiers}{Question f)}
    \begin{itemize}
        \item As seen in the MLE example from the lecture, we will need to
              compute the gradient to maximize our expression.
        \item Compute the gradient of the likelihood function for your logistic
              regression model w.r.t.\ model parameters $\vw$.
        \item That is, $\nabla_{\vw}\set{LL}$ where $\nabla$ is defined as:
              \begin{align}\label{eq:gradient}
                  \nabla_{\vw}\set{LL} = \left[\frac{\partial\set{LL}}{\partial w_1},
                      \frac{\partial\set{LL}}{\partial w_2}, \ldots, \frac{\partial\set{LL}}{\partial w_d}\right].
              \end{align}
    \end{itemize}
\end{frame}

\begin{frame}{Training Log-linear Classifiers}{Answer f)}
    \begin{itemize}
        {\tiny
        \item Let $\nabla=\nabla_{\vw}$. We have:
              \begin{align}
                  \nabla\set{LL} & = \nabla \left(\sum_{x_p\in X_{pos}} \left(\vw^\top\vx_p - \ln(1 + e^{\vw^\top\vx_p})\right) - \sum_{x_n\in X_{neg}}\ln\left(1 + e^{\vw^\top\vx_n}\right)\right)                           \\
                                 & = \nabla\sum_{x_p\in X_{pos}} \left(\vw^\top\vx_p - \ln(1 + e^{\vw^\top\vx_p})\right) - \nabla\sum_{x_n\in X_{neg}}\ln\left(1 + e^{\vw^\top\vx_n}\right)                                   \\
                                 & = \sum_{x_p\in X_{pos}} \left(\vx_p - \nabla\ln(1 + e^{\vw^\top\vx_p})\right) - \sum_{x_n\in X_{neg}}\nabla\ln\left(1 + e^{\vw^\top\vx_n}\right)                                           \\
                                 & = \sum_{x_p\in X_{pos}} \left(\vx_p - \frac{e^{\vw^\top\vx_p}}{1 + e^{\vw^\top\vx_p}}\vx_p\right) - \sum_{x_n\in X_{neg}}\left(\frac{e^{\vw^\top\vx_n}}{1 + e^{\vw^\top\vx_n}}\vx_n\right) \\
                                 & = \sum_{x_p\in X_{pos}} \left(1 - \frac{e^{\vw^\top\vx_p}}{1 + e^{\vw^\top\vx_p}}\right)\vx_p - \sum_{x_n\in X_{neg}}\left(\frac{e^{\vw^\top\vx_n}}{1 + e^{\vw^\top\vx_n}}\vx_n\right)     \\
                                 & = \sum_{x_p\in X_{pos}} \left(1 - \sigma(\vw^\top\vx_p)\right)\vx_p - \sum_{x_n\in X_{neg}}\sigma(\vw^\top\vx_n)\vx_n.
              \end{align}
              }
    \end{itemize}
\end{frame}

\begin{frame}{Training Log-linear Classifiers}{Question g) (1)}
    \begin{itemize}
        \item When MLE has no closed-form solution (often), we use
              gradient-based methods, e.g.\ gradient descent (GD):
              {\small
              \begin{algorithm}[H]
                  \caption{Gradient Descent}
                  \label{alg:gd}
                  \begin{algorithmic}[1]
                      \Procedure{Gradient Descent}{Dataset $X$, Objective function $J$, Model $p_{\vw}$, Learning rate $lr$, Number of Epochs $N$}
                      \State $\vw\leftarrow random()$ \qquad   // random weight initialization
                      \For{$i \in\{1,\ldots N\}$}
                      \State $g_i\leftarrow \nabla_{\vw}J(p_{\vw}(X))$
                      \State $\vw_{i+1}\leftarrow \vw_i - lr \cdot g_i$ \qquad   // update rule
                      \EndFor
                      \State{\textbf{Return} $p_{\vw}$}
                      \EndProcedure
                  \end{algorithmic}
              \end{algorithm}
              }
        \item Given previous questions, $p_{\vw}$ is logistic regression, $J$ is
              answer to question (e), $g$ is Eq.~\ref{eq:gradient}.
    \end{itemize}
\end{frame}

\begin{frame}{Training Log-linear Classifiers}{Question g) (2)}
    \begin{itemize}
        \item Modify Algorithm~\ref{alg:gd} so it performs (i) stochastic
              gradient descent (\textbf{SGD}) and (ii) mini-batch gradient
              descent (\textbf{BGD}).
              Recall that the former uses a single example for each update, and
              the latter a subset of the data for each update.
        \item Make sure you define each new function or component you use in the
              algorithm.
    \end{itemize}
\end{frame}

\begin{frame}{Training Log-linear Classifiers}{Answer g) (1)}
    \begin{itemize}
        \item \textbf{SGD}:
              {\small
              \begin{algorithm}[H]
                  \caption{Stochastic Gradient Descent}
                  \label{alg:sgd}
                  \begin{algorithmic}[1]
                      \Procedure{Gradient Descent}{Dataset $X$, Objective function $J$, Model $p_{\vw}$, Learning rate $lr$, Number of Epochs $N$}
                      \State $\vw\leftarrow random()$ \qquad   // random weight initialization
                      \For{$i \in\{1,\ldots N\}$} \qquad   // we iterate over epochs
                      \For{$x_i\in X$} \qquad   // we iterate over the dataset
                      \State $g_i\leftarrow \nabla_{\vw}J(p_{\vw}(x_i))$
                      \State $\vw_{i+1}\leftarrow \vw_i - lr \cdot g_i$ \qquad   // update rule
                      \EndFor
                      \EndFor
                      \State{\textbf{Return} $p_{\vw}$}
                      \EndProcedure
                  \end{algorithmic}
              \end{algorithm}
              }
    \end{itemize}
\end{frame}

\begin{frame}{Training Log-linear Classifiers}{Answer g) (2)}
    \begin{itemize}
        \item \textbf{BGD}:
              {\small
              \begin{algorithm}[H]
                  \caption{Batch Gradient Descent}
                  \label{alg:bgd}
                  \begin{algorithmic}[1]
                      \Procedure{Gradient Descent}{Dataset $X$, Objective function $J$, Model $p_{\vw}$, Learning rate $lr$, Number of Epochs $N$, Batch Size $b$}
                      \State $\vw\leftarrow random()$ \qquad   // random weight initialization
                      \For{$i \in\{1,\ldots N\}$}
                      \For{$batch = [x_1,\ldots,x_b]$ where $batch$ is sampled without replacement from $X$}
                      \State $g_i\leftarrow \nabla_{\vw}J(p_{\vw}(batch))$
                      \State $\vw_{i+1}\leftarrow \vw_i - lr \cdot g_i$ \qquad   // update rule
                      \EndFor
                      \EndFor
                      \State{\textbf{Return} $p_{\vw}$}
                      \EndProcedure
                  \end{algorithmic}
              \end{algorithm}
              }
    \end{itemize}
\end{frame}

\begin{frame}{Training Log-linear Classifiers}{Answer g) (3)}
    \begin{itemize}
        \item Note that we still rely on gradients that need to be computed
              every time we update our parameters.
        \item But we don't compute gradients manually when working with these
              models.
        \item This is generally handled by tools
              like \href{https://pytorch.org/tutorials/beginner/blitz/autograd_tutorial.html}{Autograd}.
        \item But it's important to understand the process underlying the
              training of models, despite it being automatic, so we can reason
              about model performance as a function of its training process.
    \end{itemize}
\end{frame}

\begin{frame}{Training Log-linear Classifiers}{Question h)}
    \begin{itemize}
        \item Assume that the function $\text{shuffle}(X)$ randomly shuffles the
              examples in dataset $X$.
        \item How would you modify your BGD algorithm to include this step?
        \item Do you know why shuffling is important?
    \end{itemize}
\end{frame}

\begin{frame}{Training Log-linear Classifiers}{Answer h) (1)}
    \begin{itemize}
        \item \textbf{BGD}:
              {\tiny
              \begin{algorithm}[H]
                  \caption{Batch Gradient Descent With Shuffling}
                  \label{alg:bgd_shuffled}
                  \begin{algorithmic}[1]
                      \Procedure{Gradient Descent}{Dataset $X$, Objective function $J$, Model $p_{\vw}$, Learning rate $lr$, Number of Epochs $N$, Batch Size $b$}
                      \State $\vw\leftarrow random()$ \qquad   // random weight initialization
                      \For{$i \in\{1,\ldots N\}$}
                      \For{$batch = [x_1,\ldots,x_b]$ where $batch$ is sampled without replacement from $X$}
                      \State $g_i\leftarrow \nabla_{\vw}J(p_{\vw}(batch))$
                      \State $\vw_{i+1}\leftarrow \vw_i - lr \cdot g_i$ \qquad   // update rule
                      \EndFor
                      \State shuffle($X$)
                      \EndFor
                      \State{\textbf{Return} $p_{\vw}$}
                      \EndProcedure
                  \end{algorithmic}
              \end{algorithm}
              }
    \end{itemize}
\end{frame}

\begin{frame}{Training Log-linear Classifiers}{Answer h) (2)}
    \begin{itemize}
        \item While GD considers the entire dataset when computing gradients,
              and SGD treats all examples equally, BGD uses a subset of the
              dataset to compute each gradient.
        \item Shuffling the dataset at the end of each epoch is a common
              practice to avoid using the same subset of examples for each BGD
              update, as this may bias the training process in favor of some
              examples over others.
        \item BGD is by far the most method used in NLP training.
        \item It has been empirically shown that SGD and BGD can converger
              faster than GD, but they can also be more unstable.
    \end{itemize}
\end{frame}

\end{document}
