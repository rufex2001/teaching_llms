\documentclass[11pt,a4paper]{article}

% packages
\usepackage[utf8]{inputenc}
\usepackage{amsmath}
\usepackage[T1]{fontenc}
\usepackage{setspace}
\usepackage{enumitem}
% \usepackage{amsmath}
\usepackage{booktabs}
\usepackage{fullpage} 
\usepackage{tabularx}
\usepackage{amssymb,amstext,amsmath}
\usepackage{fancyhdr}
\usepackage{graphicx}
% \usepackage{algorithmic}
% \usepackage[ruled,vlined]{algorithm2e}
\usepackage{url}
\usepackage[bookmarks,unicode=true,pdftex,a4paper]{hyperref}
\usepackage[round]{natbib}
\usepackage{algpseudocode}
\usepackage{algorithm}
\headsep1cm

% macros
% misc
\newcommand\todo[1]{\textcolor{red}{TODO: #1}}
\newcommand\hide[1]{\textcolor{white}{#1}}

% formatting
\newcommand\bld[1]{\textbf{#1}}
\newcommand\ul[1]{\underline{#1}}
\newcommand\n[1]{\numprint{#1}}
\newcommand{\ts}{\textsuperscript}
\newcommand\red[1]{\textcolor{red}{#1}}
\newcommand\blue[1]{\textcolor{blue}{#1}}
\newcommand\link[2]{\href{#1}{\textcolor{blue}{\underline{#2}}}}

% sets
\newcommand\set[1]{\mathcal{#1}}
\newcommand\bb[1]{\mathbb{#1}}
\renewcommand\:{\colon} % for use with \sset, etc.
\newcommand{\sset}[1]{\left\{\,#1\,\right\}} % { ? }, automatic brackets
\newcommand{\ssets}[1]{\left\{#1\right\}} % {?}, automatic brackets
\newcommand{\ssetn}[1]{\{\,#1\,\}} % { ? }, normal brackets

% table formatting
% To better align bold entries in S columns (still broken)
% \usepackage{siunitx}
% \robustify\bfseries
% \newrobustcmd{\bfcell}{\bfseries}

% vector variables (taken from macros by Rainer Gemulla)
\newcommand\vect[1]{{\boldsymbol{#1}}}
\newcommand\va{\vect{a}}
\newcommand\vb{\vect{b}}
\newcommand\vc{\vect{c}}
\newcommand\vd{\vect{d}}
\newcommand\ve{\vect{e}}
\newcommand\vf{\vect{f}}
\newcommand\vg{\vect{g}}
\newcommand\vh{\vect{h}}
\newcommand\vi{\vect{i}}
\newcommand\vj{\vect{j}}
\newcommand\vk{\vect{k}}
\newcommand\vl{\vect{l}}
\newcommand\vm{\vect{m}}
\newcommand\vn{\vect{n}}
\newcommand\vo{\vect{o}}
\newcommand\vp{\vect{p}}
\newcommand\vq{\vect{q}}
\newcommand\vr{\vect{r}}
\newcommand\vs{\vect{s}}
\newcommand\vt{\vect{t}}
\newcommand\vu{\vect{u}}
\newcommand\vv{\vect{v}}
\newcommand\vw{\vect{w}}
\newcommand\vx{\vect{x}}
\newcommand\vy{\vect{y}}
\newcommand\vz{\vect{z}}
\newcommand\vzero{\vect{0}}
\newcommand\vone{\vect{1}}

\newcommand\valpha{\vect{\alpha}}
\newcommand\vbeta{\vect{\beta}}
\newcommand\veps{\vect{\epsilon}}
\newcommand\vdelta{\vect{\delta}}
\newcommand\vtheta{\vect{\theta}}
\newcommand\vsigma{\vect{\sigma}}
\newcommand\vpi{\vect{\pi}}
\newcommand\vlambda{\vect{\lambda}}

% matrix variables (taken from macros by Rainer Gemulla)
\newcommand\mA{\vect{A}}
\newcommand\mB{\vect{B}}
\newcommand\mC{\vect{C}}
\newcommand\mD{\vect{D}}
\newcommand\mE{\vect{E}}
\newcommand\mF{\vect{F}}
\newcommand\mG{\vect{G}}
\newcommand\mH{\vect{H}}
\newcommand\mI{\vect{I}}
\newcommand\mJ{\vect{J}}
\newcommand\mK{\vect{K}}
\newcommand\mL{\vect{L}}
\newcommand\mM{\vect{M}}
\newcommand\mN{\vect{N}}
\newcommand\mO{\vect{O}}
\newcommand\mP{\vect{P}}
\newcommand\mQ{\vect{Q}}
\newcommand\mR{\vect{R}}
\newcommand\mS{\vect{S}}
\newcommand\mT{\vect{T}}
\newcommand\mU{\vect{U}}
\newcommand\mV{\vect{V}}
\newcommand\mW{\vect{W}}
\newcommand\mX{\vect{X}}
\newcommand\mY{\vect{Y}}
\newcommand\mZ{\vect{Z}}
\newcommand\mzero{\vect{0}}

\newcommand{\mPi}{{\ensuremath{\vect{\Pi}}}}
\newcommand{\mSigma}{{\ensuremath{\vect{\Sigma}}}}
\newcommand{\mLambda}{{\ensuremath{\vect{\Lambda}}}}

% argmin, argmax
\DeclareMathOperator*{\argmin}{argmin} % amsmath package required
\DeclareMathOperator*{\argmax}{argmax} % amsmath package required

% matrix operations
\newcommand\xdiag{\operatorname{diag}}      
\newcommand\diag[1]{\xdiag\left(#1\right)}    % diagonal matrix


% header and footer
\lhead{Advanced Methods in Text Analytics, FSS2025}
\chead{}
\rhead{\thepage\ }
\cfoot{}
\pagestyle{fancy}

\title{Advanced Methods in Text Analytics \\ 
Exercise 1: Machine Learning Basics \\ 
\textbf{Solutions}}
\author{Daniel Ruffinelli}
\date{FSS 2025}

\begin{document}
\maketitle

\section{Machine Learning Basics}

\begin{enumerate}[label=(\alph*)]
    \item
          \begin{enumerate}[label=(\roman*)]
              \item $\vx$ is an input vector, typically representing a data
                    point in some dataset. These vectors are often referred to
                    as feature vectors, and their corresponding vector space
                    $\bb{R}^n$ as a feature space.
                    These names are derived from each $\vx_i$ being referred to
                    as a \emph{feature}, which models use as input during
                    training and inference.
                    Thus, a feature vector is the set of features we use to
                    represent our datapoints when modeling our task of choice.
              \item The output of our function, used for inference or to make
                    predictions with the model.
              \item This is read as ``function $\vh$ of $\vx$ \emph{given}
                    parameters $\vtheta$''.
              \item $n$ can take any value, typically in $\bb{N}$. This value is
                    typically a design decision, e.g.\ the number of features we
                    manually crafted, the number of PCA components we use for
                    dimensionality reduction, or the size of each hidden layer
                    in a feed-forward neural network (more on this in future
                    lectures/tutorials). \\
                    $m$ can also take any value in $\bb{N}$, i.e.\
                    $\vy\in\bb{R}^d$, where $d$ is not necessarily equal to
                    $n$. This is less a design decision and more a property of
                    the task we are modeling. E.g.\ for binary classification,
                    we typically have $m=1$, whereas for a classification
                    problem with $k$ classes, $m=k$.
          \end{enumerate}
    \item \emph{Classification} when output is categorical, e.g.\ part-of-speech
          (POS) tagging or next-word prediction. \emph{Regression} when the
          output is a real number, e.g. in automated essay scoring.
    \item Learning refers to the process of estimating the parameters $\vtheta$
          of function $f$ in order to solve an optimization problem, usually the
          minimization of some training objective.
          The goal is not to solve this optimization problem as best as
          possible, but to perform well on unseen data, i.e.\ to perform a
          task well \emph{generally}.
    \item Parameters that we do not learn, but manually set. These are often not
          parameters of the model, but they related to the training setting.
          For example, using a bias or not with linear classifiers, the choice
          of a loss function, and even the choice of a model, can all be seen as
          hyperparameters.
          We typically find useful values for hyperparameters by conducting
          grid search or random search, although more involved methods exist.
    \item This related to the goal of machine learning, which is for models to
          generalize well to \emph{unseen} data.
          To that end, datasets are usually split into \emph{training} and
          \emph{test} splits, so that we may estimate the parameters of our
          models on training data, but evaluate our models performance on unseen
          data using the test set.
          However, we typically have to make a lot of decisions when training a
          model, e.g.\ which model to use, how to set the various
          hyperparameters in our setting, etc.
          Generally, we would like to know how these decisions impact our model
          performance, but if we compare two different hyperparameter settings
          on test data, e.g. whether to use a bias or not, then we would
          identify which one works best on that test data, which would make that
          data no longer unseen (we would already know something about it, e.g.\
          which settings work best for it).

          Validation split exists to allow for the model selection process
          without compromising the test data split. That is, this is the
          evaluation data we can use to train various models under different
          settings and choose the settings that yield the best performing model
          on the validation set, to then finally report its performance on the
          test set, which has remained unseen throughout the entire process
          (typically by training on the training and validation sets combined to
          get more data).
          Note that these splits are typically constructed by taking an existing
          dataset, shuffling its examples and then randomly sampling examples
          for each split without replacement, in order to ensure all splits come
          from the same distribution.
    \item We say underfitting occurs when a model is too simple to perform well
          on a task, even on the data it is trained on.
          Conversely, we say overfitting occurs when a model is clearly capable
          of performing well on a task to the point that it learns all the
          detail and noise in the data it was trained on, while still not
          performing well on unseen data.

          We can measure underfitting by looking at the model's performance on
          the training set, e.g.\ compute the accuracy of the training set in a
          classification setting.
          We can measure overfitting by looking at a model's performance on
          unseen data, e.g.\ a validation split and contrasting it with its
          performance on the training set. High performance during training but
          low performance on unseen data is a clear indication of overfitting.
          In practice, this is not so straightforward to see, as difference in
          performance across training and validation are common, so a baseline
          is required to determine whether the difference is
          significant/problematic.
\end{enumerate}

\section{Log-linear Classifiers}

\begin{enumerate}[label=(\alph*)]
    \item Given that logistic regression is a linear classifier, a logistic
          regression model that takes an $n$-dimensional vector as input, and
          does not use a bias, can be formally described with the following
          linear combination:
          \begin{align}
              p(y=1\mid x) & = \sigma\left(\sum_{i=1}^{n}w_i\cdot x_i\right) \\
              p(y=0\mid x) & = 1-p(y=1),
          \end{align}
          where $w_i\in\bb{R}$ is the parameter (weight) that corresponds to
          input feature $x_i$. Being a linear combination of the input features,
          this model has as many parameters as there are features in the input
          vector.
    \item We have:
          \begin{align}
              p(y=1\mid x) & = \sigma(\vw^\top\vx).
          \end{align}
    \item We have:
          \begin{align}
              p(y=1\mid x) & = \sigma\left(\sum_{i=1}^{n}(w_i\cdot x_i) + b\right) \\
              p(y=0\mid x) & = 1-p(y=1),
          \end{align}
          where $b\in\bb{R}$ is the bias term.

          We can still write this in vectorized form, but to do this comfortably
          we make use of what we call a \emph{bias feature} $x_0=1$, which acts
          as the coefficient for the bias weight in the linear combination. That
          is:
          \begin{align}
              p(y=1\mid x) & = \sigma(\vw^\top\vx).
          \end{align}
          where $\vx\in\bb{R}^{n+1}$, i.e.\ $\vx=(x_o, x_1, \ldots, x_n)$.
    \item We have:
          \begin{align}
              f(y=1\mid\vx) & = \exp(\vw^\top\vx) = e^{\vw^\top\vx}, \\
              f(y=0\mid\vx) & = 1 - f(y=1\mid\vx).
          \end{align}
    \item For $c\in C$ classes, we have:
          \begin{align}\label{eq:log_linear}
              f(y=c\mid\vx) & = \exp(\vw_c^\top\vx) = e^{\vw_c^\top\vx},
          \end{align}
          where $\vw_c$ are the parameters used for making predictions for class
          $c$.
          That is, this general log-linear model has as many parameters as there
          are input features \emph{for each class} $c\in C$.
    \item We can accomplish this by normalizing the output of the model, so that
          it represents a probability distribution over the classes.
          Concretely, we have:
          \begin{align}\label{eq:softmax}
              p(y=c\mid\vx) & = \frac{\exp(\vw_c^\top\vx)}{\sum_{c'\in C}\exp(\vw_{c'}^\top\vx)},
          \end{align}
          where $C$ is the set of classes in our classification task.

          Eq.~\ref{eq:softmax} is known as a \emph{softmax function}, which
          gives the name to this multi-class log-linear model: softmax
          regression.
          Note that given input vector $\vx$ and class $c$, the function returns
          a single probability value. That means that, for a given input vector
          $\vx$, we can get an entire ``softmax distribution'' over classes by
          computing this function for all classes $c\in C$.
          Specifically, we can compute $\vw_c^\top\vx$ for all $c\in C$ and 
          store these results in a $C$-dimensional vector $\vl\in\bb{R}^C$.
          Then, by normalizing each entry in that vector as with
          Eq.~\ref{eq:softmax}, we get a probability vector $\vy\in\bb{R}^C$
          that contains the distribution over all classes, i.e.\ where all of
          its entries add up to 1.
          That is:
          \begin{align}
              \vy = softmax(\vl),
          \end{align}
          where the $i$-th component of $\vy$ is given by:
          \begin{align}
              \vy_i = \frac{e^{l_i}}{\sum_{j=1}^{C}e^{l_j}}.
          \end{align}

          This is typically the way the softmax function is represented in
          vectorized form. 
          This is a very interpretable output for a classification model.
    \item Let $\mW\in\bb{R}^{n\times C}$ be parameters of our model such that
          column $\vw_i$ corresponds to class $i\in C$.
          Using our vectorized softmax function from the previous question, we
          have:
          \begin{align}
              \mY = softmax(\mX\mW),
          \end{align}
          where we assume our softmax operation is applied row-wise.
    \item There are a few reasons for this, but in short, there are some
          computational reasons, and some statistical reasons.
          The computational reasons are (i) it simplifies the math and therefore
          computations, e.g. when computing likehoods of multiple data points,
          and (ii) it helps to avoid numerical underflow issues that
          can occur when multiplying many probabilities together.
          The main statistical reason is that it makes the model more
          interpretable, as it can be related to concepts like odds and
          independence.
\end{enumerate}

\section{Training Log-linear Classifiers}

\begin{enumerate}[label=(\alph*)]
    \item We have:
          \begin{align}
              p(X\mid\vw) = p(x_1\mid\vw)p(x_2\mid\vw)\ldots p(x_n\mid\vw) = \prod_{i=1}^{n}p(x_i\mid\vw).
          \end{align}
    \item We have:
          \begin{align}
              p(X\mid\vw) = \prod_{x_p\in X_{pos}}p(x_p\mid\vw)\prod_{x_n\in X_{neg}}\left(1 - p(x_n\mid\vw)\right).
          \end{align}
    \item We have:
          \begin{align}
              \set{L}(\vw\mid X) = \prod_{x_p\in X_{pos}}p(x_p\mid\vw)\prod_{x_n\in X_{neg}}\left(1 - p(x_n\mid\vw)\right).
          \end{align}
    \item We have:
          \begin{align}
              \set{LL}(\vw\mid X) & = \ln p(X\mid\vw)                                                                                                             \\
                                  & = \ln \left(\prod_{i=1}^{n}p(x_i\mid\vw)\right)                                                                               \\
                                  & = \ln \left(\prod_{x_p\in X_{pos}}p(x_p\mid\vw)\prod_{x_n\in X_{neg}}\left(1 - p(x_n\mid\vw)\right)\right)                    \\
                                  & = \sum_{x_p\in X_{pos}}\ln p(x_p\mid\vw) + \sum_{x_n\in X_{neg}}\ln\left(1 - p(x_n\mid\vw)\right). \label{eq:log_linear_mle}
          \end{align}
          This is the general expression we want to maximize when training log-linear models with MLE using the i.i.d.
          assumption. Specifically, we want to find the value of $\vw$ that maximizes Eq.~\ref{eq:log_linear_mle}.
    \item We have:
          \begin{align}
              \set{LL}(\vw\mid X) & = \sum_{x_p\in X_{pos}}\ln\sigma(\vw^\top\vx) + \sum_{x_n\in X_{neg}}\ln\left(1 - \sigma(\vw^\top\vx)\right)                                                          \\
                                  & = \sum_{x_p\in X_{pos}}\ln \frac{e^{\vw^\top\vx_p}}{1 + e^{\vw^\top\vx_p}} + \sum_{x_n\in X_{neg}}\ln\left(1 - \frac{e^{\vw^\top\vx_n}}{1 + e^{\vw^\top\vx_n}}\right) \\
                                  & = \sum_{x_p\in X_{pos}}\ln \frac{e^{\vw^\top\vx_p}}{1 + e^{\vw^\top\vx_p}} + \sum_{x_n\in X_{neg}}\ln\left(\frac{1}{1 + e^{\vw^\top\vx_n}}\right)                     \\
                                  & = \sum_{x_p\in X_{pos}} \left(\vw^\top\vx_p - \ln(1 + e^{\vw^\top\vx_p})\right) - \sum_{x_n\in X_{neg}}\ln\left(1 + e^{\vw^\top\vx_n}\right).
          \end{align}
    \item Let $\nabla=\nabla_{\vw}$. We have:
          \begin{align}
              \nabla\set{LL} & = \nabla \left(\sum_{x_p\in X_{pos}} \left(\vw^\top\vx_p - \ln(1 + e^{\vw^\top\vx_p})\right) - \sum_{x_n\in X_{neg}}\ln\left(1 + e^{\vw^\top\vx_n}\right)\right)                           \\
                             & = \nabla\sum_{x_p\in X_{pos}} \left(\vw^\top\vx_p - \ln(1 + e^{\vw^\top\vx_p})\right) - \nabla\sum_{x_n\in X_{neg}}\ln\left(1 + e^{\vw^\top\vx_n}\right)                                   \\
                             & = \sum_{x_p\in X_{pos}} \left(\vx_p - \nabla\ln(1 + e^{\vw^\top\vx_p})\right) - \sum_{x_n\in X_{neg}}\nabla\ln\left(1 + e^{\vw^\top\vx_n}\right)                                           \\
                             & = \sum_{x_p\in X_{pos}} \left(\vx_p - \frac{e^{\vw^\top\vx_p}}{1 + e^{\vw^\top\vx_p}}\vx_p\right) - \sum_{x_n\in X_{neg}}\left(\frac{e^{\vw^\top\vx_n}}{1 + e^{\vw^\top\vx_n}}\vx_n\right) \\
                             & = \sum_{x_p\in X_{pos}} \left(1 - \frac{e^{\vw^\top\vx_p}}{1 + e^{\vw^\top\vx_p}}\right)\vx_p - \sum_{x_n\in X_{neg}}\left(\frac{e^{\vw^\top\vx_n}}{1 + e^{\vw^\top\vx_n}}\vx_n\right)     \\
                             & = \sum_{x_p\in X_{pos}} \left(1 - \sigma(\vw^\top\vx_p)\right)\vx_p - \sum_{x_n\in X_{neg}}\sigma(\vw^\top\vx_n)\vx_n.
          \end{align}
    \item For SGD, see Algorithm~\ref{alg:sgd}. For BGD, see
          Algorithm~\ref{alg:bgd}.

          \begin{algorithm}
              \caption{Stochastic Gradient Descent}
              \label{alg:sgd}
              \begin{algorithmic}[1]
                  \Procedure{Gradient Descent}{Dataset $X$, Objective function $J$, Model $p_{\vw}$, Learning rate $lr$, Number of Epochs $N$}
                  \State $\vw\leftarrow random()$ \qquad   // random weight initialization
                  \For{$i \in\{1,\ldots N\}$} \qquad   // we iterate over epochs
                  \For{$x_i\in X$} \qquad   // we iterate over the dataset
                  \State $g_i\leftarrow \nabla_{\vw}J(p_{\vw}(x_i))$
                  \State $\vw_{i+1}\leftarrow \vw_i - lr \cdot g_i$ \qquad   // update rule
                  \EndFor
                  \EndFor
                  \State{\textbf{Return} $p_{\vw}$}
                  \EndProcedure
              \end{algorithmic}
          \end{algorithm}
          \begin{algorithm}
              \caption{Batch Gradient Descent}
              \label{alg:bgd}
              \begin{algorithmic}[1]
                  \Procedure{Gradient Descent}{Dataset $X$, Objective function $J$, Model $p_{\vw}$, Learning rate $lr$, Number of Epochs $N$, Batch Size $b$}
                  \State $\vw\leftarrow random()$ \qquad   // random weight initialization
                  \For{$i \in\{1,\ldots N\}$}
                  \For{$batch = [x_1,\ldots,x_b]$ where $batch$ is sampled without replacement from $X$}
                  \State $g_i\leftarrow \nabla_{\vw}J(p_{\vw}(batch))$
                  \State $\vw_{i+1}\leftarrow \vw_i - lr \cdot g_i$ \qquad   // update rule
                  \EndFor
                  \EndFor
                  \State{\textbf{Return} $p_{\vw}$}
                  \EndProcedure
              \end{algorithmic}
          \end{algorithm}

          Note that we still rely on gradients that need to be computed every
          time we update our parameters. But we don't compute gradients manually
          when working with these models. This is generally handled by tools
          like \href{https://pytorch.org/tutorials/beginner/blitz/autograd_tutorial.html}{Autograd}.
          But it's important to understand the process underlying the training
          of models, despite it being automatic, so we can reason about model 
          performance as a function of its training process.
    \item For SGD, see Algorithm~\ref{alg:sgd_shuffled}. For BGD, see
          Algorithm~\ref{alg:bgd_shuffled}.

          \begin{algorithm}
              \caption{Stochastic Gradient Descent with Shuffling}
              \label{alg:sgd_shuffled}
              \begin{algorithmic}[1]
                  \Procedure{Gradient Descent}{Dataset $X$, Objective function $J$, Model $p_{\vw}$, Learning rate $lr$, Number of Epochs $N$}
                  \State $\vw\leftarrow random()$ \qquad   // random weight initialization
                  \For{$i \in\{1,\ldots N\}$} \qquad   // we iterate over epochs
                  \For{$x_i\in X$} \qquad   // we iterate over the dataset
                  \State $g_i\leftarrow \nabla_{\vw}J(p_{\vw}(x_i))$
                  \State $\vw_{i+1}\leftarrow \vw_i - lr \cdot g_i$ \qquad   // update rule
                  \EndFor
                  \State shuffle($X$)
                  \EndFor
                  \State{\textbf{Return} $p_{\vw}$}
                  \EndProcedure
              \end{algorithmic}
          \end{algorithm}
          \begin{algorithm}[h!]
              \caption{Batch Gradient Descent with Shuffling}
              \label{alg:bgd_shuffled}
              \begin{algorithmic}[1]
                  \Procedure{Gradient Descent}{Dataset $X$, Objective function $J$, Model $p_{\vw}$, Learning rate $lr$, Number of Epochs $N$, Batch Size $b$}
                  \State $\vw\leftarrow random()$ \qquad   // random weight initialization
                  \For{$i \in\{1,\ldots N\}$}
                  \For{$batch = [x_1,\ldots,x_b]$ where $batch$ is sampled without replacement from $X$}
                  \State $g_i\leftarrow \nabla_{\vw}J(p_{\vw}(batch))$
                  \State $\vw_{i+1}\leftarrow \vw_i - lr \cdot g_i$ \qquad   // update rule
                  \EndFor
                  \State shuffle($X$)
                  \EndFor
                  \State{\textbf{Return} $p_{\vw}$}
                  \EndProcedure
              \end{algorithmic}
          \end{algorithm}

          While GD considers the entire dataset when computing gradients, and 
          SGD treats all examples equally, BGD uses a subset of the dataset to 
          compute each gradient.
          Shuffling the dataset at the end of each epoch is a common practice to
          avoid using the same subset of examples for each BGD update, as this 
          may bias the training process in favor of some examples over others.
          BGD is by far the most method used in NLP training. It has been
          empirically shown that SGD and BGD can converger faster than GD, but
          they can also be more unstable.
\end{enumerate}

\end{document}
