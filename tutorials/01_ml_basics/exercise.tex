\documentclass[11pt,a4paper]{article}

% packages
\usepackage[utf8]{inputenc}
\usepackage{amsmath}
\usepackage[T1]{fontenc}
\usepackage{setspace}
\usepackage{enumitem}
% \usepackage{amsmath}
\usepackage{booktabs}
\usepackage{fullpage} 
\usepackage{tabularx}
\usepackage{amssymb,amstext,amsmath}
\usepackage{fancyhdr}
\usepackage{graphicx}
% \usepackage{algorithmic}
% \usepackage[ruled,vlined]{algorithm2e}
\usepackage{url}
\usepackage[bookmarks,unicode=true,pdftex,a4paper]{hyperref}
\usepackage[round]{natbib}
\usepackage{algpseudocode}
\usepackage{algorithm}
\headsep1cm

% macros
% misc
\newcommand\todo[1]{\textcolor{red}{TODO: #1}}
\newcommand\hide[1]{\textcolor{white}{#1}}

% formatting
\newcommand\bld[1]{\textbf{#1}}
\newcommand\ul[1]{\underline{#1}}
\newcommand\n[1]{\numprint{#1}}
\newcommand{\ts}{\textsuperscript}
\newcommand\red[1]{\textcolor{red}{#1}}
\newcommand\blue[1]{\textcolor{blue}{#1}}
\newcommand\link[2]{\href{#1}{\textcolor{blue}{\underline{#2}}}}

% sets
\newcommand\set[1]{\mathcal{#1}}
\newcommand\bb[1]{\mathbb{#1}}
\renewcommand\:{\colon} % for use with \sset, etc.
\newcommand{\sset}[1]{\left\{\,#1\,\right\}} % { ? }, automatic brackets
\newcommand{\ssets}[1]{\left\{#1\right\}} % {?}, automatic brackets
\newcommand{\ssetn}[1]{\{\,#1\,\}} % { ? }, normal brackets

% table formatting
% To better align bold entries in S columns (still broken)
% \usepackage{siunitx}
% \robustify\bfseries
% \newrobustcmd{\bfcell}{\bfseries}

% vector variables (taken from macros by Rainer Gemulla)
\newcommand\vect[1]{{\boldsymbol{#1}}}
\newcommand\va{\vect{a}}
\newcommand\vb{\vect{b}}
\newcommand\vc{\vect{c}}
\newcommand\vd{\vect{d}}
\newcommand\ve{\vect{e}}
\newcommand\vf{\vect{f}}
\newcommand\vg{\vect{g}}
\newcommand\vh{\vect{h}}
\newcommand\vi{\vect{i}}
\newcommand\vj{\vect{j}}
\newcommand\vk{\vect{k}}
\newcommand\vl{\vect{l}}
\newcommand\vm{\vect{m}}
\newcommand\vn{\vect{n}}
\newcommand\vo{\vect{o}}
\newcommand\vp{\vect{p}}
\newcommand\vq{\vect{q}}
\newcommand\vr{\vect{r}}
\newcommand\vs{\vect{s}}
\newcommand\vt{\vect{t}}
\newcommand\vu{\vect{u}}
\newcommand\vv{\vect{v}}
\newcommand\vw{\vect{w}}
\newcommand\vx{\vect{x}}
\newcommand\vy{\vect{y}}
\newcommand\vz{\vect{z}}
\newcommand\vzero{\vect{0}}
\newcommand\vone{\vect{1}}

\newcommand\valpha{\vect{\alpha}}
\newcommand\vbeta{\vect{\beta}}
\newcommand\veps{\vect{\epsilon}}
\newcommand\vdelta{\vect{\delta}}
\newcommand\vtheta{\vect{\theta}}
\newcommand\vsigma{\vect{\sigma}}
\newcommand\vpi{\vect{\pi}}
\newcommand\vlambda{\vect{\lambda}}

% matrix variables (taken from macros by Rainer Gemulla)
\newcommand\mA{\vect{A}}
\newcommand\mB{\vect{B}}
\newcommand\mC{\vect{C}}
\newcommand\mD{\vect{D}}
\newcommand\mE{\vect{E}}
\newcommand\mF{\vect{F}}
\newcommand\mG{\vect{G}}
\newcommand\mH{\vect{H}}
\newcommand\mI{\vect{I}}
\newcommand\mJ{\vect{J}}
\newcommand\mK{\vect{K}}
\newcommand\mL{\vect{L}}
\newcommand\mM{\vect{M}}
\newcommand\mN{\vect{N}}
\newcommand\mO{\vect{O}}
\newcommand\mP{\vect{P}}
\newcommand\mQ{\vect{Q}}
\newcommand\mR{\vect{R}}
\newcommand\mS{\vect{S}}
\newcommand\mT{\vect{T}}
\newcommand\mU{\vect{U}}
\newcommand\mV{\vect{V}}
\newcommand\mW{\vect{W}}
\newcommand\mX{\vect{X}}
\newcommand\mY{\vect{Y}}
\newcommand\mZ{\vect{Z}}
\newcommand\mzero{\vect{0}}

\newcommand{\mPi}{{\ensuremath{\vect{\Pi}}}}
\newcommand{\mSigma}{{\ensuremath{\vect{\Sigma}}}}
\newcommand{\mLambda}{{\ensuremath{\vect{\Lambda}}}}

% argmin, argmax
\DeclareMathOperator*{\argmin}{argmin} % amsmath package required
\DeclareMathOperator*{\argmax}{argmax} % amsmath package required

% matrix operations
\newcommand\xdiag{\operatorname{diag}}      
\newcommand\diag[1]{\xdiag\left(#1\right)}    % diagonal matrix


% header and footer
\lhead{Advanced Methods in Text Analytics, FSS2025}
\chead{}
\rhead{\thepage\ }
\cfoot{}
\pagestyle{fancy}

\title{Advanced Methods in Text Analytics \\ 
Exercise 1: Machine Learning Basics}
\author{Daniel Ruffinelli}
\date{FSS 2025}

\begin{document}
\maketitle

The following is some of the mathematical notation we will use throughout the
semester:

\begin{itemize}
    \item Scalars: $a,b\in\bb{R}$
    \item Vectors: $\vx,\vy\in\bb{R}^n$ (by default, vectors are all \emph{column}
          vectors, i.e.\ size is $n\times 1$)
    \item Matrices: $\mX,\mY\in\bb{R}^{m\times n}$
    \item Scalar product: $a\vx\in\bb{R}^n$
    \item Dot product: $\vx^\top\vy\in\bb{R}$ (some sources call this a scalar product)
    \item Matrix-vector product: $\mX\vx\in\bb{R}^{m\times 1}$
    \item Matrix-matrix product: $\mX^\top\mY\in\bb{R}^{n\times n}$
\end{itemize}

\section{Machine Learning Basics}

In machine learning, we typically develop models that can be trained to perform
a specific task.
Formally, a model can be represented by a function
$f_{\vtheta}\colon \bb{R}^n\mapsto \bb{R}^m$, where $\vtheta\in\bb{R}^k$ are the
parameters of the model.
For a given $\vx\in\bb{R}^n$, we can write this as follows:
$f(\vx\mid\vtheta) = \vy$.

\begin{enumerate}[label=(\alph*)]
    \item What part of the basic machine learning pipeline do the symbols in the
          formal description above denote?
          Specifically:
          \begin{enumerate}[label=(\roman*)]
              \item What does $\vx$ represent? And each of $\vx_i$?
              \item What is $\vy$ and what is it used for?
              \item What is $f(\vx\mid\vtheta)$? Explain this notation.
              \item What is a possible value for the input dimension $n$?
                    And for the output dimension $m$? What factors determine
                    these values?
          \end{enumerate}
    \item What are the common types of machine learning tasks in natural
          language processing based on the output a model produces?
          What does the output look like for each of these task types?
    \item How does \emph{``learning''} take place in machine learning? What is
          the goal of this learning process?
    \item What are the \textit{hyperparameters} of a machine learning model?
          Provide an example.
          How do we find the optimal values for a model's hyperparameters?
    \item Why do machine learning pipelines typically rely on datasets split
          into three parts: training, validation, and test sets? What is the
          role of each of these data splits?
    \item Briefly explain the concepts of \emph{underfitting} and
          \emph{overfitting}. Why do they often occur? How may we recognize them
          in practice?
\end{enumerate}

\section{Log-linear Classifiers}

Linear classifiers are those that model the classification task as a
\emph{linear combination} of input features and model parameters, where the
model parameters act as the \emph{weights} of that linear combination.
In this task, we review some log-linear classifiers commonly used in NLP.

\begin{enumerate}[label=(\alph*)]
    \item Recall that the logistic regression model is a linear classifier
          designed for binary classification. We introduced this model in the
          lectures with the following equations:
          \begin{align}
              p(y=1\mid x) & = \sigma(w x + b)     \\
              p(y=0\mid x) & = 1 - \sigma(w x + b)
          \end{align}
          where $x,y,w,b\in\bb{R}$, $\vtheta=(w,b)$ are model parameters, $x$ is
          the input and $\sigma$ is the logistic (sigmoid) function defined
          as follows:
          \begin{align}\label{eq:sigmoid}
              \sigma(z) = \frac{\exp{(z)}}{1 + \exp{(z)}}
          \end{align}
          Give a formal expression for a logistic regression model that takes as
          input a vector $\vx\in\bb{R}^n$ and that does not use a bias term.
          How many parameters does this model have?
    \item If not done already, write the same logistic regression model from the
          previous question but in vectorized form, i.e.\ as a function of its
          parameter vector $\vw$.
    \item Now add a bias term to your logistic regression model. Can we still
          write it in vectorized form?
    \item \emph{Log}-linear models are those that model classification as a
          linear combination of $\vx$ and $\vw$ with the exponential function
          applied to it.
          Give a formal expression for a \emph{general} log-linear model
          $f(y\mid\vx)$ designed for binary classification and parameterized by
          $\vw$, where $\vx,\vw\in\bb{R}^n$.
    \item Now generalize your answer from the previous question so the model is
          designed for multi-class classification. How many parameters does this
          model have?
    \item If not done already, rewrite your model from the previous question so
          it's a probabilistic model, i.e.\ for each input vector $\vx$ and
          class $c$, the models provides the probability that the input belongs 
          to that given class $c$. Can you do this in vectorized form? I.e.\ can 
          you define a function that returns a distribution over classes for a 
          given input vector $\vx$?
    \item Let's put it all together now. Given $\mX\in\bb{R}^{m\times n}$ that
          represents $m$ examples of $n$ dimensions, write a formal expression
          for a general log-linear model designed for multi-class classification
          over $C$ classes \emph{in vectorized form}. Assume no biases.
    \item Can you guess why we use \emph{log}-probabilities in log-linear models
          instead of just probabilities as in linear classifiers?
\end{enumerate}

\section{Training Log-linear Classifiers}

In this task, we are interested in training a probabilistic log-linear model,
like those from the previous task, using maximum likelihood estimation (MLE).

\begin{enumerate}[label=(\alph*)]
    \item In MLE, we aim to maximize the likelihood of a model given some data.
          Let $X=\{x_1, x_2, \ldots, x_n\}$ be a dataset of $n$ examples and
          $p$ a probabilistic log-linear model parameterized by $\vw\in\bb{R}^d$
          and designed for binary classification.
          Given the following joint distribution of the data as given by the
          model:
          \begin{align}
              p(X\mid\vw) = p(x_1, x_2, \ldots, x_n),
          \end{align}
          rewrite this expression assuming that all examples are independent and
          identically distributed (i.i.d.).
    \item For clarity, let $X_{pos}$ be the set examples from the positive class,
          and $X_{neg}$ the set of examples from the negative class.
          Rewrite your answer from the previous question as a function of
          $X_{pos}$ and $X_{neg}$.
    \item Write the corresponding likelihood function $\set{L}$ for the model
          $p$ described in your previous answer.
    \item For reasons discussed in the previous task, we often work with
          log-probabilities instead of probabilities.
          Write the log-likelihood function $\set{LL}$ for the model $p$.
          Use $\ln$ throughout.
    \item Assume $p$ is a logistic regression model and each data point in $X$
          is represented by $\vx_i\in\bb{R}^d$.
          Rewrite the log-likelihood function you wrote in the previous answer
          by including the logistic regression model explicitly, i.e.\ using the
          sigmoid function in Eq.~\ref{eq:sigmoid}.
    \item As seen in the MLE example from the lecture, we will need to compute
          the gradient to maximize our expression. Compute the gradient of the
          likelihood function for your logistic regression model w.r.t.\ model
          parameters $\vw$.
          That is, $\nabla_{\vw}\set{LL}$ where $\nabla$ is defined as:
          \begin{align}\label{eq:gradient}
              \nabla_{\vw}\set{LL} = \left[\frac{\partial\set{LL}}{\partial w_1},
                  \frac{\partial\set{LL}}{\partial w_2}, \ldots, \frac{\partial\set{LL}}{\partial w_d}\right].
          \end{align}
    \item In most cases, there are no closed-form solutions to the MLE problem.
          Instead, we use gradient-based optimization methods.
          Recall the gradient descent (GD) method described in the following
          algorithm:
          \begin{algorithm}
              \caption{Gradient Descent}
              \label{alg:gd}
              \begin{algorithmic}[1]
                  \Procedure{Gradient Descent}{Dataset $X$, Objective function $J$, Model $p_{\vw}$, Learning rate $lr$, Number of Epochs $N$}
                  \State $\vw\leftarrow random()$ \qquad   // random weight initialization
                  \For{$i \in\{1,\ldots N\}$}
                  \State $g_i\leftarrow \nabla_{\vw}J(p_{\vw}(X))$
                  \State $\vw_{i+1}\leftarrow \vw_i - lr \cdot g_i$ \qquad   // update rule
                  \EndFor
                  \State{\textbf{Return} $p_{\vw}$}
                  \EndProcedure
              \end{algorithmic}
          \end{algorithm}

          When relating this algorithm to the previous questions, $p_{\vw}$ is a
          logistic regression model, $J$ is given by the answer to question (e),
          and $g$ is given by Eq.~\ref{eq:gradient}.
          Modify Algorithm~\ref{alg:gd} so it performs (i) stochastic gradient
          descent (SGD) and (ii) mini-batch gradient descent (BGD).
          Recall that the former uses a single example for each update, and the
          latter a subset of the data for each update.
          Make sure you define each new function or component you use in the
          algorithm.
    \item Assume that the function $\text{shuffle}(X)$ randomly shuffles the
          examples in dataset $X$.
          How would you modify your BGD algorithm to include this step?
          Do you know why shuffling is important?
\end{enumerate}

\end{document}
