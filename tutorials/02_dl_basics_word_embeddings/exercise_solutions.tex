\documentclass[11pt,a4paper]{article}

% packages
% \usepackage[fleqn]{amsmath}
\usepackage{mathtools}
% \usepackage{empheq}
\usepackage[utf8]{inputenc}
% \usepackage{amsmath}
\usepackage[T1]{fontenc}
\usepackage{setspace}
\usepackage{enumitem}
\usepackage{amsmath}
\usepackage{booktabs}
\usepackage{fullpage} 
\usepackage{tabularx}
% \usepackage{amssymb,amstext,amsmath}
\usepackage{amssymb,amstext}
\usepackage{fancyhdr}
\usepackage{graphicx}
\usepackage{algorithmic}
\usepackage[ruled,vlined]{algorithm2e}
\usepackage{url}
\usepackage[bookmarks,unicode=true,pdftex,a4paper]{hyperref}
\usepackage[round]{natbib}
\usepackage[usenames,dvipsnames]{color, xcolor}
\headsep1cm

% macros
% misc
\newcommand\todo[1]{\textcolor{red}{TODO: #1}}
\newcommand\hide[1]{\textcolor{white}{#1}}

% formatting
\newcommand\bld[1]{\textbf{#1}}
\newcommand\ul[1]{\underline{#1}}
\newcommand\n[1]{\numprint{#1}}
\newcommand{\ts}{\textsuperscript}
\newcommand\red[1]{\textcolor{red}{#1}}
\newcommand\blue[1]{\textcolor{blue}{#1}}
\newcommand\link[2]{\href{#1}{\textcolor{blue}{\underline{#2}}}}

% sets
\newcommand\set[1]{\mathcal{#1}}
\newcommand\bb[1]{\mathbb{#1}}
\renewcommand\:{\colon} % for use with \sset, etc.
\newcommand{\sset}[1]{\left\{\,#1\,\right\}} % { ? }, automatic brackets
\newcommand{\ssets}[1]{\left\{#1\right\}} % {?}, automatic brackets
\newcommand{\ssetn}[1]{\{\,#1\,\}} % { ? }, normal brackets

% table formatting
% To better align bold entries in S columns (still broken)
% \usepackage{siunitx}
% \robustify\bfseries
% \newrobustcmd{\bfcell}{\bfseries}

% vector variables (taken from macros by Rainer Gemulla)
\newcommand\vect[1]{{\boldsymbol{#1}}}
\newcommand\va{\vect{a}}
\newcommand\vb{\vect{b}}
\newcommand\vc{\vect{c}}
\newcommand\vd{\vect{d}}
\newcommand\ve{\vect{e}}
\newcommand\vf{\vect{f}}
\newcommand\vg{\vect{g}}
\newcommand\vh{\vect{h}}
\newcommand\vi{\vect{i}}
\newcommand\vj{\vect{j}}
\newcommand\vk{\vect{k}}
\newcommand\vl{\vect{l}}
\newcommand\vm{\vect{m}}
\newcommand\vn{\vect{n}}
\newcommand\vo{\vect{o}}
\newcommand\vp{\vect{p}}
\newcommand\vq{\vect{q}}
\newcommand\vr{\vect{r}}
\newcommand\vs{\vect{s}}
\newcommand\vt{\vect{t}}
\newcommand\vu{\vect{u}}
\newcommand\vv{\vect{v}}
\newcommand\vw{\vect{w}}
\newcommand\vx{\vect{x}}
\newcommand\vy{\vect{y}}
\newcommand\vz{\vect{z}}
\newcommand\vzero{\vect{0}}
\newcommand\vone{\vect{1}}

\newcommand\valpha{\vect{\alpha}}
\newcommand\vbeta{\vect{\beta}}
\newcommand\veps{\vect{\epsilon}}
\newcommand\vdelta{\vect{\delta}}
\newcommand\vtheta{\vect{\theta}}
\newcommand\vsigma{\vect{\sigma}}
\newcommand\vpi{\vect{\pi}}
\newcommand\vlambda{\vect{\lambda}}

% matrix variables (taken from macros by Rainer Gemulla)
\newcommand\mA{\vect{A}}
\newcommand\mB{\vect{B}}
\newcommand\mC{\vect{C}}
\newcommand\mD{\vect{D}}
\newcommand\mE{\vect{E}}
\newcommand\mF{\vect{F}}
\newcommand\mG{\vect{G}}
\newcommand\mH{\vect{H}}
\newcommand\mI{\vect{I}}
\newcommand\mJ{\vect{J}}
\newcommand\mK{\vect{K}}
\newcommand\mL{\vect{L}}
\newcommand\mM{\vect{M}}
\newcommand\mN{\vect{N}}
\newcommand\mO{\vect{O}}
\newcommand\mP{\vect{P}}
\newcommand\mQ{\vect{Q}}
\newcommand\mR{\vect{R}}
\newcommand\mS{\vect{S}}
\newcommand\mT{\vect{T}}
\newcommand\mU{\vect{U}}
\newcommand\mV{\vect{V}}
\newcommand\mW{\vect{W}}
\newcommand\mX{\vect{X}}
\newcommand\mY{\vect{Y}}
\newcommand\mZ{\vect{Z}}
\newcommand\mzero{\vect{0}}

\newcommand{\mPi}{{\ensuremath{\vect{\Pi}}}}
\newcommand{\mSigma}{{\ensuremath{\vect{\Sigma}}}}
\newcommand{\mLambda}{{\ensuremath{\vect{\Lambda}}}}

% argmin, argmax
\DeclareMathOperator*{\argmin}{argmin} % amsmath package required
\DeclareMathOperator*{\argmax}{argmax} % amsmath package required

% matrix operations
\newcommand\xdiag{\operatorname{diag}}      
\newcommand\diag[1]{\xdiag\left(#1\right)}    % diagonal matrix


% header and footer
\lhead{Advanced Methods in Text Analytics, FSS2025}
\chead{}
\rhead{\thepage\ }
\cfoot{}
\pagestyle{fancy}

\title{Advanced Methods in Text Analytics \\ 
Exercise 2: Deep Learning Basics and Word Embeddings \\ 
\textbf{Solutions}}
\author{Daniel Ruffinelli}
\date{FSS 2025}

\begin{document}
\maketitle

\section{Fully-Connected Neural Networks}

\begin{enumerate}[label=(\alph*)]
    \item As many as we need, so $n$ input units, $m$ output units, where $n$ is
          not necessarily equal to $m$.
    \item Nodes hold values that are computed based on other nodes and some
          edges, whereas edges hold parameters of the function represented by
          the network.
    \item Generally, the input of the operation computed by each node is the
          nodes from the previous layer
          \emph{that are connected to it by an edge.}
          Let's call these \emph{input} nodes.
          The general operation is a linear combination between the value in the
          input nodes and the corresponding weight vector $\vw_i$ made up of the
          values on the edges connecting the input vectors.
          Then, an \emph{activation} function $f$ is applied to the output of
          this linear combination.
          $f$ may or may not be a linear function.
          Concretely, let $h_i^{(k)}$ be the (output) value of the $i$-th unit
          in the $k$-th hidden layer in the network.
          We have:
          \begin{equation*}
              h^{(k)}_i = f\left(\sum_{j=1}^{L^{k-1}}{h_j^{(k-1)} w_{i,j}^{(k)}}\right),
          \end{equation*}
          where $L^{k-1}$ is the size of the $(k-1)$-th layer and
          $w_{i,j}^{(k)}$ is the $j$-th component of $\vw_i$.
          Note that we can write this more compactly in vectorized form:
          \begin{equation*}
              h^{(k)}_i = f(\vh_{(k-1)}^\top\vw_k),
          \end{equation*}
          where both $\vh,\vw$ are column vectors, $\vh_{(k-1)}$ is the vector
          with values in the $(k-1)$-th layer, and $f$ is applied element-wise.
    \item The activation function. For example, ReLU, \emph{tanh}, the
          logistic function.
    \item $\mH_1=f(\mX\mW^T)\in\bb{R}^{N\times p}$ where
          $\mW\in\bb{R}^{p\times d}$ is the matrix constructed by stacking the
          weight vectors of each hidden unit in the first layer as rows, and $f$
          is an activation function that is applied element-wise.
\end{enumerate}

\section{Backpropagation}
\begin{enumerate}[label=(\alph*)]
    \item $\frac{\partial}{\partial\vw}L$ where $L$ is our loss function and
          $\vw$ the model parameters we want to adjust during learning.
    \item We use standard derivation rules followed by some rewriting to obtain
          the desired result.
          \begin{align*}
              \sigma(x)' & = \frac{d\sigma(x)}{d(x)}                                      \\
                         & = \frac{d(\frac{1}{1+e^{-x}})}{dx}                             \\
                         & = -\frac{1}{(1+e^{-x})^2}\cdot \frac{d(1+e^{-x})}{dx}          \\
                         & = -\frac{1}{(1+e^{-x})^2}\cdot (e^{-x}) \cdot \frac{d(-x)}{dx} \\
                         & = \frac{1}{(1+e^{-x})^2}\cdot (e^{-x})                         \\
                         & = \frac{e^{-x}}{(1+e^{-x})^2}                                  \\
                         & = \frac{1+e^{-x}}{(1+e^{-x})^2} - \frac{1}{(1+e^{-x})^2}       \\
                         & = \frac{1}{(1+e^{-x})} - \frac{1}{(1+e^{-x})^2}                \\
                         & = \sigma(x)-\sigma(x)^2                                        \\
                         & = \sigma(x)\cdot (1-\sigma(x))
          \end{align*}
    \item In our setting, the network produces two numbers $y_1,y_2\in [0,1]$,
          each corresponding to the probability of one class.
          The given labels match this setting, since they are given by a pair of
          binary numbers. E.g.\ label $[0,1]$ means the example corresponds to
          the 2nd class, not the first.
          Using squared error loss, we want the model to match its outputs to
          the given labels during training, so for label $[0,1]$, the model
          should match $y_1$ to 0 and $y_2$ to 1.
          This is achieved with the loss we were given:
          \begin{align*}
              L & = -\frac{1}{2}\frac{1}{N}\sum_{e=1}^{N} \sum_{l=1}^{2} (t_{e,l}-y_{e,l})^2.
          \end{align*}
          where $N$ is the number of training examples (which we will keep as
          $N$ for clarity during derivations).
          Also, we add the $1/2$ factor to simplify gradient computations.
          This is commonly done because is does not affect optimization as the
          results are proportional to the exact results. \\

          \begin{enumerate}[label=(\roman*)]
              \item Let $h_j$ be the value of hidden unit $j$, and
                    $w_{i,j}^{(k)}$ the weight from input feature $i$ in layer
                    $k-1$ to neuron $j$ in layer $k$.
                    Our network architecture tells us that
                    \begin{align*}
                        y_1 & = \sigma(w^{(2)}_{1,1}\cdot h_1 + w^{(2)}_{2,1}\cdot h_2)  \\
                        y_2 & = \sigma(w^{(2)}_{1,2}\cdot h_1 + w^{(2)}_{2,2}\cdot h_2).
                    \end{align*}
                    We derive the required gradient w.r.t. $w_{1,2}^{(2)}$ for
                    the update rule as follows:
                    \begin{align*}
                        \frac{\partial L}{\partial w_{1,2}^{(2)}} & = - \frac{1}{2}\frac{1}{N}\sum_{e=1}^N \sum_{l=1}^2 \frac{\partial}{\partial w_{1,2}^{(2)}} \left(t_{e,l}-y_{e,l}\right)^2                                                                                                                                                     \\
                                                                  & = - \frac{1}{N} \sum_{e=1}^{N} \left( (-1)(t_{e,1}-y_{e,1})\cdot \frac{\partial y_{e,1}}{\partial w_{1,2}^{(2)}} + (t_{e,2}-y_{e,2}) \cdot \frac{\partial y_{e,2}}{\partial w_{1,2}^{(2)}}  \right) &  & \text{($\frac{\partial y_{e,1}}{\partial w_{1,2}^{(2)}}$ evaluates to 0)} \\
                                                                  & = - \frac{1}{N} \sum_{e=1}^{N} (t_{e,2}-y_{e,2})\cdot\frac{\partial y_{e,2}}{\partial w_{1,2}^{(2)}}                                                                                            &  & \text{(use gradient from (b))}                                            \\
                                                                  & = - \frac{1}{N} \sum_{e=1}^{N} (t_{e,2}-y_{e,2})\cdot y_{e,2}\cdot\left(1-y_{e,2}\right) \cdot \frac{\partial (w^{(2)}_{1,2}\cdot h_1 + w^{(2)}_{2,2}\cdot h_2)}{\partial w_{1,2}^{(2)}}        &  & \text{(chain rule)}                                                       \\
                                                                  & = - \frac{1}{N} \sum_{e=1}^{N} (t_{e,2}-y_{e,2})\cdot y_{e,2}\cdot\left(1-y_{e,2}\right) \cdot h_1                                                                                                                                                                             \\
                                                                  & = - \frac{1}{N} \sum_{e=1}^{N} \delta_{e,2} \cdot h_1,
                    \end{align*}
                    where $\delta_{e,2} = (t_{e,2}-y_{e,2})\cdot y_{e,2}\cdot\left(1-y_{e,2}\right).$
                    Similarly, our network architecture tells us that:
                    \begin{align*}
                        h_1 & = \sigma(w^{(1)}_{1,1}\cdot x_1 + w^{(1)}_{2,1}\cdot x_2 + w^{(1)}_{3,1}\cdot x_3)  \\
                        h_2 & = \sigma(w^{(1)}_{1,2}\cdot x_1 + w^{(1)}_{2,2}\cdot x_2 + w^{(1)}_{3,2}\cdot x_3).
                    \end{align*}
                    Then, we obtain our gradient w.r.t. $w_{2,2}^{(1)}$ as follows:
                    \begin{align*}
                        \frac{\partial L}{\partial w_{2,2}^{(1)}} & = - \frac{1}{2N} \sum_{e=1}^N \sum_{l=2}^2 \frac{\partial}{\partial w_{2,2}^{(1)}} \left(t_{e,l}-y_{e,l}\right)^2                                                                                                                                                                                                                      \\
                                                                  & = - \frac{1}{N} \sum_{e=1}^{N} \left( (t_{e,1}-y_{e,1})\cdot \frac{\partial y_{e,1}}{\partial w_{2,2}^{(1)}} + (t_{e,2}-y_{e,2}) \cdot \frac{\partial y_{e,2}}{\partial w_{2,2}^{(1)}}  \right)                                                                                                                                        \\
                                                                  & = - \frac{1}{N} \sum_{e=1}^{N} \left( \delta_{e,1}\cdot\frac{\partial(w^{(2)}_{1,1}\cdot h_1 + w^{(2)}_{2,1}\cdot h_2)}{\partial w_{2,2}^{(1)}} + \delta_{e,2}\cdot\frac{\partial(w^{(2)}_{1,2}\cdot h_1 + w^{(2)}_{2,2}\cdot h_2)}{\partial w_{2,2}^{(1)}}  \right)                                                                   \\
                                                                  & = - \frac{1}{N} \sum_{e=1}^{N} \left( \delta_{e,1}\cdot\frac{\partial w^{(2)}_{2,1}\cdot h_2}{\partial w_{2,2}^{(1)}} + \delta_{e,2}\cdot\frac{\partial w^{(2)}_{2,2}\cdot h_2}{\partial w_{2,2}^{(1)}}  \right)                                                     &  & \text{($\frac{\partial h_{1}}{\partial w_{2,2}^{(1)}} = 0$)} \\
                                                                  & = - \frac{1}{N} \sum_{e=1}^{N} \left( \delta_{e,1}\cdot w^{(2)}_{2,1} \cdot h_2\cdot(1 - h_2)\cdot x_2 + \delta_{e,2}\cdot w^{(2)}_{2,2}\cdot h_2\cdot(1 - h_2)\cdot x_2  \right)                                                                                                                                                      \\
                                                                  & = - \frac{1}{N} \sum_{e=1}^{N} \cdot h_2\cdot(1 - h_2)\cdot x_2 \sum_{l=1}^{2} \delta_{e,l}\cdot w^{(2)}_{2,l}.                                                                                                                                                                                                                         \\
                    \end{align*}
                    We can now write our update rules for each of the weights:
                    \begin{align*}
                        w_{1,2}^{(2)} & \coloneqq w_{1,2}^{(2)} + \frac{1}{N} \sum_{e=1}^{N} (t_{e,2}-y_{e,2})\cdot y_{e,2}\cdot\left(1-y_{e,2}\right) \cdot h_1             \\
                        w_{2,2}^{(1)} & \coloneqq w_{2,2}^{(1)} + \frac{1}{N} \sum_{e=1}^{N} \cdot h_2\cdot(1 - h_2)\cdot x_2 \sum_{l=1}^{2} \delta_{e,l}\cdot w^{(2)}_{2,l}
                    \end{align*}
              \item We compute our forward pass with $w^{(i)}_{j,k} = 1$ for all
                    possible values of $i,j,k$ and the two given training 
                    examples: \\
                    \vspace{10pt} \\
                    $\vx = \begin{pmatrix}
                            -1 \\
                            0  \\
                            2
                        \end{pmatrix}, \vh = \begin{pmatrix}
                            0.73 \\
                            0.73
                        \end{pmatrix}, \vy = \begin{pmatrix}
                            0.81 \\
                            0.81
                        \end{pmatrix}$ \qquad {\color{red} \text{Example 1}}\\
                    \vspace{10pt} \\
                    $\vx = \begin{pmatrix}
                            2 \\
                            1 \\
                            1
                        \end{pmatrix}, \vh = \begin{pmatrix}
                            0.98 \\
                            0.98
                        \end{pmatrix}, \vy = \begin{pmatrix}
                            0.88 \\
                            0.88
                        \end{pmatrix}$ \qquad {\color{blue} \text{Example 2} }\\
                    \vspace{10pt} \\
                    And finally, we compute our updates:
                    \begin{align*}
                        w_{1,2}^{(2)} & \coloneqq 1+ \frac{1}{2} ({\color{red} (1-0.81)\cdot 0.81 \cdot (1-0.81) \cdot 0.73} + { \color{blue}(0-0.88) \cdot 0.88 \cdot (1-0.88) \cdot 0.98}) \\
                                      & \coloneqq 0.965                                                                                                                                      \\
                        w_{2,2}^{(1)} & \coloneqq 1+ \frac{1}{2} ({\color{red}0} + {\color{blue}1\cdot (1-0.98)\cdot 0.98 \cdot ((1-0.88)\cdot 0.88 \cdot (1-0.98) \cdot 1}                  \\
                                      & \qquad\qquad\qquad\qquad\qquad\qquad\qquad\quad{\color{blue}+(0-0.88)\cdot0.88\cdot (1-0.98)\cdot 1)})                                               \\
                                      & \coloneqq 0.99987.
                    \end{align*}
          \end{enumerate}
\end{enumerate}


\section{Skip-Gram Basics}

\begin{enumerate}[label=(\alph*)]
    \item It's multi-class classification where the number of classes is the
          number of words in a given vocabulary $V$.
          We can describe the task with the expression $p(c_j|w_i)$.
          In such a setting, it's common to obtain probabilities using the
          softmax function.
          Formally, let $l_{w_i,c_j}$ be the logit score produced by a
          classifier trained for this task given center word $w_i$ and context
          word $c_j$.
          We then have:
          \begin{align*}
              p(c_j|w_i) = \frac{\text{exp}{(l_{w_i,c_j}})}{\sum_{k\in V}\text{exp}(l_{w_i,c_k})},
          \end{align*}
          where $w_i$ is a word in position $i$ and $c_j$ is a word in a given
          context of size L, e.g.\ for L = 2, we have
          $j\in\{i-2, i-1,i+1,i+2\}$.
    \item The sliding window moves through the sentence to create the following
          context windows:
          %   \begin{enumerate}[label=(\roman*)]
          \begin{enumerate}[label=(\arabic*)]
              \item \colorbox{green!30}{The} \colorbox{yellow!30}{quick}  \colorbox{yellow!30}{brown} fox jumped over the lazy dog.
              \item \colorbox{yellow!30}{The} \colorbox{green!30}{quick} \colorbox{yellow!30}{brown} \colorbox{yellow!30}{fox} jumped over the lazy dog.
              \item \colorbox{yellow!30}{The} \colorbox{yellow!30}{quick} \colorbox{green!30}{brown} \colorbox{yellow!30}{fox} \colorbox{yellow!30}{jumped} over the lazy dog.
              \item The \colorbox{yellow!30}{quick} \colorbox{yellow!30}{brown} \colorbox{green!30}{fox} \colorbox{yellow!30}{jumped} \colorbox{yellow!30}{over} the lazy dog.
              \item The quick\colorbox{yellow!30}{brown} \colorbox{yellow!30}{fox} \colorbox{green!30}{jumped} \colorbox{yellow!30}{over} \colorbox{yellow!30}{the} lazy dog.
              \item The quick brown \colorbox{yellow!30}{fox} \colorbox{yellow!30}{jumped} \colorbox{green!30}{over} \colorbox{yellow!30}{the} \colorbox{yellow!30}{lazy} dog.
              \item The quick brown fox \colorbox{yellow!30}{jumped} \colorbox{yellow!30}{over} \colorbox{green!30}{the} \colorbox{yellow!30}{lazy} \colorbox{yellow!30}{dog}.
              \item The quick brown fox jumped \colorbox{yellow!30}{over} \colorbox{yellow!30}{the} \colorbox{green!30}{lazy} \colorbox{yellow!30}{dog}.
              \item The quick brown fox jumped over \colorbox{yellow!30}{the} \colorbox{yellow!30}{lazy} \colorbox{green!30}{dog}.
          \end{enumerate}
          Here, green marks the center words and yellow marks the context words.
          Table~\ref{tab:skipgram_examples} lists all examples generated for the
          original skip-gram task using a context window of size 2 and the
          given sentence.
\end{enumerate}

\begin{table}[t]
    \centering
    \begin{tabular}{ccc}
        \toprule
        Context Window & Source Word & Target Word \\
        \midrule
        1              & The         & quick       \\
        1              & The         & brown       \\
        2              & quick       & The         \\
        2              & quick       & brown       \\
        2              & quick       & fox         \\
        3              & brown       & The         \\
        3              & brown       & quick       \\
        3              & brown       & fox         \\
        3              & brown       & jumped      \\
        4              & fox         & quick       \\
        4              & fox         & brown       \\
        4              & fox         & jumped      \\
        4              & fox         & over        \\
        5              & jumped      & brown       \\
        5              & jumped      & fox         \\
        5              & jumped      & over        \\
        5              & jumped      & the         \\
        6              & over        & fox         \\
        6              & over        & jumped      \\
        6              & over        & the         \\
        6              & over        & lazy        \\
        7              & the         & jumped      \\
        7              & the         & over        \\
        7              & the         & lazy        \\
        7              & the         & dog         \\
        8              & lazy        & over        \\
        8              & lazy        & the         \\
        8              & lazy        & dog         \\
        9              & dog         & the         \\
        9              & dog         & lazy        \\
        \bottomrule
    \end{tabular}
    \caption{
        Positive examples for the original skip-gram task and a context
        window of size 2.
    }
    \label{tab:skipgram_examples}
\end{table}

\begin{enumerate}[label=(\alph*)]
    \setcounter{enumi}{2}
    \item The skip-gram model learns two word vectors for each word in our
          vocabulary: one for when words are center (or source) words, and one
          for when words are context (or target) words.
          Let $\mW$ be the matrix constructed by stacking vectors of center
          representations as rows, and $\mW'$ the matrix constructed in the same
          way using context word representations.
          Then $\vtheta=\{\mW, \mW'\}$.
          The size of these matrices, and thus the number of parameters in our
          skip-gram model, for our given vocabulary and vector size is:
          \begin{align*}
               & \mW\in\mathbb{R}^{101425\times 300}   \\
               & \mW'\in\mathbb{R}^{101425\times 300}.
          \end{align*}
    \item We obtain vectors for a given word with a simple lookup function,
          i.e.\ a 1-to-1 mapping between words in our vocabulary and rows of
          $\mW$ and $\mW'$.
          To obtain a single (final) representation of a given word, we need to
          decide how to use the corresponding parameters in $\mW$ and $\mW'$.
          For example, we may discard the representations in $\mW'$ and simply
          use $\vw_{Frodo} = \mW_{79}$, i.e.\ the 79th row of matrix $\mW$.
          Or we may choose to concatenate the corresponding representations from
          $\mW$ and $\mW'$, i.e. $\vw_{Frodo} = \mW_{79}\oplus\mW'_{79}$.
          Instead of concatenation, one could also use other operations between
          vectors, such as computing the sum or average between the two.
          Concatenation is a safe operation in that it does not change the
          learned features, at the expense of doubling the dimension size of
          final vectors.
\end{enumerate}

\section{Skip-Gram Training}

\begin{enumerate}[label=(\alph*)]
    \item The likelihood of the entire text corpus $T$ with a context window of
          size $L$ is given by:
          \begin{align*}
              L(\vtheta|T) = \prod_{t=1}^{|T|}\prod_{-L\leq j\leq L, j\neq 0} p(w_{t+j}|w_t,\vtheta).
          \end{align*}
          Note that here we assume all words are \emph{conditionally}
          independent of each other \emph{given} parameters $\vtheta$, i.e.\ the
          embeddings.
          In other words, we assume the embeddings encode the dependencies
          between words.
          The corresponding negative log-likelihood function using empirical
          risk minimization is given by:
          \begin{align*}
              \text{NLL}(\vtheta) = - \frac{1}{|T|} \sum_{t=1}^{|T|} \sum_{-L\leq j\leq L, j\neq 0} \log p(w_{t+j}|w_t,\vtheta).
          \end{align*}
          During training, we find the values of $\vtheta$ that minimize this
          expression using gradient-based optimizers.
    \item We first need to obtain the vector for word \textit{Frodo} when used
          as center (source) word.
          We can represent the necessary lookup function with a matrix-vector
          multiplication between our embedding matrix and the one-hot encoding
          vector that represents our word.
          Since \textit{Frodo} is the 2nd word in $V$, we get the center word
          representation for \textit{Frodo} as follows:
          \begin{align*}
              \vw_{Frodo}^T & = [0, 1, 0, 0, 0, 0] \cdot \mW \\
                            & = [0.91, 0.74, 0.27].
          \end{align*}
          Given that our model computes logits using dot products, we need the
          dot products of $\vw_{Frodo}$ with all other context vectors $\vw'$,
          i.e.\ each row in $\mW'$.
          We obtain this with the following operation:
          \begin{align*}
              \vl = \vw_{Frodo}^T\cdot \mW'^T & = [0.9111, 0.5924, 0.9265, 1.2742, 0.9519, 0.7397].
          \end{align*}
          We then compute the log loss (i.e.\ cross-entropy) as follows:
          \begin{align*}
              L_{CE}(\textit{ring}|\textit{Frodo}) & = - \ln p(\textit{ring}|\textit{Frodo})                                                            &  & \text{(log loss)}        \\
                                                   & = - \ln \frac{\text{exp}(\vw_{ring}^T\vw_{Frodo})}{\sum_{v\in V}\text{exp}(\vw_{v}^T\vw_{Frodo})} &  & \text{(softmax)}         \\
                                                   & = - \vw_{ring}^T\vw_{Frodo} + \ln \sum_{v\in V} \text{exp}(\vw_{v}^T\vw_{Frodo})                  &  & \text{(log of division)} \\
                                                   & = - 0.9265 + \ln(\text{exp}(0.9111) + \ldots + \text{exp}(0.7397))                                                               \\
                                                   & = - 0.9265 + \ln(2.49 + 1.81 + 3.58 + 2.59 + 2.10)                                                                               \\
                                                   & = - 0.9265 + 2.53                                                                                                                \\
                                                   & = 1.60.
          \end{align*}
          Note that we get the desired behavior of a loss function.
          Namely, the model is penalized (i.e.\ the loss increases) when it
          assigns a high probability to negative examples.
          Conversely, the model is rewarded (i.e. the loss decreases) when
          it assigns a high probability to the positive example.
          Given that these probabilities are determined by dot products, we
          force the model to increase the similarity between vectors of words
          only when they are found in the same context, i.e. we follow the
          distributional hypothesis.
    \item For a given positive example $(c,w)$ where $c$ is target word and $w$
          is source word, we update the row of $\mW$ that corresponds to $w$ and
          all context vectors for all words in $V$ except source word $w$, i.e.\
          every row but one in $\mW'$.
          This is because we use all our vocabulary in the denominator of the
          softmax function to compute the probability of our given positive
          example $(c,w)$.
          Consequently, the cost of computing the probability of a single
          example, and thus the number of relevant parameters during
          backpropagation, is proportional to the size of $V$, which can be very
          large.
          This is why the authors used the binary clasification task with
          negative sampling discussed in the lecture, to minimize training
          costs.
          It can be shown that the binary classification task with negative
          sampling is an approximation to this approach.
\end{enumerate}

\section{Word Embeddings as Matrix Factorization}

\begin{enumerate}[label=(\alph*)]
    \item We can explicitly express the sampling of negatives in the objective
          as follows:
          \begin{align}\label{eq:basic_objective_with_expectation}
              L(w,c) = \log \sigma(\vw_w^t\vw_c) + k\cdot \bb{E}_{i\sim P}[\log(-\vw_w^t\vw_i)].
          \end{align}
          In other words, each negative is weighted by its probability of being
          sampled.
          We can write this more explicitly in the following way:
          \begin{align}\label{eq:basic_objective_with_relative_freq}
              L(w,c) = \log \sigma(\vw_w^t\vw_c) + k\cdot \sum_{i\in V} \frac{\#(i)}{|V|}\log(-\vw_w^t\vw_i).
          \end{align}
    \item We can do this by applying
          Equation~\ref{eq:basic_objective_with_relative_freq} over the entire
          training set as follows:
          \begin{align}\label{eq:general_objective_with_relative_freq}
              L = \sum_{w\in V} \sum_{c\in V} \#(w,c)\left(\log \sigma(\vw_w^t\vw_c) + k\cdot \sum_{i\in V} \frac{\#(i)}{|V|}\log(-\vw_w^t\vw_i)\right),
          \end{align}
          where $\#(w,c)$ is the number of times tuple $(w,c)$ appears in the
          training set.
    \item In our objective, each training example $(w,c)$ is represented by
          $\vw_w^T\vw_c$.
          Using the fact that the gradient is a linear operator, and that
          $\sum_{w'\in V} \sum_{c'\in V} \#(w',c') = \sum_{w'\in V}\#(w')$,
          we first write the gradient we need to compute as follows:
          \begin{align*}
              \frac{\partial L}{\partial \vw_w^T\vw_c} & = \sum_{w'\in V} \sum_{c'\in V} \#(w',c')\left(\frac{\partial \log \sigma(\vw_{w'}^t\vw_{c'})}{\partial \vw_w^T\vw_c} + k\cdot \sum_{i\in V} \frac{\#(i)}{|V|}\frac{\partial \log\sigma(-\vw_{w'}^t\vw_i)}{\partial \vw_w^T\vw_c}\right)              \\
                                                       & = \sum_{w'\in V} \sum_{c'\in V} \#(w',c')\frac{\partial \log \sigma(\vw_{w'}^t\vw_{c'})}{\partial \vw_w^T\vw_c} + k\cdot\sum_{w'\in V} \#(w')\cdot\sum_{i\in V} \frac{\#(i)}{|V|}\frac{\partial \log\sigma(-\vw_{w'}^t\vw_i)}{\partial \vw_w^T\vw_c}.
          \end{align*}
          We then compute this gradient, using the fact that
          $\sigma(-x) = 1 - \sigma(x)$ and starting with the given hint:
          \begin{align*}
              \frac{\partial L}{\partial \vw_w^T\vw_c} & =  \#(w,c)\frac{\partial \log \sigma(\vw_{w}^t\vw_{c})}{\partial \vw_w^T\vw_c} + k\cdot\#(w)\cdot\frac{\#(c)}{|V|}\frac{1}{\sigma(-\vw_{w}^t\vw_c)}\frac{\partial (1 - \sigma(\vw_{w}^t\vw_c))}{\partial \vw_w^T\vw_c}             \\
                                                       & =  \#(w,c)\frac{\sigma(\vw_{w}^t\vw_{c})(1 - \sigma(\vw_{w}^t\vw_{c}))}{\sigma{(\vw_{w}^t\vw_{c})}}  - k\cdot\#(w)\cdot\frac{\#(c)}{|V|}\frac{\sigma(\vw_{w}^t\vw_{c})(1 - \sigma(\vw_{w}^t\vw_{c}))}{\sigma{(-\vw_{w}^t\vw_{c})}} \\
                                                       & =  \#(w,c)\sigma(-\vw_{w}^t\vw_{c}) - k\cdot\#(w)\cdot\frac{\#(c)}{|V|}\sigma(\vw_{w}^t\vw_{c}).
          \end{align*}
          Now, we set this gradient to be equal to zero and solve for
          $\vw_{w}^t\vw_{c}$, which we set to $x = \vw_{w}^t\vw_{c}$ for
          simplicity:
          \begin{align*}
              \#(w,c)\sigma(-x)                 & = k\cdot\#(w)\cdot\frac{\#(c)}{|V|}\sigma(x)                                                                  \\
              \frac{\sigma(x)}{1 - \sigma(x)}   & = \frac{1}{k}\frac{\#(w,c)|V|}{\#(w)\#(c)}                                                                    \\
              \frac{1+e^{-x}}{e^{-x}(1+e^{-x})} & = \frac{1}{k}\frac{\#(w,c)|V|}{\#(w)\#(c)}                  &  & (\text{definition of }\sigma)                \\
              e^{x}                             & = \frac{1}{k}\frac{\#(w,c)|V|}{\#(w)\#(c)}                                                                    \\
              x                                 & = \ln\left(\frac{1}{k}\frac{\#(w,c)|V|}{\#(w)\#(c)}\right) &  & (\text{applied logs})                        \\
              \vw_{w}^t\vw_{c}                  & = \ln\left(\frac{\#(w,c)|V|}{\#(w)\#(c)}\right) - \ln k   &  & (\text{def. of } x, \text{log of division}). \\
          \end{align*}
          Recall that $k$ is the number of negative examples generated per
          positive example in the training set (a hyperparameter).
          If we set $k = 1$, we have
          \begin{align*}
              \vw_{w}^t\vw_{c} & = \log\left(\frac{\#(w,c)|V|}{\#(w)\#(c)}\right). \\
          \end{align*}
          This expression is known as the pointwise mutual information (PMI)
          between $w$ and $c$, a commonly used concept in NLP that was already
          covered in the basic Text Analytics course.
          This concept is based on information-theoretic principles and is
          defined as follows:
          \begin{align*}
              \operatorname{PMI}(w,c) = \log\frac{p(w,c)}{p(w)p(c)}.
          \end{align*}
          In other words, PMI measures the log of the ratio between the joint
          probability of $w$ and $c$, i.e.\ how often they occur together, and
          their marginal probabilities, i.e.\ how often they occur
          independently.
          This can be estimated by counting relative frequencies in a given
          corpus like so:
          \begin{align*}
              \operatorname{PMI}(w,c) = \log\frac{\#(w,c)}{\#(w)\#(c)}\propto\log\frac{\#(w,c)|V|}{\#(w)\#(c)}.
          \end{align*}
          Intuitively, PMI tells us about how strongly (or weakly) related two
          words are by counting how often they are seen together in a corpus and
          contrasting this with how often they are seen independently.
          The derivation in this task shows that the optimal solution of the
          skip-gram approach as expressed by
          Equation~\ref{eq:general_objective_with_relative_freq} is the PMI
          matrix of our training corpus.
          We can thus relate skip-gram to count-based sparse word
          representations, because this allow us to interpret it as a form of
          learning a matrix factorization of the PMI matrix of our training
          corpus.
          This is similar to how latent semantic analysis (LSA) derives
          representations from the singular value decomposition of a
          co-occurrence matrix, as also seen in the basic Text Analytics course.
          Finally, note that if we set $k > 1$ as commonly done, we get a PMI
          matrix shifted by $\log k$.
          For more details,
          see~\href{https://papers.nips.cc/paper_files/paper/2014/file/feab05aa91085b7a8012516bc3533958-Paper.pdf}{Levy and Goldberg}.
\end{enumerate}

\end{document}
