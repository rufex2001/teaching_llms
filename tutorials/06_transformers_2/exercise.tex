\documentclass[11pt,a4paper]{article}

% packages
\usepackage[utf8]{inputenc}
\usepackage{amsmath}
\usepackage[T1]{fontenc}
\usepackage{setspace}
\usepackage{enumitem}
\usepackage{booktabs}
\usepackage{fullpage} 
\usepackage{tabularx}
\usepackage{amssymb,amstext,amsmath}
\usepackage{fancyhdr}
\usepackage{graphicx}
\usepackage{algorithmic}
\usepackage[ruled,vlined]{algorithm2e}
\usepackage{url}
\usepackage[colorlinks=true,bookmarks,unicode=true,pdftex,a4paper]{hyperref}
\usepackage[round]{natbib}
\usepackage[usenames,dvipsnames]{color, xcolor}
\headsep1cm

% macros
% misc
\newcommand\todo[1]{\textcolor{red}{TODO: #1}}
\newcommand\hide[1]{\textcolor{white}{#1}}

% formatting
\newcommand\bld[1]{\textbf{#1}}
\newcommand\ul[1]{\underline{#1}}
\newcommand\n[1]{\numprint{#1}}
\newcommand{\ts}{\textsuperscript}
\newcommand\red[1]{\textcolor{red}{#1}}
\newcommand\blue[1]{\textcolor{blue}{#1}}
\newcommand\link[2]{\href{#1}{\textcolor{blue}{\underline{#2}}}}

% sets
\newcommand\set[1]{\mathcal{#1}}
\newcommand\bb[1]{\mathbb{#1}}
\renewcommand\:{\colon} % for use with \sset, etc.
\newcommand{\sset}[1]{\left\{\,#1\,\right\}} % { ? }, automatic brackets
\newcommand{\ssets}[1]{\left\{#1\right\}} % {?}, automatic brackets
\newcommand{\ssetn}[1]{\{\,#1\,\}} % { ? }, normal brackets

% table formatting
% To better align bold entries in S columns (still broken)
% \usepackage{siunitx}
% \robustify\bfseries
% \newrobustcmd{\bfcell}{\bfseries}

% vector variables (taken from macros by Rainer Gemulla)
\newcommand\vect[1]{{\boldsymbol{#1}}}
\newcommand\va{\vect{a}}
\newcommand\vb{\vect{b}}
\newcommand\vc{\vect{c}}
\newcommand\vd{\vect{d}}
\newcommand\ve{\vect{e}}
\newcommand\vf{\vect{f}}
\newcommand\vg{\vect{g}}
\newcommand\vh{\vect{h}}
\newcommand\vi{\vect{i}}
\newcommand\vj{\vect{j}}
\newcommand\vk{\vect{k}}
\newcommand\vl{\vect{l}}
\newcommand\vm{\vect{m}}
\newcommand\vn{\vect{n}}
\newcommand\vo{\vect{o}}
\newcommand\vp{\vect{p}}
\newcommand\vq{\vect{q}}
\newcommand\vr{\vect{r}}
\newcommand\vs{\vect{s}}
\newcommand\vt{\vect{t}}
\newcommand\vu{\vect{u}}
\newcommand\vv{\vect{v}}
\newcommand\vw{\vect{w}}
\newcommand\vx{\vect{x}}
\newcommand\vy{\vect{y}}
\newcommand\vz{\vect{z}}
\newcommand\vzero{\vect{0}}
\newcommand\vone{\vect{1}}

\newcommand\valpha{\vect{\alpha}}
\newcommand\vbeta{\vect{\beta}}
\newcommand\veps{\vect{\epsilon}}
\newcommand\vdelta{\vect{\delta}}
\newcommand\vtheta{\vect{\theta}}
\newcommand\vsigma{\vect{\sigma}}
\newcommand\vpi{\vect{\pi}}
\newcommand\vlambda{\vect{\lambda}}

% matrix variables (taken from macros by Rainer Gemulla)
\newcommand\mA{\vect{A}}
\newcommand\mB{\vect{B}}
\newcommand\mC{\vect{C}}
\newcommand\mD{\vect{D}}
\newcommand\mE{\vect{E}}
\newcommand\mF{\vect{F}}
\newcommand\mG{\vect{G}}
\newcommand\mH{\vect{H}}
\newcommand\mI{\vect{I}}
\newcommand\mJ{\vect{J}}
\newcommand\mK{\vect{K}}
\newcommand\mL{\vect{L}}
\newcommand\mM{\vect{M}}
\newcommand\mN{\vect{N}}
\newcommand\mO{\vect{O}}
\newcommand\mP{\vect{P}}
\newcommand\mQ{\vect{Q}}
\newcommand\mR{\vect{R}}
\newcommand\mS{\vect{S}}
\newcommand\mT{\vect{T}}
\newcommand\mU{\vect{U}}
\newcommand\mV{\vect{V}}
\newcommand\mW{\vect{W}}
\newcommand\mX{\vect{X}}
\newcommand\mY{\vect{Y}}
\newcommand\mZ{\vect{Z}}
\newcommand\mzero{\vect{0}}

\newcommand{\mPi}{{\ensuremath{\vect{\Pi}}}}
\newcommand{\mSigma}{{\ensuremath{\vect{\Sigma}}}}
\newcommand{\mLambda}{{\ensuremath{\vect{\Lambda}}}}

% argmin, argmax
\DeclareMathOperator*{\argmin}{argmin} % amsmath package required
\DeclareMathOperator*{\argmax}{argmax} % amsmath package required

% matrix operations
\newcommand\xdiag{\operatorname{diag}}      
\newcommand\diag[1]{\xdiag\left(#1\right)}    % diagonal matrix


% new commands
\newcommand\op[1]{\operatorname{#1}}

% header and footer
\lhead{Advanced Methods in Text Analytics, FSS 2025}
\chead{}
\rhead{\thepage\ }
\cfoot{}
\pagestyle{fancy}

\title{Advanced Methods in Text Analytics \\ 
Exercise 6: Transformers - Part 2}
\author{Daniel Ruffinelli}
\date{FSS 2025}

\begin{document}
\maketitle

\section{The Residual Stream}

In this task, we take a closer look at both the forward and backward pass of a
\href{https://arxiv.org/pdf/1706.03762}{\underline{transformer}} layer.
By \emph{transformer layer}, we mean the combination of a multi-head
self-attention followed by a feed-forward neural network, each with layer
normalization applied after it, and each with a residual connection around them.
For example, the encoder block in Figure 1 on the transformers paper linked
above.

\begin{enumerate}[label=(\alph*)]
    \item Let $\op{SA}$ be a self-attention layer parameterized by
          $\mW^Q,\mW^K,\mW^V$, and let $\mX\in \bb{R}^{n\times d}$ be the
          $d$-dimensional tokens from an input sequence of length $n$.
          Give a formal expression for the output of $SA$ given $\mX$ as input.
    \item Let $\op{FNN}$ be the feed-forward neural network applied after
          $\op{SA}$ in a transformer layer, and assume it projects the vectors
          in input $\mX$ to a $m$-dimensional space and then back down to a
          $d$-dimensional space.
          Give a formal expression for the output of $\op{FNN}$ given $\mX$ as
          input.
          Assume a $ReLU$ activation function is used after the first
          projection, and make sure to specify the size of the parameters in
          $\op{FNN}$.
    \item Using $\op{SA}$ and $\op{FNN}$, give an expression for the output of a
          transformer layer $\op{TF}$ given $\mX$ as input.
          Use $\op{LN_1}$ and ${LN_2}$ for layer normalization after $\op{SA}$
          and $\op{FNN}$, respectively.
    \item What do the layer normalization operators $LN_i$ do? And could we use
          them in a different part of the transformer layer?
    \item Let us now focus on the backward pass.
          Recall that during backpropagation, we apply the chain rule of
          calculus to compute the gradients of all parameterized operators used
          in the forward pass, usually w.r.t. some loss function $L$.
          In the context of some transformer layer $\op{TF_i}$, these operators
          are $\op{SA},\op{FNN},\op{LN_1}$ and $\op{LN_2}$.
          Give a formal expression for the gradient needed to update parameters
          $\theta_{SA}$ of operator $\op{SA}$ w.r.t.\ some loss function $L$.
          Omit the use of layer normalization and residual connections for
          simplicity.
          Indicate which part of the expression is ``received'' from the
          $\op{FNN}$ layer above during backpropagation.
    \item Following your previous answer, what is the gradient that is passed
          down to lower layers, e.g.\ the lower transformer layer
          $\op{TF_{i-1}}$?
          Give a formal expression for this gradient and briefly describe why
          this gradient is needed during training.
    \item Now rewrite your previous answer while considering the residual
          connection around $\op{SA}$.
          Simplify the resulting expression as much as possible with the goal of
          answering the question: what happens during training to the gradient
          that we received from higher layers as it's passed through an
          $\op{SA}$ operator that uses residual connection?
          \textbf{Hint:} No need to compute any gradients, but do use the fact
          that the gradient is a linear operator.
\end{enumerate}

\section{Attention Heads are Independent and Additive}

In this task, we take a closer look at the multi-head attention mechanism.

\begin{enumerate}[label=(\alph*)]
    \item Let $\op{MHA}$ be a multi-head attention layer, $\op{SA}_i$ its
          $i$-th self-attention layer and $\mX\in \bb{R}^{n\times d}$ the
          $d$-dimensional tokens from an input sequence of length $n$.
          Give a formal expression for the output of $\op{MHA}$ as a function of
          its $k$ attention heads $SA_i$, i.e. $1 <= i <= k$, given $\mX$ as
          input.
          As usual, make sure to formally define all components in your
          expression.
          How is $\op{MHA}$ parameterized?
    \item Assume MHA takes as input the output $\mO^i$ of each self-attention
          head $\op{SA}_i(\mX)$, i.e.\ $\mO^i = \op{SA}_i(\mX)$.
          Give a formal expression for the computations needed to compute a 
          single component of the output matrix of $\op{MHA}$ as a function of
          each $\mO^i$.
    \item Using the intuition built in the previous subtask, show that the
          output of the multi-head attention layer $\op{MHA}$ can be expressed 
          as the following linear combination:
          \begin{align}
              \op{MHA}(\mX) & = \sum_{i=1}^{k} \mO^i\mM^i + \mX,
          \end{align}
          and make sure to define exactly what each $\mM^i$ is. 
          What does this rewriting of the multi-head attention layer imply about
          the operations computed by each attention head?
\end{enumerate}

\section{Language Models with Transformers}

In this exercise, we have a quick look at how to implement a language model with
transformers using \href{https://pytorch.org/}{PyTorch}.
As before with our FNN-based and RNN-based language models, we will need to
define a class for the model, a training loop and an evaluation loop.
In addition, we also need to implement positional encodings.
We use the ones used by the original transformer architecture.

\begin{enumerate}[label=(\alph*)]
    \item Read the code for the class \emph{PositionalEncoding} to make sure
          it's correct w.r.t.\ how these positional embeddings are defined.
          Then check documentation for PyTorch's
          \href{https://pytorch.org/docs/stable/generated/torch.nn.Module.html}{Module}.
          What does the method \emph{register\_buffer} do?
    \item Read the code for the TransformerModel class. What kind of
          architecture does it use? Encoder-decoder? Encoder-only?
    \item What is the role of the function
          \emph{\_generate\_square\_subsequent\_mask}? Read the documentation
          for the
          \href{https://pytorch.org/docs/stable/generated/torch.nn.TransformerEncoder.html}{Transformer Class}
          to find out more.
    \item What kind of language model is it? Causal or masked? Why?
          \textbf{Hint:} What does the function triu do?
    \item Complete the \emph{forward} function of your transformer.
    \item We can train this transformer in the same way that we trained our RNN
          in our last coding exercise: with \emph{teacher forcing}.
          So, we will use the same methods for preprocessing the data, creating
          the batches and training the model.
          Note that the training method was modified slightly to accomodate this
          model's forward function (no need to pass the hidden state as with
          RNNs).
          Use the given code to train the transformer on the \emph{Shakespeare}
          data.
          What perplexity do we get on validation data?
          Is it higher or lower than when using an RNN as before?
          Is the task comparable w.r.t. that case? Discuss.
    \item Can you modify the model so it performs better? Consider using
          different embedding sizes, different number of layers, etc.
\end{enumerate}

\end{document}
