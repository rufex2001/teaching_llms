\documentclass[11pt,a4paper]{article}

% packages
\usepackage[utf8]{inputenc}
\usepackage{amsmath}
\usepackage[T1]{fontenc}
\usepackage{setspace}
\usepackage{enumitem}
\usepackage{booktabs}
\usepackage{fullpage} 
\usepackage{tabularx}
\usepackage{amssymb,amstext,amsmath}
\usepackage{fancyhdr}
\usepackage{graphicx}
\usepackage{algorithmic}
\usepackage[ruled,vlined]{algorithm2e}
\usepackage{url}
\usepackage[colorlinks=true,bookmarks,unicode=true,pdftex,a4paper]{hyperref}
\usepackage[round]{natbib}
\usepackage[usenames,dvipsnames]{color, xcolor}
\headsep1cm

% macros
% misc
\newcommand\todo[1]{\textcolor{red}{TODO: #1}}
\newcommand\hide[1]{\textcolor{white}{#1}}

% formatting
\newcommand\bld[1]{\textbf{#1}}
\newcommand\ul[1]{\underline{#1}}
\newcommand\n[1]{\numprint{#1}}
\newcommand{\ts}{\textsuperscript}
\newcommand\red[1]{\textcolor{red}{#1}}
\newcommand\blue[1]{\textcolor{blue}{#1}}
\newcommand\link[2]{\href{#1}{\textcolor{blue}{\underline{#2}}}}

% sets
\newcommand\set[1]{\mathcal{#1}}
\newcommand\bb[1]{\mathbb{#1}}
\renewcommand\:{\colon} % for use with \sset, etc.
\newcommand{\sset}[1]{\left\{\,#1\,\right\}} % { ? }, automatic brackets
\newcommand{\ssets}[1]{\left\{#1\right\}} % {?}, automatic brackets
\newcommand{\ssetn}[1]{\{\,#1\,\}} % { ? }, normal brackets

% table formatting
% To better align bold entries in S columns (still broken)
% \usepackage{siunitx}
% \robustify\bfseries
% \newrobustcmd{\bfcell}{\bfseries}

% vector variables (taken from macros by Rainer Gemulla)
\newcommand\vect[1]{{\boldsymbol{#1}}}
\newcommand\va{\vect{a}}
\newcommand\vb{\vect{b}}
\newcommand\vc{\vect{c}}
\newcommand\vd{\vect{d}}
\newcommand\ve{\vect{e}}
\newcommand\vf{\vect{f}}
\newcommand\vg{\vect{g}}
\newcommand\vh{\vect{h}}
\newcommand\vi{\vect{i}}
\newcommand\vj{\vect{j}}
\newcommand\vk{\vect{k}}
\newcommand\vl{\vect{l}}
\newcommand\vm{\vect{m}}
\newcommand\vn{\vect{n}}
\newcommand\vo{\vect{o}}
\newcommand\vp{\vect{p}}
\newcommand\vq{\vect{q}}
\newcommand\vr{\vect{r}}
\newcommand\vs{\vect{s}}
\newcommand\vt{\vect{t}}
\newcommand\vu{\vect{u}}
\newcommand\vv{\vect{v}}
\newcommand\vw{\vect{w}}
\newcommand\vx{\vect{x}}
\newcommand\vy{\vect{y}}
\newcommand\vz{\vect{z}}
\newcommand\vzero{\vect{0}}
\newcommand\vone{\vect{1}}

\newcommand\valpha{\vect{\alpha}}
\newcommand\vbeta{\vect{\beta}}
\newcommand\veps{\vect{\epsilon}}
\newcommand\vdelta{\vect{\delta}}
\newcommand\vtheta{\vect{\theta}}
\newcommand\vsigma{\vect{\sigma}}
\newcommand\vpi{\vect{\pi}}
\newcommand\vlambda{\vect{\lambda}}

% matrix variables (taken from macros by Rainer Gemulla)
\newcommand\mA{\vect{A}}
\newcommand\mB{\vect{B}}
\newcommand\mC{\vect{C}}
\newcommand\mD{\vect{D}}
\newcommand\mE{\vect{E}}
\newcommand\mF{\vect{F}}
\newcommand\mG{\vect{G}}
\newcommand\mH{\vect{H}}
\newcommand\mI{\vect{I}}
\newcommand\mJ{\vect{J}}
\newcommand\mK{\vect{K}}
\newcommand\mL{\vect{L}}
\newcommand\mM{\vect{M}}
\newcommand\mN{\vect{N}}
\newcommand\mO{\vect{O}}
\newcommand\mP{\vect{P}}
\newcommand\mQ{\vect{Q}}
\newcommand\mR{\vect{R}}
\newcommand\mS{\vect{S}}
\newcommand\mT{\vect{T}}
\newcommand\mU{\vect{U}}
\newcommand\mV{\vect{V}}
\newcommand\mW{\vect{W}}
\newcommand\mX{\vect{X}}
\newcommand\mY{\vect{Y}}
\newcommand\mZ{\vect{Z}}
\newcommand\mzero{\vect{0}}

\newcommand{\mPi}{{\ensuremath{\vect{\Pi}}}}
\newcommand{\mSigma}{{\ensuremath{\vect{\Sigma}}}}
\newcommand{\mLambda}{{\ensuremath{\vect{\Lambda}}}}

% argmin, argmax
\DeclareMathOperator*{\argmin}{argmin} % amsmath package required
\DeclareMathOperator*{\argmax}{argmax} % amsmath package required

% matrix operations
\newcommand\xdiag{\operatorname{diag}}      
\newcommand\diag[1]{\xdiag\left(#1\right)}    % diagonal matrix


% new commands
\newcommand\op[1]{\operatorname{#1}}

% header and footer
\lhead{Advanced Methods in Text Analytics, FSS 2025}
\chead{}
\rhead{\thepage\ }
\cfoot{}
\pagestyle{fancy}

\title{Advanced Methods in Text Analytics \\ 
Exercise 7: Large Language Models - Part 1}
\author{Daniel Ruffinelli}
\date{FSS 2025}

\begin{document}
\maketitle

\section{Transfer Learning}

In this task, we discuss the basic concepts behind transfer learning.

\begin{enumerate}[label=(\alph*)]
    \item What do we mean when we say that a model is ``pre-trained''?
          Why don't we say the model is simply ``trained''?
          What type of data do we need for pre-training a language model?
    \item What are the two more common training objectives for pre-training
          large language models?
    \item Let $S$ be the set of sequences used for training language model $p$.
          Using $p$ and $S$, give formal definitions for the training objectives
          mentioned in your previous answer.
    \item What type of data should we use for pre-training an LLM? What factors
          should we consider when choosing what data to train on?
    \item We typically talk of fine-tuning a pre-trained model. What do we mean
          by fine-tuning? How is it different from ``pre-training'' a model?
          What type of data do we need for fine-tuning a language model?
    \item What is the main challenge behind traning models with a large number
          of parameters?
\end{enumerate}

\section{Parameter Efficient Fine-Tuning}

Given the large number of parameters in LLMs, a set of methods have focused on
fine-tuning large-size models by tuning only a subset of their parameters.
In this task, we go over some of the concepts behind such parameter efficient
fine-tuning (PEFT) methods.

\begin{enumerate}[label=(\alph*)]
    \item What is the main idea behind parameter-efficient fine-tuning (PEFT)?
          Briefly describe two different PEFT methods.
    \item Let $\mX = \{\vx_1,\ldots,\vx_n\}$ be the input sequence and
          $\mH = \{\vh_1,\ldots, \vh_n\}$ the corresponding output sequence of
          the FNN operator in a transformer block, where $\op{FNN}(\mX)$ is
          parameterized by $\mW_{up}\in\bb{R}^{d\times 4d}$ and
          $\mW_{down}\in\bb{R}^{4d\times d}$, i.e.\ the up and down projections
          in FNN, respectively.
          In other words, $\mH = \op{FNN}(\mX\mid \mW_{up}, \mW_{down})$ (note
          that we omit biases and the residual connection for simplicity).
          Give a formal expression for the output of this FNN operator.
    \item Following the previous subtask, assume we fine-tune the FNN operator
          using the Houlsby adapters introduced in the lectures, and call this
          fine-tuned operator FNN'.
          Give a formal expression for FNN' given input sequence $\mX$ (don't
          use any biases in the adapter).
          Make sure to formally define all parameters introduced by the adapter
          as well as their exact sizes.
          What are the hyperparameters used in these adapters? How are these
          hyperparameters typically set?
    \item Can Houlsby adapters only be applied to FNN operators? Why or why not?
    \item Now assume we fine-tune \emph{only} the up projection of the FNN
          operator using LoRA adapters.
          Give a formal expression for FNN' given input sequence $\mX$ (again,
          no biases in the adapter).
          Make sure to formally define all parameters introduced by LoRA as well
          as their exact sizes.
          As before, specify the hyperparameters used in LoRA adapters and
          say how they are typically set.
    \item Can LoRA adapters only be applied to FNN operators? Why or why not?
    \item If you haven't already, show that the application of LoRA adapters
          can be expressed as follows:
          \begin{align}
              \op{FNN'}(\mX) & = \mX(\mW_{up} + \Delta\mW_{up})\mW_{down},
          \end{align}
          where $\Delta\mW_{up}$ is the update introduced during fine-tuning.
          What is the advantage of this formulation?
          Can Houlsby adapters be expressed in a similar way?
\end{enumerate}

\section{Transformer-Based Large Language Models}

In this task, we discuss the main components of a transformer-based large
language model (LLM).
In particular, we focus on a causal language model (CLM) similar to the GPT-3
model released by~\href{https://arxiv.org/pdf/2005.14165}{Brown et al. (2020)}.

\begin{enumerate}[label=(\alph*)]
    \item Why do transformer-based language models require positional encodings?
          Given a language model with $L$ transformer layers, how many
          positional encoding layers does this model use?
    \item What is the role of the language model head in a transformer-based
          language model?
          What sort of ``weight tying'' is typically used with the language
          model head?
    \item As done in previous exercises, give a formal expression for the
          output of a multi-head attention operator $\op{MHA}$ that uses $n$
          self-attention heads $\op{SA}_i$?
          What are the parameters $\vtheta_{\op{MHA}}$ assuming each $\op{SA}_i$
          operator is parameterized by $\vtheta_{\op{SA}_i}$?
    \item Assume you have a pre-trained causal language model $\op{CLM}$
          parameterized by $\vtheta_{CLM}$.
          The model is implemented with transformers and follows the GPT-3
          architecture, meaning the size of the input and output representations
          in each transformer layer is $d_H = 12288$.
          If queries, keys and values in each self-attention head are all
          projections of the input representations to dimension
          $d_{\op{SA}_i} = 128$ and the projection \emph{inside} the multi-head
          attention operator $\op{MHA}$ does not change the dimension of its
          inputs, how many attention heads does each $\op{MHA}$ operator have?
          Give a numeric answer and show the calculations that lead to your
          answer.
          \textbf{Hint:} remember that the size of the output representations of
          a $\op{MHA}$ operator are also the same size as its input
          representations.
    \item Given that the model above follows the GPT-3 architecture, it has
          $L = 96$ transformer layers, a maximum input sequence length of
          $2048$, and a vocabulary size of $V = 50000$.
          How many parameters does this CLM model have?
          I.e.\ what is the size of $\vtheta_{\op{CLM}}$?
          And which component has the most parameters?
          Give numeric answers and show the calculations that lead you to them.
          To answer the question, you may ignore all biases and layer
          normalization operators.
          In addition, assume positional embeddings are not learned, weight
          tying takes place in the language model head, and that the $\op{FNN}$
          layers inside each transformer layer project representations up to a
          space 4 times the size of the input and then back down (see forward
          pass of transformer layer in Exercise 6).
\end{enumerate}


\end{document}
