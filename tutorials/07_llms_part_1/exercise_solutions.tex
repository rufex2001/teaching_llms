\documentclass[11pt,a4paper]{article}

% packages
\usepackage[utf8]{inputenc}
\usepackage{amsmath}
\usepackage[T1]{fontenc}
\usepackage{setspace}
\usepackage{enumitem}
\usepackage{amsmath}
\usepackage{booktabs}
\usepackage{fullpage} 
\usepackage{tabularx}
\usepackage{amssymb, amstext, amsmath}
\usepackage{fancyhdr}
\usepackage{graphicx}
\usepackage{algorithmic}
\usepackage[ruled,vlined]{algorithm2e}
\usepackage{url}
\usepackage[bookmarks,unicode=true,pdftex,a4paper]{hyperref}
\usepackage[round]{natbib}
\usepackage[usenames,dvipsnames]{color, xcolor}
\headsep1cm

% macros
% misc
\newcommand\todo[1]{\textcolor{red}{TODO: #1}}
\newcommand\hide[1]{\textcolor{white}{#1}}

% formatting
\newcommand\bld[1]{\textbf{#1}}
\newcommand\ul[1]{\underline{#1}}
\newcommand\n[1]{\numprint{#1}}
\newcommand{\ts}{\textsuperscript}
\newcommand\red[1]{\textcolor{red}{#1}}
\newcommand\blue[1]{\textcolor{blue}{#1}}
\newcommand\link[2]{\href{#1}{\textcolor{blue}{\underline{#2}}}}

% sets
\newcommand\set[1]{\mathcal{#1}}
\newcommand\bb[1]{\mathbb{#1}}
\renewcommand\:{\colon} % for use with \sset, etc.
\newcommand{\sset}[1]{\left\{\,#1\,\right\}} % { ? }, automatic brackets
\newcommand{\ssets}[1]{\left\{#1\right\}} % {?}, automatic brackets
\newcommand{\ssetn}[1]{\{\,#1\,\}} % { ? }, normal brackets

% table formatting
% To better align bold entries in S columns (still broken)
% \usepackage{siunitx}
% \robustify\bfseries
% \newrobustcmd{\bfcell}{\bfseries}

% vector variables (taken from macros by Rainer Gemulla)
\newcommand\vect[1]{{\boldsymbol{#1}}}
\newcommand\va{\vect{a}}
\newcommand\vb{\vect{b}}
\newcommand\vc{\vect{c}}
\newcommand\vd{\vect{d}}
\newcommand\ve{\vect{e}}
\newcommand\vf{\vect{f}}
\newcommand\vg{\vect{g}}
\newcommand\vh{\vect{h}}
\newcommand\vi{\vect{i}}
\newcommand\vj{\vect{j}}
\newcommand\vk{\vect{k}}
\newcommand\vl{\vect{l}}
\newcommand\vm{\vect{m}}
\newcommand\vn{\vect{n}}
\newcommand\vo{\vect{o}}
\newcommand\vp{\vect{p}}
\newcommand\vq{\vect{q}}
\newcommand\vr{\vect{r}}
\newcommand\vs{\vect{s}}
\newcommand\vt{\vect{t}}
\newcommand\vu{\vect{u}}
\newcommand\vv{\vect{v}}
\newcommand\vw{\vect{w}}
\newcommand\vx{\vect{x}}
\newcommand\vy{\vect{y}}
\newcommand\vz{\vect{z}}
\newcommand\vzero{\vect{0}}
\newcommand\vone{\vect{1}}

\newcommand\valpha{\vect{\alpha}}
\newcommand\vbeta{\vect{\beta}}
\newcommand\veps{\vect{\epsilon}}
\newcommand\vdelta{\vect{\delta}}
\newcommand\vtheta{\vect{\theta}}
\newcommand\vsigma{\vect{\sigma}}
\newcommand\vpi{\vect{\pi}}
\newcommand\vlambda{\vect{\lambda}}

% matrix variables (taken from macros by Rainer Gemulla)
\newcommand\mA{\vect{A}}
\newcommand\mB{\vect{B}}
\newcommand\mC{\vect{C}}
\newcommand\mD{\vect{D}}
\newcommand\mE{\vect{E}}
\newcommand\mF{\vect{F}}
\newcommand\mG{\vect{G}}
\newcommand\mH{\vect{H}}
\newcommand\mI{\vect{I}}
\newcommand\mJ{\vect{J}}
\newcommand\mK{\vect{K}}
\newcommand\mL{\vect{L}}
\newcommand\mM{\vect{M}}
\newcommand\mN{\vect{N}}
\newcommand\mO{\vect{O}}
\newcommand\mP{\vect{P}}
\newcommand\mQ{\vect{Q}}
\newcommand\mR{\vect{R}}
\newcommand\mS{\vect{S}}
\newcommand\mT{\vect{T}}
\newcommand\mU{\vect{U}}
\newcommand\mV{\vect{V}}
\newcommand\mW{\vect{W}}
\newcommand\mX{\vect{X}}
\newcommand\mY{\vect{Y}}
\newcommand\mZ{\vect{Z}}
\newcommand\mzero{\vect{0}}

\newcommand{\mPi}{{\ensuremath{\vect{\Pi}}}}
\newcommand{\mSigma}{{\ensuremath{\vect{\Sigma}}}}
\newcommand{\mLambda}{{\ensuremath{\vect{\Lambda}}}}

% argmin, argmax
\DeclareMathOperator*{\argmin}{argmin} % amsmath package required
\DeclareMathOperator*{\argmax}{argmax} % amsmath package required

% matrix operations
\newcommand\xdiag{\operatorname{diag}}      
\newcommand\diag[1]{\xdiag\left(#1\right)}    % diagonal matrix


% new commands
\newcommand\op[1]{\operatorname{#1}}

% header and footer
\lhead{Advanced Methods in Text Analytics, FSS 2025}
\chead{}
\rhead{\thepage\ }
\cfoot{}
\pagestyle{fancy}

\title{Advanced Methods in Text Analytics \\ 
Exercise 7: Large Language Models - Part 1\\
\textbf{Solutions}}
\author{Daniel Ruffinelli}
\date{FSS 2025}

\begin{document}
\maketitle

\section{Transfer Learning}

\begin{enumerate}[label=(\alph*)]
    \item A model is said to be \emph{pre-trained} when the goal of the training
          process is to learn \emph{general} representations of the input data,
          without a specific task in mind where these representations could be
          used.
          In that sense, the prefix ``pre'' indicates that the model
          is trained \emph{before} being fine-tuned on a specific task.
          In the context of LLMs, fine-tuning is not done as frequently as in
          the era of pre-trained language models (PLMs), e.g. BERT.
          So we can speak of pre-training as the training of a model before it
          is actually used on a different set of (downstream) tasks.
    \item Pre-training usually involves training a model on a corpus of data
          using a \emph{self-supervised} learning objective.
          The most common ones are: (i) causal language modeling (CLM) and (ii)
          masked language modeling (MLM).
    \item The causal language modeling (CLM) objective is defined as follows:
          \begin{align}
              \op{CLM}(S) & = \prod_{s=1}^{S} p(x_{|s|} \mid x_1, x_2, \ldots, x_{|s|-1}),
          \end{align}
          where $x_i$ are tokens in sequence $s$.
          Similarly, the masked language modeling (MLM) objective is defined as
          follows:
          \begin{align}
              \op{MLM}(S) & = \prod_{s=1}^{S} \prod_{m=1}^{M_s} p(x_m \mid x_1, x_2, \ldots, x_{|s|} \setminus M),
          \end{align}
          where $M_s$ is set of masked out tokens in sequence $s$ and
          $A \setminus B$ is set difference.
          In practice, we use empirical risk minimization with log loss during
          training. That is,
          \begin{align}
              \op{CLM}(S) & = - \frac{1}{|S|} \sum_{s=1}^{S} \log p(x_{|s|} \mid x_1, x_2, \ldots, x_{|s|-1}),
          \end{align}
          \begin{align}
              \op{MLM}(S) & = - \frac{1}{|M|} \sum_{s=1}^{S} \sum_{m=1}^{M_s} \log p(x_m \mid x_1, x_2, \ldots, x_{|s|} \setminus M),
          \end{align}
          where $M = \{M_1 \cup M_2 \cup \ldots \cup M_{|S|}\}$ is the set of 
          all masked tokens in $S$.
    \item The main factors are: (i) size, (ii) quality, and (iii) domain.
          \textbf{Size.} The data should be large enough to capture a wide range
          of linguistic patterns and concepts, allowing the model to learn a
          rich set of representations for the language.
          It should also be large enough to avoid overfitting, easily done with
          a large number of parameters.

          \textbf{Quality}. The quality is also important, but it can be
          measured in different ways.
          We can speak of quality in terms of grammatical correctness, or in
          terms of lexical variety, etc.
          What is good quality usually depends on the goal we have for the
          model.
          E.g.\ if the goal is to train a model for speaking formal English,
          then University textbooks are likely a high quality source.
          But if the goal in on colloquial English, then online forums are
          likely a better source.

          \textbf{Domain.} Pre-training data should be a diverse collection of
          text from various sources that cover a set of domains of interest,
          e.g.\ medicine, biology, pop culture, etc.
    \item Fine-tuning refers to the process of taking a pre-trained model and
          adapting it to a specific downstream task by updating its parameters
          on a smaller labeled dataset that represents this task.
          Thus, fine-tuning is different from pre-training in that pre-training
          involves training a model without a specific \emph{target} task
          in mind.

          Fine-tuning typically relies on supervised data that represents the
          downtream task of interest, e.g.\ pairs of the form
          (review, sentiment) for sentiment analysis of product reviews.
          Thus, during fine-tuning, the model should learn task-specific
          patterns and improve its performance on the downstream task, at the
          potential cost of catastrophically forgetting what was learned during
          pre-training.
    \item The main challenge behind training models with a large number of
          parameters is the computational cost and memory requirements that come
          with the training process.
          To reduce runtime costs, specialized hardware is commonly used for
          training models, e.g.\ GPUs that are more efficient at computing
          common operations used during training, such as matrix products.
          In addition, large models can be memory-intensive, requiring large
          amounts of memory just to store the model parameters and intermediate
          activations during training (as computed soon in Task 3).
\end{enumerate}

\section{Parameter Efficient Fine-Tuning}

\begin{enumerate}[label=(\alph*)]
    \item Parameter-efficient fine-tuning (PEFT) method aim to reduce the
          computational cost and memory requirements of fine-tuning large
          language models by updating only a subset of the parameters in the
          pre-trained model.

          One common approch is the use of adapters, which are small
          parameterized components added to the pre-trained architecture,
          typically a transformer-based LM, which modify the model's forward
          pass.
          Then, during fine-tuning, only the parameters of the adapters are
          updated, while the parameters of the pre-trained model are kept
          \emph{frozen}.
          Adapters can be placed anywhere in the model's architecture and thus
          have different impacts on the forward pass.
          In addition, adapters can themselves be somewhat involved, e.g.\ by
          including an attention mechanism.

          Another approach to PEFT is \bld{prompt tuning}, which adds a
          small number of task-specific tokens to the input sequence during
          fine-tuning.
          These tokens are used to guide the model towards the task of
          interest, e.g.\ by specifying the task in the promp given to the
          model, e.g. ``summarize this:''
          During fine-tuning, only these special tokens are updated, giving
          the model a small set of parameters that can be used to modify its
          outputs to better fit the downstream task of interest.
    \item We have:
          \begin{align}
              \op{FNN}(\mX) & = f(\mX\mW_{up})\mW_{down},
          \end{align}
          where $f$ is a non-linear activation function, originally ReLU,
          lately different ones like SwiGLU.
    \item Let $\mA \in \bb{R}^{d\times r}$, $\mB \in R^{r\times d}$ be the
          projection matrices in the adapter.
          Then:
          \begin{align}\label{houlsby}
              \op{FNN'}(\mX) & =  \op{FNN}(\mX) + g(\op{FNN}(\mX)\mA)\mB,
          \end{align}
          where $g$ is a non-linear activation function (multiple ones were
          originally tested).
          That is, Houlsby adapters are applied sequentially after the operator
          being fine-tuned, and they include a residual connection as with the
          other components in the transformer block (hence the addition).
          Thus, Houslby adapters effectively add a new \emph{sequential}
          component to each transformer block.
          Each adapter has $2dr$ number of parameters, where $r$ is
          a hyperparameter, the bottleneck dimension used to control the
          size/capacity of the adapter.
          Typically, $r << d$ to keep the number of trainable parameters low.
    \item Houlsby adapters can be placed after any layer or sublayer in the
          model. In fact, Houlsby originally added one adapter after the
          multi-head attention operator and another after the FNN operator in
          each transformer block.
    \item Let $\mA \in \bb{R}^{d\times r}$, $\mB \in R^{r\times d}$ be the
          projection matrices in the LoRA adapter.
          Then:
          \begin{align}\label{lora}
              \op{FNN'}(\mX) & = (\mX\mW_{up} + \mX\mA\mB)\mW_{down}
          \end{align}
          The addition comes from the LoRA adapter, which is applied using a
          residual connection around the weight matrix we are fine-tuning.
          Each adapter has $2dr$ number of parameters, where $r << d$ is
          the hyperparameter that controls the rank of both projection matrices
          used by LoRA adapters.
    \item LoRA adapters can be applied to any projection matrix, e.g.\ the
          projections used in self-attention layers. This should be clear from
          the fact that in the FNN operator, we applied a LoRA adapter only to
          one of the two possible projection matrices.
          In fact, in the original paper, LoRA adapters were applied to matrices
          $\mW^Q$ and $\mW^V$ in self-attention layers.
    \item This is easy to see from Eq.~\ref{lora} and the fact
          that matrix products are distributive w.r.t.\ matrix sum.
          That is,
          \begin{align}
              \op{FNN'}(\mX) & = (\mX\mW_{up} + \mX\mA\mB)\mW_{down} = \mX(\mW_{up} + \mA\mB)\mW_{down}
          \end{align}
          where $\Delta\mW_{up} = \mA\mB$.

          The advantage of this formulation is that it allows us to see that
          the learned adapters can be added back into the original model after
          learning is done. This results in a fine-tuned model without an
          increase in number of parameters and without the corresponding
          additional runtime costs in the forward pass.
          In addition, it allows for modularity, as not only can we choose which
          matrices to specifically tune, but it allows us to remove the effect
          of the adapters from the model by simply subtracting $\mA\mB$ from the
          tuned weight matrix to recover the original model.

          Such advantages are not immediately possible with Houlsby adapters.
          For one, they were designed to behave similarly to the sublayers in a
          transformer block, so it's not clear what impact they would have when
          applied to, e.g., matrix $\mW^K$ in a self-attention layer.
          In addition, Houlsby adapters make changes to the residual space using
          linear projections that include a non-linear activation function in
          between, so if we want to remove the impact of these adapters to
          recover the original model, we would have to ensure the non-linear
          activation function has an inverse function and that the learned
          projections are invertible.

\end{enumerate}

\section{Transformer-Based Large Language Models}

\begin{enumerate}[label=(\alph*)]
    \item Transformer-based language models require positional encodings to
          provide the model with information about the position of each token in
          the input sequence.
          This is because the transformer architecture does not have any
          built-in mechanism to understand the order of the tokens in the input
          sequence, unlike RNNs which process tokens sequentially, thus
          naturally having a ``sense of time''.
          The positional encodings are \emph{typically} added to the input
          embeddings before feeding them into the first transformer layer.
          Thus, a language model with $L$ transformer layers uses a single
          positional encoding layer, independently of the number of transformer
          layers in the model.
          However, some positional encoding layers are added to each transformer 
          layer.
    \item The language model head in a transformer-based language model is
          responsible for producing the distribution used for predicting the
          next token given an input sequence.
          It is a softmax layer that projects the contextualized representations
          of the input tokens to the vocabulary space and applies the softmax
          function to normalize the logit scores into a probability
          distribution.

          The weights of the language model head are typically tied to the
          input embeddings, because both the static embedding matrix and the
          projection matrix in the language model head are of the same size.
          This means that the weights of the language model head are the same
          as the weights of the input embeddings.
          This weight tying helps in reducing the number of parameters in the
          model and improves the generalization of the model.
    \item The output of a multi-head attention operator $\op{MHA}$ that uses $n$
          self-attention heads $\op{SA}_i$ is the concatenation of the outputs
          of each head.
          To ensure the size of the output is the same as the input, these
          concatenated outputs are projected to the same size as the input.
          That is,
          \begin{align}\label{eq:mha}
              \op{MHA} = (\op{SA}_1 \oplus \op{SA}_2 \oplus \ldots \oplus \op{SA}_n)\mW^O,
          \end{align}
          where $\oplus$ denotes the concatenation operation.
          The parameters of the multi-head attention operator are the
          parameters of each attention head plus the output projection matrix
          $\mW^O$.
          In other words,
          \begin{align}
              \vtheta_{\op{MHA}} = [\vtheta_{\op{SA}_1}, \vtheta_{\op{SA}_2}, \ldots, \vtheta_{\op{SA}_n}, \mW^O].
          \end{align}
    \item Let's start with what we know.
          \begin{enumerate}[label=(\roman*)]
              \item The size of query, key and value vectors in each
                    self-attention layer $\op{SA}_i$ are each of size $128$.
                    Specifically, the size of the value vectors tells us that
                    the output of each self-attention head is of size $128$.
              \item The size of the output representations of a $\op{MHA}$
                    operator are the same size as the input representations, in
                    this case $d_H = 12288$ (the hint).
              \item The $\op{MHA}$ operation is given by Eq.~\ref{eq:mha}.
              \item The projection inside the multi-head attention operator,
                    i.e.\ $\mW^O$, does not change the dimension of its inputs.
          \end{enumerate}
          The first three points above tell us that
          $\mW^O\in\bb{R}^{c\times 12288}$, where $c$ is the size of the
          concatenations of the outputs of each self-attention head, i.e.\
          $c = n\cdot 128$, where $n$ is the number of attention heads we are
          looking for.
          But the fourth point above tells us that $\mW^O$ is a square matrix,
          which means $c = 12288$.
          Thus, we have:
          \begin{align}
              n = \frac{12288}{128} = 96.
          \end{align}
          The largest variant of GPT-3 has $96$ attention heads in each
          multi-head attention operator, and the projection inside the
          multi-head attention indeed does not change the dimension of the
          concatenated outputs of each attention head.
          This means that this projection could be easily dropped from each
          multi-head attention layer without affecting the forward pass in this
          case.
          But this projection is nevertheless still used, likely because the
          large number of parameters is useful for the model.
    \item The components that such a $\op{CLM}$ has are:
          \begin{enumerate}[label=(\roman*)]
              \item a static embedding matrix followed by
              \item a positional embeddings layer followed by
              \item a stack of $L$ transformer layers followed by
              \item a language modeling head $lm\_head$ applied after the last
                    transformer layer.
          \end{enumerate}
          Let's count the number of parameters in each component, ignoring
          all biases and layer normalization operators, and knowing positional
          embeddings have no parameters.

          First, $lm\_head$ is parameterized by
          $\vtheta_{lm\_head} = \mW_{lm\_head}\in\bb{R}^{d_H\times V}$.
          Given that we have weight tying, this matrix will also serve as our
          static embedding matrix.
          Second, each transformer layer is a multi-head attention layer
          $\op{MHA}$ followed by an $\op{FNN}$ layer, which projects the
          representations to some m-dimensional space, and then back down to the
          original input dimension (see Exercise 6 Task 1.b).
          So, the total number of parameters in such an architecture would be:
          \begin{align}
              |\vtheta_{CLM}| = \sum_{i=1}^{L} \left(|\vtheta_{\op{MHA}_i}| + |\vtheta_{\op{FNN}_i}|\right) + |\vtheta_{lm\_head}|.
          \end{align}
          where $\vtheta_{\op{MHA}_i}$ and $\vtheta_{\op{FNN}_i}$ are the
          parameters in the multi-head attention and FNN components of the
          $i$-th transformer layer, respectively.

          Let's now compute the number of parameters in each component.
          \begin{itemize}
              \item $\vtheta_{\op{SA}} = [\mW^Q,\mW^K,\mW^V$], each of size
                    $12288\times 128$, so that
                    $|\vtheta_{\op{SA}_i}| = 3\cdot 12288\cdot 128 = 4718592 \approx 4.7M$.
              \item $|\vtheta_{\op{MHA}_i}| = (96\times |\vtheta_{\op{SA}}|) + |\mW^O|$,
                    and since $|\mW^O| = 12288^2$, we have that
                    $|\vtheta_{\op{MHA}_i}| = (96\times 4718592) + 12288^2 = 452984832 + 150994944 = 603979776 \approx 604M$.
              \item $|\vtheta_{\op{FNN}_i}| = 2\times \left(12288\times (12288\times 4)\right) = 12288^2\cdot 8 = 1207959552 \approx 1.2B$.
              \item $|\vtheta_{lm\_head}| = d_H \times V = 12288 \times 50000 = 614400000 \approx 614M$.
          \end{itemize}
          All together, we have:
          \begin{align}
              |\vtheta_{CLM}| = 96\cdot (604M + 1.2B) + 614M = 174550155264 \approx 175B.
          \end{align}
          Indeed, the largest variant of GPT-3 has around $175$ billion
          parameters.
          The projection by the $\op{FNN}$ operator does indeed project up to
          a space four times the size in the original transformers, and has
          remained common practice ever since, e.g.\ in the GPT-3 architecture.

          Of course, most of the parameters come from the stack of transformer
          layers, but note that two thirds of those parameters come from the
          $\op{FNN}$ operators in the transformer layers, as each of these takes
          has twice the parameters of a each multi-head attention operator.
          Now, assuming
          \link{https://en.wikipedia.org/wiki/Bfloat16_floating-point_format}{bfloat16},
          which is a 16-bit floating-point format commonly used in LLMs, we
          would need around $350$GB of memory just to store this model.
          To train it, you need to further store the gradients of each of those
          parameters during the backward pass.
          This is why it's a massive engineering effort is needed so that these
          models can be trained in a distributed manner over thousands of GPUs.
\end{enumerate}


\end{document}
